\documentclass[11pt]{scrartcl}

\usepackage[sexy]{evan}
\usepackage{pgfplots}
\pgfplotsset{compat=1.15}
\usepackage{mathrsfs}
\usetikzlibrary{arrows}
\usepackage{graphics}
\usepackage{listings}
\usepackage{tikz}
\usepackage{ amssymb }
\usepackage[dvipsnames]{xcolor}
\definecolor{red1}{RGB}{255, 153, 153}
\definecolor{green1}{RGB}{204, 255, 204}
\definecolor{blue1}{RGB}{204, 255, 255}
\definecolor{yellow1}{RGB}{255, 247, 160}

\definecolor{red2}{RGB}{255, 102, 102}
\definecolor{green2}{RGB}{108, 255, 108}
\definecolor{blue2}{RGB}{94, 204, 255}
\definecolor{yellow2}{RGB}{255, 250, 104}

\definecolor{red2.5}{RGB}{255,76,76}
\definecolor{green2.5}{RGB}{54, 247, 54}
\definecolor{blue2.5}{RGB}{51, 189, 255}
\definecolor{yellow2.5}{RGB}{255, 242, 52}


\definecolor{red3}{RGB}{255, 51, 51}
\definecolor{green3}{RGB}{0, 240, 0}
\definecolor{blue3}{RGB}{9, 175, 255}
\definecolor{yellow3}{RGB}{255, 234, 0}

\definecolor{red3.5}{RGB}{229, 25, 25}
\definecolor{green3.5}{RGB}{0, 194, 0}
\definecolor{blue3.5}{RGB}{4, 143, 209}
\definecolor{yellow3.5}{RGB}{255,220,0}

\definecolor{red4}{RGB}{204, 0, 0}
\definecolor{green4}{RGB}{0, 149, 0}
\definecolor{blue4}{RGB}{0, 111, 164}
\definecolor{yellow4}{RGB}{255, 206, 0}

\newcommand{\camod}[1]{\frac{\ZZ}{#1 \ZZ}}
\newcommand{\modm}[1]{\text{ mod } #1}
\newcommand{\campm}[1]{\frac{\ZZ}{m\ZZ}}
\newcommand{\paren}[1]{\left(  #1 \right)}
\newcommand{\proc}[1]{Esto va a aparecer si no procrastino :S (Procrastinando desde #1)}


\title{SL22}
\subtitle{PONTE A ENTRENAR}
\author{Emmanuel Buenrostro}


\begin{document}

\maketitle

\section{Problemas}
\begin{problem}[IMOSL 2022 C4]
Let $n > 3$ be a positive integer. Suppose that $n$ children are arranged in a circle, and $n$ coins are distributed between them (some children may have no coins). At every step, a child with at least 2 coins may give 1 coin to each of their immediate neighbors on the right and left. Determine all initial distributions of the coins from which it is possible that, after a finite number of steps, each child has exactly one coin.
\end{problem}
\begin{problem}[IMOSL 2022 C2]
The Bank of Oslo issues two types of coin: aluminum (denoted A) and bronze (denoted B). Marianne has $n$ aluminum coins and $n$ bronze coins arranged in a row in some arbitrary initial order. A chain is any subsequence of consecutive coins of the same type. Given a fixed positive integer $k \leq 2n$, Gilberty repeatedly performs the following operation: he identifies the longest chain containing the $k^{th}$ coin from the left and moves all coins in that chain to the left end of the row. For example, if $n=4$ and $k=4$, the process starting from the ordering $AABBBABA$ would be $AABBBABA \to BBBAAABA \to AAABBBBA \to BBBBAAAA \to ...$

Find all pairs $(n,k)$ with $1 \leq k \leq 2n$ such that for every initial ordering, at some moment during the process, the leftmost $n$ coins will all be of the same type.
\end{problem}
\begin{problem}[IMOSL 2022 G3]
Let $ABCD$ be a cyclic quadrilateral. Assume that the points $Q, A, B, P$ are collinear in this order, in such a way that the line $AC$ is tangent to the circle $ADQ$, and the line $BD$ is tangent to the circle $BCP$. Let $M$ and $N$ be the midpoints of segments $BC$ and $AD$, respectively. Prove that the following three lines are concurrent: line $CD$, the tangent of circle $ANQ$ at point $A$, and the tangent to circle $BMP$ at point $B$.
\end{problem}
\begin{problem}[IMOSL 2022 N5]
	For each $1\leq i\leq 9$ and $T\in\mathbb N$, define $d_i(T)$ to be the total number of times the digit $i$ appears when all the multiples of $1829$ between $1$ and $T$ inclusive are written out in base $10$.

Show that there are infinitely many $T\in\mathbb N$ such that there are precisely two distinct values among $d_1(T)$, $d_2(T)$, $\dots$, $d_9(T)$
\end{problem}
\begin{problem}[IMOSL 2022 G2]
In the acute-angled triangle $ABC$, the point $F$ is the foot of the altitude from $A$, and $P$ is a point on the segment $AF$. The lines through $P$ parallel to $AC$ and $AB$ meet $BC$ at $D$ and $E$, respectively. Points $X \ne A$ and $Y \ne A$ lie on the circles $ABD$ and $ACE$, respectively, such that $DA = DX$ and $EA = EY$.
Prove that $B, C, X,$ and $Y$ are concyclic.
\end{problem}
\begin{problem}[IMOSL 2022 A2]
Let $k\ge2$ be an integer. Find the smallest integer $n \ge k+1$ with the property that there exists a set of $n$ distinct real numbers such that each of its elements can be written as a sum of $k$ other distinct elements of the set.
\end{problem}
\begin{problem}[IMOSL 2022 N1]
A number is called Norwegian if it has three distinct positive divisors whose sum is equal to $2022$. Determine the smallest Norwegian number.
(Note: The total number of positive divisors of a Norwegian number is allowed to be larger than $3$.)
\end{problem}
\begin{problem}[IMOSL 2022 A1]
Let $(a_n)_{n\geq 1}$ be a sequence of positive real numbers with the property that
$$(a_{n+1})^2 + a_na_{n+2} \leq a_n + a_{n+2}$$for all positive integers $n$. Show that $a_{2022}\leq 1$.
\end{problem}
\begin{problem}[IMOSL 2022 C5]
Let $m,n \geqslant 2$ be integers, let $X$ be a set with $n$ elements, and let $X_1,X_2,\ldots,X_m$ be pairwise distinct non-empty, not necessary disjoint subset of $X$. A function $f \colon X \to \{1,2,\ldots,n+1\}$ is called nice if there exists an index $k$ such that\[\sum_{x \in X_k} f(x)>\sum_{x \in X_i} f(x) \quad \text{for all } i \ne k.\]Prove that the number of nice functions is at least $n^n$.
\end{problem}
\begin{problem}[IMOSL 2022 N2]
Find all positive integers $n>2$ such that
$$ n! \mid \prod_{ p<q\le n, p,q \, \text{primes}} (p+q)$$
\end{problem}
\begin{problem}[IMOSL 2022 N4]
Find all triples $(a,b,p)$ of positive integers with $p$ prime and\[ a^p=b!+p. \]
\end{problem}
\begin{problem}[IMOSL 2022 N3]
Let $a > 1$ be a positive integer and $d > 1$ be a positive integer coprime to $a$. Let $x_1=1$, and for $k\geq 1$, define
$$x_{k+1} = \begin{cases}
x_k + d &\text{if } a \text{ does not divide } x_k \\
x_k/a & \text{if } a \text{ divides } x_k
\end{cases}$$Find, in terms of $a$ and $d$, the greatest positive integer $n$ for which there exists an index $k$ such that $x_k$ is divisible by $a^n$.
\end{problem}
\begin{problem}[IMOSL 2022 C3]
In each square of a garden shaped like a $2022 \times 2022$ board, there is initially a tree of height $0$. A gardener and a lumberjack alternate turns playing the following game, with the gardener taking the first turn:
The gardener chooses a square in the garden. Each tree on that square and all the surrounding squares (of which there are at most eight) then becomes one unit taller.
The lumberjack then chooses four different squares on the board. Each tree of positive height on those squares then becomes one unit shorter.
We say that a tree is majestic if its height is at least $10^6$. Determine the largest $K$ such that the gardener can ensure there are eventually $K$ majestic trees on the board, no matter how the lumberjack plays.
\end{problem}
\begin{problem}[IMOSL 2022 C1]
	A $\pm 1$-sequence is a sequence of $2022$ numbers $a_1, \ldots, a_{2022},$ each equal to either $+1$ or $-1$. Determine the largest $C$ so that, for any $\pm 1$-sequence, there exists an integer $k$ and indices $1 \le t_1 < \ldots < t_k \le 2022$ so that $t_{i+1} - t_i \le 2$ for all $i$, and$$\left| \sum_{i = 1}^{k} a_{t_i} \right| \ge C.$$
\end{problem}
\begin{problem}[IMOSL 2022 A5]
Find all positive integers $n \geqslant 2$ for which there exist $n$ real numbers $a_1<\cdots<a_n$ and a real number $r>0$ such that the $\tfrac{1}{2}n(n-1)$ differences $a_j-a_i$ for $1 \leqslant i<j \leqslant n$ are equal, in some order, to the numbers $r^1,r^2,\ldots,r^{\frac{1}{2}n(n-1)}$.
\end{problem}
\begin{problem}[IMOSL 2022 A3]
Let $\mathbb{R}^+$ denote the set of positive real numbers. Find all functions $f: \mathbb{R}^+ \to \mathbb{R}^+$ such that for each $x \in \mathbb{R}^+$, there is exactly one $y \in \mathbb{R}^+$ satisfying$$xf(y)+yf(x) \leq 2$$
\end{problem}
\begin{problem}[IMOSL 2022 G4]
	Let $ABC$ be an acute-angled triangle with $AC > AB$, let $O$ be its circumcentre, and let $D$ be a point on the segment $BC$. The line through $D$ perpendicular to $BC$ intersects the lines $AO, AC,$ and $AB$ at $W, X,$ and $Y,$ respectively. The circumcircles of triangles $AXY$ and $ABC$ intersect again at $Z \ne A$.
Prove that if $W \ne D$ and $OW = OD,$ then $DZ$ is tangent to the circle $AXY.$
\end{problem}
\begin{problem}[IMOSL 2022 G5]
Let $ABC$ be a triangle and $\ell_1,\ell_2$ be two parallel lines. Let $\ell_i$ intersects line $BC,CA,AB$ at $X_i,Y_i,Z_i$, respectively. Let $\Delta_i$ be the triangle formed by the line passed through $X_i$ and perpendicular to $BC$, the line passed through $Y_i$ and perpendicular to $CA$, and the line passed through $Z_i$ and perpendicular to $AB$. Prove that the circumcircles of $\Delta_1$ and $\Delta_2$ are tangent.
\end{problem}
\begin{problem}[IMOSL 2022 A4]
Let $n \geqslant 3$ be an integer, and let $x_1,x_2,\ldots,x_n$ be real numbers in the interval $[0,1]$. Let $s=x_1+x_2+\ldots+x_n$, and assume that $s \geqslant 3$. Prove that there exist integers $i$ and $j$ with $1 \leqslant i<j \leqslant n$ such that
\[2^{j-i}x_ix_j>2^{s-3}.\]
\end{problem}
\begin{problem}[IMOSL 2022 G1]
Let $ABCDE$ be a convex pentagon such that $BC=DE$. Assume that there is a point $T$ inside $ABCDE$ with $TB=TD,TC=TE$ and $\angle ABT = \angle TEA$. Let line $AB$ intersect lines $CD$ and $CT$ at points $P$ and $Q$, respectively. Assume that the points $P,B,A,Q$ occur on their line in that order. Let line $AE$ intersect $CD$ and $DT$ at points $R$ and $S$, respectively. Assume that the points $R,E,A,S$ occur on their line in that order. Prove that the points $P,S,Q,R$ lie on a circle.
\end{problem}

\end{document}
