\documentclass[11pt]{scrartcl}

\usepackage[sexy]{evan}
\usepackage{pgfplots}
\pgfplotsset{compat=1.15}
\usepackage{mathrsfs}
\usetikzlibrary{arrows}
\usepackage{graphics}
\usepackage{listings}
\usepackage{tikz}
\usepackage{ amssymb }
\usepackage[dvipsnames]{xcolor}
\definecolor{red1}{RGB}{255, 153, 153}
\definecolor{green1}{RGB}{204, 255, 204}
\definecolor{blue1}{RGB}{204, 255, 255}
\definecolor{yellow1}{RGB}{255, 247, 160}

\definecolor{red2}{RGB}{255, 102, 102}
\definecolor{green2}{RGB}{108, 255, 108}
\definecolor{blue2}{RGB}{94, 204, 255}
\definecolor{yellow2}{RGB}{255, 250, 104}

\definecolor{red2.5}{RGB}{255,76,76}
\definecolor{green2.5}{RGB}{54, 247, 54}
\definecolor{blue2.5}{RGB}{51, 189, 255}
\definecolor{yellow2.5}{RGB}{255, 242, 52}


\definecolor{red3}{RGB}{255, 51, 51}
\definecolor{green3}{RGB}{0, 240, 0}
\definecolor{blue3}{RGB}{9, 175, 255}
\definecolor{yellow3}{RGB}{255, 234, 0}

\definecolor{red3.5}{RGB}{229, 25, 25}
\definecolor{green3.5}{RGB}{0, 194, 0}
\definecolor{blue3.5}{RGB}{4, 143, 209}
\definecolor{yellow3.5}{RGB}{255,220,0}

\definecolor{red4}{RGB}{204, 0, 0}
\definecolor{green4}{RGB}{0, 149, 0}
\definecolor{blue4}{RGB}{0, 111, 164}
\definecolor{yellow4}{RGB}{255, 206, 0}

\newcommand{\camod}[1]{\frac{\ZZ}{#1 \ZZ}}
\newcommand{\modm}[1]{\text{ mod } #1}
\newcommand{\campm}[1]{\frac{\ZZ}{m\ZZ}}
\newcommand{\paren}[1]{\left(  #1 \right)}
\newcommand{\proc}[1]{Esto va a aparecer si no procrastino :S (Procrastinando desde #1)}


\title{matate}
\subtitle{PONTE A ENTRENAR}
\author{Emmanuel Buenrostro}


\begin{document}

\maketitle

\section{Problemas}
\begin{problem}[571352513856417722]
	A cyclic quadrilateral $ABCD$ has circumcircle $\Gamma$, and $AB+BC=AD+DC$. Let $E$ be the midpoint of arc $BCD$, and $F (\neq C)$ be the antipode of $A$ wrt $\Gamma$. Let $I,J,K$ be the incenter of $\triangle ABC$, the $A$-excenter of $\triangle ABC$, the incenter of $\triangle BCD$, respectively.
Suppose that a point $P$ satisfies $\triangle BIC \stackrel{+}{\sim} \triangle KPJ$. Prove that $EK$ and $PF$ intersect on $\Gamma.$
\end{problem}
\begin{problem}[7500559455615129254]
For every positive integer $N$, determine the smallest real number $b_{N}$ such that, for all real $x$,
\[
\sqrt[N]{\frac{x^{2 N}+1}{2}} \leqslant b_{N}(x-1)^{2}+x .
\]
\end{problem}
\begin{problem}[579228243242060]
Let $ABCD$ be a parallelogram. A line through $C$ crosses the side $AB$ at an interior point $X$,
and the line $AD$ at $Y$. The tangents of the circle $AXY$ at $X$ and $Y$, respectively, cross at $T$.
Prove that the circumcircles of triangles $ABD$ and $TXY$ intersect at two points, one lying on the line $AT$ and the other one lying on the line $CT$.
\end{problem}
\begin{problem}[1293772592063302344]
In non-isosceles acute ${}{\triangle ABC}$, $AP$, $BQ$, $CR$ is the height of the triangle. $A_1$ is the midpoint of $BC$, $AA_1$ intersects $QR$ at $K$, $QR$ intersects a straight line that crosses ${A}$ and is parallel to $BC$ at point ${D}$, the line connecting the midpoint of $AH$ and ${K}$ intersects $DA_1$ at $A_2$. Similarly define $B_2$, $C_2$. ${}\triangle A_2B_2C_2$ is known to be non-degenerate, and its circumscribed circle is $\omega$. Prove that: there are circles $\odot A'$, $\odot B'$, $\odot C'$ tangent to and INSIDE $\omega$ satisfying:
(1) $\odot A'$ is tangent to $AB$ and $AC$, $\odot B'$ is tangent to $BC$ and $BA$, and $\odot C'$ is tangent to $CA$ and $CB$.
(2) $A'$, $B'$, $C$' are different and collinear.
\end{problem}
\begin{problem}[3245291910836201005]
	Let $P$ be a point inside triangle $ABC$. Let $AP$ meet $BC$ at $A_1$, let $BP$ meet $CA$ at $B_1$, and let $CP$ meet $AB$ at $C_1$. Let $A_2$ be the point such that $A_1$ is the midpoint of $PA_2$, let $B_2$ be the point such that $B_1$ is the midpoint of $PB_2$, and let $C_2$ be the point such that $C_1$ is the midpoint of $PC_2$. Prove that points $A_2, B_2$, and $C_2$ cannot all lie strictly inside the circumcircle of triangle $ABC$.
\end{problem}
\begin{problem}[6612845742708555351]
	Cyclic quadrilateral $ABCD$ has circumcircle $(O)$. Points $M$ and $N$ are the midpoints of $BC$ and $CD$, and $E$ and $F$ lie on $AB$ and $AD$ respectively such that $EF$ passes through $O$ and $EO=OF$. Let $EN$ meet $FM$ at $P$. Denote $S$ as the circumcenter of $\triangle PEF$. Line $PO$ intersects $AD$ and $BA$ at $Q$ and $R$ respectively. Suppose $OSPC$ is a parallelogram. Prove that $AQ=AR$.
\end{problem}
\begin{problem}[227919487650283]
Let $ABC$ be an acute triangle with orthocenter $H$ and circumcircle $\Omega$. Let $M$ be the midpoint of side $BC$. Point $D$ is chosen from the minor arc $BC$ on $\Gamma$ such that $\angle BAD = \angle MAC$. Let $E$ be a point on $\Gamma$ such that $DE$ is perpendicular to $AM$, and $F$ be a point on line $BC$ such that $DF$ is perpendicular to $BC$. Lines $HF$ and $AM$ intersect at point $N$, and point $R$ is the reflection point of $H$ with respect to $N$.

Prove that $\angle AER + \angle DFR = 180^\circ$.
\end{problem}
\begin{problem}[165465510156789]
	Let $\Omega$ be the circumcircle of an isosceles trapezoid $ABCD$, in which $AD$ is parallel to $BC$. Let $X$ be the reflection point of $D$ with respect to $BC$. Point $Q$ is on the arc $BC$ of $\Omega$ that does not contain $A$. Let $P$ be the intersection of $DQ$ and $BC$. A point $E$ satisfies that $EQ$ is parallel to $PX$, and $EQ$ bisects $\angle BEC$. Prove that $EQ$ also bisects $\angle AEP$.
\end{problem}
\begin{problem}[132497611943266]
Suppose that $a,b,c,d$ are positive real numbers satisfying $(a+c)(b+d)=ac+bd$. Find the smallest possible value of
$$\frac{a}{b}+\frac{b}{c}+\frac{c}{d}+\frac{d}{a}.$$
\end{problem}
\begin{problem}[3866807698726339637]
Let $n$ and $k$ be two integers with $n>k\geqslant 1$. There are $2n+1$ students standing in a circle. Each student $S$ has $2k$ neighbors - namely, the $k$ students closest to $S$ on the left, and the $k$ students closest to $S$ on the right.

Suppose that $n+1$ of the students are girls, and the other $n$ are boys. Prove that there is a girl with at least $k$ girls among her neighbors.
\end{problem}
\begin{problem}[7550072974614174968]
Let $n \geqslant 3$ be an integer, and let $x_1,x_2,\ldots,x_n$ be real numbers in the interval $[0,1]$. Let $s=x_1+x_2+\ldots+x_n$, and assume that $s \geqslant 3$. Prove that there exist integers $i$ and $j$ with $1 \leqslant i<j \leqslant n$ such that
\[2^{j-i}x_ix_j>2^{s-3}.\]
\end{problem}
\begin{problem}[7220404010846068686]
	Let $ABC$ be a acute, non-isosceles triangle. $D,\ E,\ F$ are the midpoints of sides $AB,\ BC,\ AC$, resp. Denote by $(O),\ (O')$ the circumcircle and Euler circle of $ABC$. An arbitrary point $P$ lies inside triangle $DEF$ and $DP,\ EP,\ FP$ intersect $(O')$ at $D',\ E',\ F'$, resp. Point $A'$ is the point such that $D'$ is the midpoint of $AA'$. Points $B',\ C'$ are defined similarly.
a. Prove that if $PO=PO'$ then $O\in(A'B'C')$;
b. Point $A'$ is mirrored by $OD$, its image is $X$. $Y,\ Z$ are created in the same manner. $H$ is the orthocenter of $ABC$ and $XH,\ YH,\ ZH$ intersect $BC, AC, AB$ at $M,\ N,\ L$ resp. Prove that $M,\ N,\ L$ are collinear.
\end{problem}
\begin{problem}[2918584823978789760]
A point $T$ is chosen inside a triangle $ABC$. Let $A_1$, $B_1$, and $C_1$ be the reflections of $T$ in $BC$, $CA$, and $AB$, respectively. Let $\Omega$ be the circumcircle of the triangle $A_1B_1C_1$. The lines $A_1T$, $B_1T$, and $C_1T$ meet $\Omega$ again at $A_2$, $B_2$, and $C_2$, respectively. Prove that the lines $AA_2$, $BB_2$, and $CC_2$ are concurrent on $\Omega$.
\end{problem}
\begin{problem}[660403976209529]
A number is called Norwegian if it has three distinct positive divisors whose sum is equal to $2022$. Determine the smallest Norwegian number.
(Note: The total number of positive divisors of a Norwegian number is allowed to be larger than $3$.)
\end{problem}
\begin{problem}[952584318797289]
Show that the inequality\[\sum_{i=1}^n \sum_{j=1}^n \sqrt{|x_i-x_j|}\leqslant \sum_{i=1}^n \sum_{j=1}^n \sqrt{|x_i+x_j|}\]holds for all real numbers $x_1,\ldots x_n.$
\end{problem}

\end{document}
