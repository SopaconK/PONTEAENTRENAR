\documentclass[11pt]{scrartcl}

\usepackage[sexy]{evan}
\usepackage{pgfplots}
\pgfplotsset{compat=1.15}
\usepackage{mathrsfs}
\usetikzlibrary{arrows}
\usepackage{graphics}
\usepackage{listings}
\usepackage{tikz}
\usepackage{ amssymb }
\usepackage[dvipsnames]{xcolor}
\definecolor{red1}{RGB}{255, 153, 153}
\definecolor{green1}{RGB}{204, 255, 204}
\definecolor{blue1}{RGB}{204, 255, 255}
\definecolor{yellow1}{RGB}{255, 247, 160}

\definecolor{red2}{RGB}{255, 102, 102}
\definecolor{green2}{RGB}{108, 255, 108}
\definecolor{blue2}{RGB}{94, 204, 255}
\definecolor{yellow2}{RGB}{255, 250, 104}

\definecolor{red2.5}{RGB}{255,76,76}
\definecolor{green2.5}{RGB}{54, 247, 54}
\definecolor{blue2.5}{RGB}{51, 189, 255}
\definecolor{yellow2.5}{RGB}{255, 242, 52}


\definecolor{red3}{RGB}{255, 51, 51}
\definecolor{green3}{RGB}{0, 240, 0}
\definecolor{blue3}{RGB}{9, 175, 255}
\definecolor{yellow3}{RGB}{255, 234, 0}

\definecolor{red3.5}{RGB}{229, 25, 25}
\definecolor{green3.5}{RGB}{0, 194, 0}
\definecolor{blue3.5}{RGB}{4, 143, 209}
\definecolor{yellow3.5}{RGB}{255,220,0}

\definecolor{red4}{RGB}{204, 0, 0}
\definecolor{green4}{RGB}{0, 149, 0}
\definecolor{blue4}{RGB}{0, 111, 164}
\definecolor{yellow4}{RGB}{255, 206, 0}

\newcommand{\camod}[1]{\frac{\ZZ}{#1 \ZZ}}
\newcommand{\modm}[1]{\text{ mod } #1}
\newcommand{\campm}[1]{\frac{\ZZ}{m\ZZ}}
\newcommand{\paren}[1]{\left(  #1 \right)}
\newcommand{\proc}[1]{Esto va a aparecer si no procrastino :S (Procrastinando desde #1)}


\title{spamgeoysl}
\subtitle{PONTE A ENTRENAR}
\author{Emmanuel Buenrostro}


\begin{document}

\maketitle

\section{Problemas}
\begin{problem}[1810915585111530473]
	Given a scalene triangle $ \triangle ABC $. $ B', C' $ are points lie on the rays $ \overrightarrow{AB}, \overrightarrow{AC}  $ such that $ \overline{AB'} = \overline{AC}, \overline{AC'} = \overline{AB} $. Now, for an arbitrary point $ P $ in the plane. Let $ Q $ be the reflection point of $ P $ w.r.t $ \overline{BC} $. The intersections of $ \odot{\left(BB'P\right)} $ and $ \odot{\left(CC'P\right)} $ is $ P' $ and the intersections of $ \odot{\left(BB'Q\right)} $ and $ \odot{\left(CC'Q\right)} $ is $ Q' $. Suppose that $ O, O' $ are circumcenters of $ \triangle{ABC}, \triangle{AB'C'} $ Show that

1. $ O', P', Q' $ are colinear

2. $  \overline{O'P'} \cdot  \overline{O'Q'} = \overline{OA}^{2} $
\end{problem}
\begin{problem}[781756252908608]
Let $n\geqslant 2$ be a positive integer and $a_1,a_2, \ldots ,a_n$ be real numbers such that\[a_1+a_2+\dots+a_n=0.\]Define the set $A$ by
\[A=\left\{(i, j)\,|\,1 \leqslant i<j \leqslant n,\left|a_{i}-a_{j}\right| \geqslant 1\right\}\]Prove that, if $A$ is not empty, then
\[\sum_{(i, j) \in A} a_{i} a_{j}<0.\]
\end{problem}
\begin{problem}[7550072974614174968]
Let $n \geqslant 3$ be an integer, and let $x_1,x_2,\ldots,x_n$ be real numbers in the interval $[0,1]$. Let $s=x_1+x_2+\ldots+x_n$, and assume that $s \geqslant 3$. Prove that there exist integers $i$ and $j$ with $1 \leqslant i<j \leqslant n$ such that
\[2^{j-i}x_ix_j>2^{s-3}.\]
\end{problem}
\begin{problem}[8963205841174892420]
	Let $ABCD$ be a convex quadrilateral with pairwise distinct side lengths such that $AC\perp BD$. Let $O_1,O_2$ be the circumcenters of $\Delta ABD, \Delta CBD$, respectively. Show that $AO_2, CO_1$, the Euler line of $\Delta ABC$ and the Euler line of $\Delta ADC$ are concurrent.

(Remark: The Euler line of a triangle is the line on which its circumcenter, centroid, and orthocenter lie.)
\end{problem}
\begin{problem}[1248852037865425410]
Let $n>1$ be a positive integer. Each cell of an $n\times n$ table contains an integer. Suppose that the following conditions are satisfied:
Each number in the table is congruent to $1$ modulo $n$.
The sum of numbers in any row, as well as the sum of numbers in any column, is congruent to $n$ modulo $n^2$.
Let $R_i$ be the product of the numbers in the $i^{\text{th}}$ row, and $C_j$ be the product of the number in the $j^{\text{th}}$ column. Prove that the sums $R_1+\hdots R_n$ and $C_1+\hdots C_n$ are congruent modulo $n^4$.
\end{problem}
\begin{problem}[4992489807901310938]
Let $ABC$ be a triangle and $\ell_1,\ell_2$ be two parallel lines. Let $\ell_i$ intersects line $BC,CA,AB$ at $X_i,Y_i,Z_i$, respectively. Let $\Delta_i$ be the triangle formed by the line passed through $X_i$ and perpendicular to $BC$, the line passed through $Y_i$ and perpendicular to $CA$, and the line passed through $Z_i$ and perpendicular to $AB$. Prove that the circumcircles of $\Delta_1$ and $\Delta_2$ are tangent.
\end{problem}
\begin{problem}[6246999615324043054]
	A site is any point $(x, y)$ in the plane such that $x$ and $y$ are both positive integers less than or equal to 20.

Initially, each of the 400 sites is unoccupied. Amy and Ben take turns placing stones with Amy going first. On her turn, Amy places a new red stone on an unoccupied site such that the distance between any two sites occupied by red stones is not equal to $\sqrt{5}$. On his turn, Ben places a new blue stone on any unoccupied site. (A site occupied by a blue stone is allowed to be at any distance from any other occupied site.) They stop as soon as a player cannot place a stone.

Find the greatest $K$ such that Amy can ensure that she places at least $K$ red stones, no matter how Ben places his blue stones.
\end{problem}
\begin{problem}[6116877365036470315]
	Determine all functions $f$ defined on the set of all positive integers and taking non-negative integer values, satisfying the three conditions:
$(i)$ $f(n) \neq 0$ for at least one $n$;
$(ii)$ $f(x y)=f(x)+f(y)$ for every positive integers $x$ and $y$;
$(iii)$ there are infinitely many positive integers $n$ such that $f(k)=f(n-k)$ for all $k<n$.
\end{problem}
\begin{problem}[3470579368412517052]
	A hunter and an invisible rabbit play a game on an infinite square grid. First the hunter fixes a colouring of the cells with finitely many colours. The rabbit then secretly chooses a cell to start in. Every minute, the rabbit reports the colour of its current cell to the hunter, and then secretly moves to an adjacent cell that it has not visited before (two cells are adjacent if they share an edge). The hunter wins if after some finite time either:
the rabbit cannot move; or
the hunter can determine the cell in which the rabbit started.
Decide whether there exists a winning strategy for the hunter.
\end{problem}
\begin{problem}[883811987981100]
Let $ABC$ be a triangle with $AB=AC$, and let $M$ be the midpoint of $BC$. Let $P$ be a point such that $PB<PC$ and $PA$ is parallel to $BC$. Let $X$ and $Y$ be points on the lines $PB$ and $PC$, respectively, so that $B$ lies on the segment $PX$, $C$ lies on the segment $PY$, and $\angle PXM=\angle PYM$. Prove that the quadrilateral $APXY$ is cyclic.
\end{problem}
\begin{problem}[651308339506337942]
Given a convex pentagon $ ABCDE. $ Let $ A_1 $ be the intersection of $ BD $ with $ CE $ and define $ B_1, C_1, D_1, E_1 $ similarly, $ A_2 $ be the second intersection of $ \odot (ABD_1),\odot (AEC_1) $ and define $ B_2, C_2, D_2, E_2 $ similarly. Prove that $ AA_2, BB_2, CC_2, DD_2, EE_2 $ are concurrent.
\end{problem}
\begin{problem}[8059760967121829853]
	Let $n\geqslant 3$ be an integer. Prove that there exists a set $S$ of $2n$ positive integers satisfying the following property: For every $m=2,3,...,n$ the set $S$ can be partitioned into two subsets with equal sums of elements, with one of subsets of cardinality $m$.
\end{problem}
\begin{problem}[456772085666528]
Let $\triangle ABC$ be an acute triangle with incenter $I$ and circumcenter $O$. The incircle touches sides $BC,CA,$ and $AB$ at $D,E,$ and $F$ respectively, and $A'$ is the reflection of $A$ over $O$. The circumcircles of $ABC$ and $A'EF$ meet at $G$, and the circumcircles of $AMG$ and $A'EF$ meet at a point $H\neq G$, where $M$ is the midpoint of $EF$. Prove that if $GH$ and $EF$ meet at $T$, then $DT\perp EF$.
\end{problem}
\begin{problem}[587866144613888]
The Bank of Oslo issues two types of coin: aluminum (denoted A) and bronze (denoted B). Marianne has $n$ aluminum coins and $n$ bronze coins arranged in a row in some arbitrary initial order. A chain is any subsequence of consecutive coins of the same type. Given a fixed positive integer $k \leq 2n$, Gilberty repeatedly performs the following operation: he identifies the longest chain containing the $k^{th}$ coin from the left and moves all coins in that chain to the left end of the row. For example, if $n=4$ and $k=4$, the process starting from the ordering $AABBBABA$ would be $AABBBABA \to BBBAAABA \to AAABBBBA \to BBBBAAAA \to ...$

Find all pairs $(n,k)$ with $1 \leq k \leq 2n$ such that for every initial ordering, at some moment during the process, the leftmost $n$ coins will all be of the same type.
\end{problem}
\begin{problem}[528087142744727]
	Let $ABC$ be a scalene triangle with orthocenter $H$ and circumcenter $O$. Let $P$ be the midpoint of $\overline{AH}$ and let $T$ be on line $BC$ with $\angle TAO=90^{\circ}$. Let $X$ be the foot of the altitude from $O$ onto line $PT$. Prove that the midpoint of $\overline{PX}$ lies on the nine-point circle* of $\triangle ABC$.

*The nine-point circle of $\triangle ABC$ is the unique circle passing through the following nine points: the midpoint of the sides, the feet of the altitudes, and the midpoints of $\overline{AH}$, $\overline{BH}$, and $\overline{CH}$.
\end{problem}
\begin{problem}[1620616963605432410]
Given an isosceles triangle $\triangle ABC$, $AB=AC$. A line passes through $M$, the midpoint of $BC$, and intersects segment $AB$ and ray $CA$ at $D$ and $E$, respectively. Let $F$ be a point of $ME$ such that $EF=DM$, and $K$ be a point on $MD$. Let $\Gamma_1$ be the circle passes through $B,D,K$ and $\Gamma_2$ be the circle passes through $C,E,K$. $\Gamma_1$ and $\Gamma_2$ intersect again at $L \neq K$. Let $\omega_1$ and $\omega_2$ be the circumcircle of $\triangle LDE$ and $\triangle LKM$. Prove that, if $\omega_1$ and $\omega_2$ are symmetric wrt $L$, then $BF$ is perpendicular to $BC$.
\end{problem}
\begin{problem}[7088779505939683183]
Find all triples $(a, b, c)$ of positive integers such that $a^3 + b^3 + c^3 = (abc)^2$.
\end{problem}
\begin{problem}[689874125173032]
Let $\omega_1,\omega_2$ be two non-intersecting circles, with circumcenters $O_1,O_2$ respectively, and radii $r_1,r_2$ respectively where $r_1 < r_2$. Let $AB,XY$ be the two internal common tangents of $\omega_1,\omega_2$, where $A,X$ lie on $\omega_1$, $B,Y$ lie on $\omega_2$. The circle with diameter $AB$ meets $\omega_1,\omega_2$ at $P$ and $Q$ respectively. If$$\angle AO_1P+\angle BO_2Q=180^{\circ},$$find the value of $\frac{PX}{QY}$ (in terms of $r_1,r_2$).
\end{problem}
\begin{problem}[526922799283626]
	For each $1\leq i\leq 9$ and $T\in\mathbb N$, define $d_i(T)$ to be the total number of times the digit $i$ appears when all the multiples of $1829$ between $1$ and $T$ inclusive are written out in base $10$.

Show that there are infinitely many $T\in\mathbb N$ such that there are precisely two distinct values among $d_1(T)$, $d_2(T)$, $\dots$, $d_9(T)$
\end{problem}
\begin{problem}[86986480818494]
Given a scalene triangle $ABC$ inscribed in the circle $(O)$. Let $(I)$ be its incircle and $BI,CI$ cut $AC,AB$ at $E,F$ respectively. A circle passes through $E$ and touches $OB$ at $B$ cuts $(O)$ again at $M$. Similarly, a circle passes through $F$ and touches $OC$ at $C$ cuts $(O)$ again at $N$. $ME,NF$ cut $(O)$ again at $P,Q$. Let $K$ be the intersection of $EF$ and $BC$ and let $PQ$ cuts $BC$ and $EF$ at $G,H$, respectively. Show that the median correspond to $G$ of the triangle $GHK$ is perpendicular to $IO$.
\end{problem}
\begin{problem}[7243491713649826569]
In the triangle $ABC$ let $B'$ and $C'$ be the midpoints of the sides $AC$ and $AB$ respectively and $H$ the foot of the altitude passing through the vertex $A$. Prove that the circumcircles of the triangles $AB'C'$,$BC'H$, and $B'CH$ have a common point $I$ and that the line $HI$ passes through the midpoint of the segment $B'C'.$
\end{problem}
\begin{problem}[4678973565823282552]
Four positive integers $x,y,z$ and $t$ satisfy the relations
\[ xy - zt = x + y = z + t. \]Is it possible that both $xy$ and $zt$ are perfect squares?
\end{problem}
\begin{problem}[902621191535073]
Given six points $ A, B, C, D, E, F $ such that $ \triangle BCD \stackrel{+}{\sim} \triangle ECA \stackrel{+}{\sim} \triangle BFA $ and let $ I $ be the incenter of $ \triangle ABC. $ Prove that the circumcenter of $ \triangle AID, \triangle BIE, \triangle CIF $ are collinear.
\end{problem}
\begin{problem}[748616641641895]
Let $ABC$ be a triangle. Let $ABC_1, BCA_1, CAB_1$ be three equilateral triangles that do not overlap with $ABC$.
Let $P$ be the intersection of the circumcircles of triangle $ABC_1$ and $CAB_1$.
Let $Q$ be the point on the circumcircle of triangle $CAB_1$ so that $PQ$ is parallel to $BA_1$. Let $R$ be the point on the circumcircle of triangle $ABC_1$ so that $PR$ is parallel to $CA_1$.

Show that the line connecting the centroid of triangle $ABC$ and the centroid of triangle $PQR$ is parallel to $BC$.
\end{problem}
\begin{problem}[702587891849077]
	Given an integer $n \geqslant 2$. Suppose there is a point $P$ inside a convex cyclic $2n$-gon $A_1 \ldots A_{2n}$ satisfying$$\angle PA_1A_2 = \angle PA_2A_3 = \ldots = \angle PA_{2n}A_1,$$prove that$$ \prod_{i=1}^{n} \left|A_{2i - 1}A_{2i} \right| = \prod_{i=1}^{n} \left|A_{2i}A_{2i+1} \right|,$$where $A_{2n + 1} = A_1$.
\end{problem}
\begin{problem}[695330092247108707]
There is an integer $n > 1$. There are $n^2$ stations on a slope of a mountain, all at different altitudes. Each of two cable car companies, $A$ and $B$, operates $k$ cable cars; each cable car provides a transfer from one of the stations to a higher one (with no intermediate stops). The $k$ cable cars of $A$ have $k$ different starting points and $k$ different finishing points, and a cable car which starts higher also finishes higher. The same conditions hold for $B$. We say that two stations are linked by a company if one can start from the lower station and reach the higher one by using one or more cars of that company (no other movements between stations are allowed). Determine the smallest positive integer $k$ for which one can guarantee that there are two stations that are linked by both companies.
\end{problem}
\begin{problem}[264456837378391]
Let $ABC$ be a triangle such that the angular bisector of $\angle BAC$, the $B$-median and the perpendicular bisector of $AB$ intersect at a single point $X$. Let $H$ be the orthocenter of $ABC$. Show that $\angle BXH = 90^{\circ}$.
\end{problem}
\begin{problem}[915997916422887]
	Let $ABC$ and $A'B'C'$ be two triangles so that the midpoints of $\overline{AA'}, \overline{BB'}, \overline{CC'}$ form a triangle as well. Suppose that for any point $X$ on the circumcircle of $ABC$, there exists exactly one point $X'$ on the circumcircle of $A'B'C'$ so that the midpoints of $\overline{AA'}, \overline{BB'}, \overline{CC'}$ and $\overline{XX'}$ are concyclic. Show that $ABC$ is similar to $A'B'C'$.
\end{problem}
\begin{problem}[2672133756769464425]
Is there a scalene triangle $ABC$ similar to triangle $IHO$, where $I$, $H$, and $O$ are the incenter, orthocenter, and circumcenter, respectively, of triangle $ABC$?
\end{problem}
\begin{problem}[836909183133087]
Given a triangle $ \triangle{ABC} $ with circumcircle $ \Omega $. Denote its incenter and $ A $-excenter by $ I, J $, respectively. Let $ T $ be the reflection of $ J $ w.r.t $ BC $ and $ P $ is the intersection of $ BC $ and $ AT $. If the circumcircle of $ \triangle{AIP} $ intersects $ BC $ at $ X \neq P $ and there is a point $ Y \neq A $ on $ \Omega $ such that $ IA = IY $. Show that $ \odot\left(IXY\right) $ tangents to the line $ AI $.
\end{problem}
\begin{problem}[156060759856343521]
Let $ABC$ be an acute triangle with $\angle ACB>2 \angle ABC$. Let $I$ be the incenter of $ABC$, $K$ is the reflection of $I$ in line $BC$. Let line $BA$ and $KC$ intersect at $D$. The line through $B$ parallel to $CI$ intersects the minor arc $BC$ on the circumcircle of $ABC$ at $E(E \neq B)$. The line through $A$ parallel to $BC$ intersects the line $BE$ at $F$.
Prove that if $BF=CE$, then $FK=AD$.
\end{problem}
\begin{problem}[4375421764909014892]
	Find all positive integers $n\geq1$ such that there exists a pair $(a,b)$ of positive integers, such that $a^2+b+3$ is not divisible by the cube of any prime, and$$n=\frac{ab+3b+8}{a^2+b+3}.$$
\end{problem}
\begin{problem}[297274918587198]
	Find all positive integers $n$ with the following property: the $k$ positive divisors of $n$ have a permutation $(d_1,d_2,\ldots,d_k)$ such that for $i=1,2,\ldots,k$, the number $d_1+d_2+\cdots+d_i$ is a perfect square.
\end{problem}
\begin{problem}[8916142707013964275]
	Let $k$ be a positive integer. The organising commitee of a tennis tournament is to schedule the matches for $2k$ players so that every two players play once, each day exactly one match is played, and each player arrives to the tournament site the day of his first match, and departs the day of his last match. For every day a player is present on the tournament, the committee has to pay $1$ coin to the hotel. The organisers want to design the schedule so as to minimise the total cost of all players' stays. Determine this minimum cost.
\end{problem}
\begin{problem}[6612845742708555351]
	Cyclic quadrilateral $ABCD$ has circumcircle $(O)$. Points $M$ and $N$ are the midpoints of $BC$ and $CD$, and $E$ and $F$ lie on $AB$ and $AD$ respectively such that $EF$ passes through $O$ and $EO=OF$. Let $EN$ meet $FM$ at $P$. Denote $S$ as the circumcenter of $\triangle PEF$. Line $PO$ intersects $AD$ and $BA$ at $Q$ and $R$ respectively. Suppose $OSPC$ is a parallelogram. Prove that $AQ=AR$.
\end{problem}
\begin{problem}[6734490609685717062]
	Let $I,G,O$ be the incenter, centroid and the circumcenter of triangle $ABC$, respectively. Let $X,Y,Z$ be on the rays $BC, CA, AB$ respectively so that $BX=CY=AZ$. Let $F$ be the centroid of $XYZ$.

Show that $FG$ is perpendicular to $IO$.
\end{problem}
\begin{problem}[165465510156789]
	Let $\Omega$ be the circumcircle of an isosceles trapezoid $ABCD$, in which $AD$ is parallel to $BC$. Let $X$ be the reflection point of $D$ with respect to $BC$. Point $Q$ is on the arc $BC$ of $\Omega$ that does not contain $A$. Let $P$ be the intersection of $DQ$ and $BC$. A point $E$ satisfies that $EQ$ is parallel to $PX$, and $EQ$ bisects $\angle BEC$. Prove that $EQ$ also bisects $\angle AEP$.
\end{problem}
\begin{problem}[952584318797289]
Show that the inequality\[\sum_{i=1}^n \sum_{j=1}^n \sqrt{|x_i-x_j|}\leqslant \sum_{i=1}^n \sum_{j=1}^n \sqrt{|x_i+x_j|}\]holds for all real numbers $x_1,\ldots x_n.$
\end{problem}
\begin{problem}[574223786384294]
Find all integers $n \geq 3$ for which there exist real numbers $a_1, a_2, \dots a_{n + 2}$ satisfying $a_{n + 1} = a_1$, $a_{n + 2} = a_2$ and
$$a_ia_{i + 1} + 1 = a_{i + 2},$$for $i = 1, 2, \dots, n$.
\end{problem}
\begin{problem}[579228243242060]
Let $ABCD$ be a parallelogram. A line through $C$ crosses the side $AB$ at an interior point $X$,
and the line $AD$ at $Y$. The tangents of the circle $AXY$ at $X$ and $Y$, respectively, cross at $T$.
Prove that the circumcircles of triangles $ABD$ and $TXY$ intersect at two points, one lying on the line $AT$ and the other one lying on the line $CT$.
\end{problem}
\begin{problem}[162618813015033]
In $\triangle {ABC}$, tangents of the circumcircle $\odot {O}$ at $B, C$ and at $A, B$ intersects at $X, Y$ respectively. $AX$ cuts $BC$ at ${D}$ and $CY$ cuts $AB$ at ${F}$. Ray $DF$ cuts arc $AB$ of the circumcircle at ${P}$. $Q, R$ are on segments $AB, AC$ such that $P, Q, R$ are collinear and $QR \parallel BO$. If $PQ^2=PR \cdot QR$, find $\angle ACB$.
\end{problem}
\begin{problem}[57940096937913]
Let $ABC$ be an acute-angled triangle and let $D, E$, and $F$ be the feet of altitudes from $A, B$, and $C$ to sides $BC, CA$, and $AB$, respectively. Denote by $\omega_B$ and $\omega_C$ the incircles of triangles $BDF$ and $CDE$, and let these circles be tangent to segments $DF$ and $DE$ at $M$ and $N$, respectively. Let line $MN$ meet circles $\omega_B$ and $\omega_C$ again at $P \ne M$ and $Q \ne N$, respectively. Prove that $MP = NQ$.
\end{problem}
\begin{problem}[4451072691230235426]
A convex quadrilateral $ABCD$ has an inscribed circle with center $I$. Let $I_a, I_b, I_c$ and $I_d$ be the incenters of the triangles $DAB, ABC, BCD$ and $CDA$, respectively. Suppose that the common external tangents of the circles $AI_bI_d$ and $CI_bI_d$ meet at $X$, and the common external tangents of the circles $BI_aI_c$ and $DI_aI_c$ meet at $Y$. Prove that $\angle{XIY}=90^{\circ}$.
\end{problem}
\begin{problem}[961350373727093]
Given a positive integer $k$ show that there exists a prime $p$ such that one can choose distinct integers $a_1,a_2\cdots, a_{k+3} \in \{1, 2, \cdots ,p-1\}$ such that p divides $a_ia_{i+1}a_{i+2}a_{i+3}-i$ for all $i= 1, 2, \cdots, k$.
\end{problem}
\begin{problem}[5441518070935718077]
	Let $ABC$ be an acute-angled triangle. The line through $C$ perpendicular to $AC$ meets the external angle bisector of $\angle ABC$ at $D$. Let $H$ be the foot of the perpendicular from $D$ onto $BC$. The point $K$ is chosen on $AB$ so that $KH \parallel AC$. Let $M$ be the midpoint of $AK$. Prove that $MC = MB + BH$.
\end{problem}
\begin{problem}[571352513856417722]
	A cyclic quadrilateral $ABCD$ has circumcircle $\Gamma$, and $AB+BC=AD+DC$. Let $E$ be the midpoint of arc $BCD$, and $F (\neq C)$ be the antipode of $A$ wrt $\Gamma$. Let $I,J,K$ be the incenter of $\triangle ABC$, the $A$-excenter of $\triangle ABC$, the incenter of $\triangle BCD$, respectively.
Suppose that a point $P$ satisfies $\triangle BIC \stackrel{+}{\sim} \triangle KPJ$. Prove that $EK$ and $PF$ intersect on $\Gamma.$
\end{problem}
\begin{problem}[522990139281725]
	For any odd prime $p$ and any integer $n,$ let $d_p (n) \in \{ 0,1, \dots, p-1 \}$ denote the remainder when $n$ is divided by $p.$ We say that $(a_0, a_1, a_2, \dots)$ is a p-sequence, if $a_0$ is a positive integer coprime to $p,$ and $a_{n+1} =a_n + d_p (a_n)$ for $n \geqslant 0.$
(a) Do there exist infinitely many primes $p$ for which there exist $p$-sequences $(a_0, a_1, a_2, \dots)$ and $(b_0, b_1, b_2, \dots)$ such that $a_n >b_n$ for infinitely many $n,$ and $b_n > a_n$ for infinitely many $n?$
(b) Do there exist infinitely many primes $p$ for which there exist $p$-sequences $(a_0, a_1, a_2, \dots)$ and $(b_0, b_1, b_2, \dots)$ such that $a_0 <b_0,$ but $a_n >b_n$ for all $n \geqslant 1?$
\end{problem}
\begin{problem}[457324036151847]
Let $O$ and $H$ be the circumcenter and the orthocenter, respectively, of an acute triangle $ABC$. Points $D$ and $E$ are chosen from sides $AB$ and $AC$, respectively, such that $A$, $D$, $O$, $E$ are concyclic. Let $P$ be a point on the circumcircle of triangle $ABC$. The line passing $P$ and parallel to $OD$ intersects $AB$ at point $X$, while the line passing $P$ and parallel to $OE$ intersects $AC$ at $Y$. Suppose that the perpendicular bisector of $\overline{HP}$ does not coincide with $XY$, but intersect $XY$ at $Q$, and that points $A$, $Q$ lies on the different sides of $DE$. Prove that $\angle EQD = \angle BAC$.
\end{problem}
\begin{problem}[3435532350205377704]
Find all functions $f:\mathbb Z_{>0}\to \mathbb Z_{>0}$ such that $a+f(b)$ divides $a^2+bf(a)$ for all positive integers $a$ and $b$ with $a+b>2019$.
\end{problem}
\begin{problem}[6978535805224432571]
	The Fibonacci numbers $F_0, F_1, F_2, . . .$ are defined inductively by $F_0=0, F_1=1$, and $F_{n+1}=F_n+F_{n-1}$ for $n \ge 1$. Given an integer $n \ge 2$, determine the smallest size of a set $S$ of integers such that for every $k=2, 3, . . . , n$ there exist some $x, y \in S$ such that $x-y=F_k$.
\end{problem}
\begin{problem}[623590906176957]
	The Bank of Bath issues coins with an $H$ on one side and a $T$ on the other. Harry has $n$ of these coins arranged in a line from left to right. He repeatedly performs the following operation: if there are exactly $k>0$ coins showing $H$, then he turns over the $k$th coin from the left; otherwise, all coins show $T$ and he stops. For example, if $n=3$ the process starting with the configuration $THT$ would be $THT \to HHT  \to HTT \to TTT$, which stops after three operations.

(a) Show that, for each initial configuration, Harry stops after a finite number of operations.

(b) For each initial configuration $C$, let $L(C)$ be the number of operations before Harry stops. For example, $L(THT) = 3$ and $L(TTT) = 0$. Determine the average value of $L(C)$ over all $2^n$ possible initial configurations $C$.
\end{problem}
\begin{problem}[7997372712267182584]
Let $ABCDE$ be a convex pentagon such that $AB=BC=CD$, $\angle{EAB}=\angle{BCD}$, and $\angle{EDC}=\angle{CBA}$. Prove that the perpendicular line from $E$ to $BC$ and the line segments $AC$ and $BD$ are concurrent.
\end{problem}
\begin{problem}[625002281186392279]
	Let $\Gamma$ be the circumcircle of acute triangle $ABC$. Points $D$ and $E$ are on segments $AB$ and $AC$ respectively such that $AD = AE$. The perpendicular bisectors of $BD$ and $CE$ intersect minor arcs $AB$ and $AC$ of $\Gamma$ at points $F$ and $G$ respectively. Prove that lines $DE$ and $FG$ are either parallel or they are the same line.
\end{problem}
\begin{problem}[8972547734710795566]
Let incircle $(I)$ of triangle $ABC$ touch the sides $BC,CA,AB$ at $D,E,F$ respectively. Let $(O)$ be the circumcircle of $ABC$. Ray $EF$ meets $(O)$ at $M$. Tangents at $M$ and $A$ of $(O)$ meet at $S$. Tangents at $B$ and $C$ of $(O)$ meet at $T$. Line $TI$ meets $OA$ at $J$. Prove that $\angle ASJ=\angle IST$.
\end{problem}
\begin{problem}[8569243655022492300]
	Given a $ \triangle ABC $ and a point $ P. $ Let $ O, D, E, F $ be the circumcenter of $ \triangle ABC, \triangle BPC, \triangle CPA, \triangle APB, $ respectively and let $ T $ be the intersection of $ BC $ with $ EF. $ Prove that the reflection of $ O $ in $ EF $ lies on the perpendicular from $ D $ to $ PT. $
\end{problem}
\begin{problem}[2918584823978789760]
A point $T$ is chosen inside a triangle $ABC$. Let $A_1$, $B_1$, and $C_1$ be the reflections of $T$ in $BC$, $CA$, and $AB$, respectively. Let $\Omega$ be the circumcircle of the triangle $A_1B_1C_1$. The lines $A_1T$, $B_1T$, and $C_1T$ meet $\Omega$ again at $A_2$, $B_2$, and $C_2$, respectively. Prove that the lines $AA_2$, $BB_2$, and $CC_2$ are concurrent on $\Omega$.
\end{problem}
\begin{problem}[7553717274310387624]
Let $ABC$ be a triangle with incentre $I$ and circumcircle $\omega$. The incircle of the triangle $ABC$
touches the sides $BC$, $CA$ and $AB$ at $D$, $E$ and $F$, respectively. The circumcircle of triangle $ADI$ crosses $\omega$ again at $P$, and the lines $PE$ and $PF$ cross $\omega$ again at $X$and $Y$, respectively. Prove that the lines $AI$, $BX$ and $CY$ are concurrent.
\end{problem}
\begin{problem}[3192129869376364982]
Let $u_1, u_2, \dots, u_{2019}$ be real numbers satisfying\[u_{1}+u_{2}+\cdots+u_{2019}=0 \quad \text { and } \quad u_{1}^{2}+u_{2}^{2}+\cdots+u_{2019}^{2}=1.\]Let $a=\min \left(u_{1}, u_{2}, \ldots, u_{2019}\right)$ and $b=\max \left(u_{1}, u_{2}, \ldots, u_{2019}\right)$. Prove that
\[
a b \leqslant-\frac{1}{2019}.
\]
\end{problem}
\begin{problem}[596902679696332]
Find all positive integers $n \geqslant 2$ for which there exist $n$ real numbers $a_1<\cdots< a_n$ and a real number $r>0$ such that the $\tfrac{1}{2}n(n-1)$ differences $a_j-a_i$ for $1 \leqslant i<j \leqslant n$ are equal, in some order, to the numbers $r^1,r^2,\ldots,r^{\frac{1}{2}n(n-1)}$.
\end{problem}
\begin{problem}[258585206260584]
Let $n \geqslant 100$ be an integer. Ivan writes the numbers $n, n+1, \ldots, 2 n$ each on different cards. He then shuffles these $n+1$ cards, and divides them into two piles. Prove that at least one of the piles contains two cards such that the sum of their numbers is a perfect square.
\end{problem}
\begin{problem}[4948608980214807448]
	Let $ABC$ be a scalene triangle with circumcenter $O$ and orthocenter $H$. Let $AYZ$ be another triangle sharing the vertex $A$ such that its circumcenter is $H$ and its orthocenter is $O$. Show that if $Z$ is on $BC$, then $A,H,O,Y$ are concyclic.
\end{problem}
\begin{problem}[684265043263216]
Let $\mathbb{Z}$ be the set of integers. Determine all functions $f: \mathbb{Z} \rightarrow \mathbb{Z}$ such that, for all integers $a$ and $b$,$$f(2a)+2f(b)=f(f(a+b)).$$
\end{problem}
\begin{problem}[2749225075653830789]
	In a regular 100-gon, 41 vertices are colored black and the remaining 59 vertices are colored white. Prove that there exist 24 convex quadrilaterals $Q_{1}, \ldots, Q_{24}$ whose corners are vertices of the 100-gon, so that
the quadrilaterals $Q_{1}, \ldots, Q_{24}$ are pairwise disjoint, and
every quadrilateral $Q_{i}$ has three corners of one color and one corner of the other color.
\end{problem}
\begin{problem}[6558910862034852540]
Let $\mathbb{R}^+$ denote the set of positive real numbers. Find all functions $f: \mathbb{R}^+ \to \mathbb{R}^+$ such that for each $x \in \mathbb{R}^+$, there is exactly one $y \in \mathbb{R}^+$ satisfying$$xf(y)+yf(x) \leq 2$$
\end{problem}
\begin{problem}[3866807698726339637]
Let $n$ and $k$ be two integers with $n>k\geqslant 1$. There are $2n+1$ students standing in a circle. Each student $S$ has $2k$ neighbors - namely, the $k$ students closest to $S$ on the left, and the $k$ students closest to $S$ on the right.

Suppose that $n+1$ of the students are girls, and the other $n$ are boys. Prove that there is a girl with at least $k$ girls among her neighbors.
\end{problem}
\begin{problem}[967014444176640]
Let $m,n \geqslant 2$ be integers, let $X$ be a set with $n$ elements, and let $X_1,X_2,\ldots,X_m$ be pairwise distinct non-empty, not necessary disjoint subset of $X$. A function $f \colon X \to \{1,2,\ldots,n+1\}$ is called nice if there exists an index $k$ such that\[\sum_{x \in X_k} f(x)>\sum_{x \in X_i} f(x) \quad \text{for all } i \ne k.\]Prove that the number of nice functions is at least $n^n$.
\end{problem}
\begin{problem}[8534263250311217423]
	In acute triangle $\triangle {ABC}$, $\angle 
A > \angle B > \angle C$. $\triangle {AC_1B}$ and $\triangle {CB_1A}$ are isosceles triangles such that $\triangle {AC_1B} \stackrel{+}{\sim}  \triangle {CB_1A}$. Let lines $BB_1, CC_1$ intersects at ${T}$. Prove that if all points mentioned above are distinct, $\angle ATC$ isn't a right angle.
\end{problem}
\begin{problem}[1293772592063302344]
In non-isosceles acute ${}{\triangle ABC}$, $AP$, $BQ$, $CR$ is the height of the triangle. $A_1$ is the midpoint of $BC$, $AA_1$ intersects $QR$ at $K$, $QR$ intersects a straight line that crosses ${A}$ and is parallel to $BC$ at point ${D}$, the line connecting the midpoint of $AH$ and ${K}$ intersects $DA_1$ at $A_2$. Similarly define $B_2$, $C_2$. ${}\triangle A_2B_2C_2$ is known to be non-degenerate, and its circumscribed circle is $\omega$. Prove that: there are circles $\odot A'$, $\odot B'$, $\odot C'$ tangent to and INSIDE $\omega$ satisfying:
(1) $\odot A'$ is tangent to $AB$ and $AC$, $\odot B'$ is tangent to $BC$ and $BA$, and $\odot C'$ is tangent to $CA$ and $CB$.
(2) $A'$, $B'$, $C$' are different and collinear.
\end{problem}
\begin{problem}[682786464566571]
Let $ABCD$ be a parallelogram with $AC=BC.$ A point $P$ is chosen on the extension of ray $AB$ past $B.$ The circumcircle of $ACD$ meets the segment $PD$ again at $Q.$ The circumcircle of triangle $APQ$ meets the segment $PC$ at $R.$ Prove that lines $CD,AQ,BR$ are concurrent.
\end{problem}
\begin{problem}[727078403801409]
Let $ABC$ be a triangle with incenter $I$ and circumcircle $\Omega$. A point $X$ on $\Omega$ which is different from $A$ satisfies $AI=XI$. The incircle touches $AC$ and $AB$ at $E, F$, respectively. Let $M_a, M_b, M_c$ be the midpoints of sides $BC, CA, AB$, respectively. Let $T$ be the intersection of the lines $M_bF$ and $M_cE$. Suppose that $AT$ intersects $\Omega$ again at a point $S$.

Prove that $X, M_a, S, T$ are concyclic.
\end{problem}
\begin{problem}[537574018594693]
Let $ABC$ be a triangle with $O$ as its circumcenter. A circle $\Gamma$ tangents $OB, OC$ at $B, C$, respectively. Let $D$ be a point on $\Gamma$ other than $B$ with $CB=CD$, $E$ be the second intersection of $DO$ and $\Gamma$, and $F$ be the second intersection of $EA$ and $\Gamma$. Let $X$ be a point on the line $AC$ so that $XB\perp BD$. Show that one half of $\angle ADF$ is equal to one of $\angle BDX$ and $\angle BXD$.
\end{problem}
\begin{problem}[5261846980754565299]
	Let $A,B,C$ be the midpoints of the three sides $B'C', C'A', A'B'$ of the triangle $A'B'C'$ respectively. Let $P$ be a point inside $\Delta ABC$, and $AP,BP,CP$ intersect with $BC, CA, AB$ at $P_a,P_b,P_c$, respectively. Lines $P_aP_b, P_aP_c$ intersect with $B'C'$ at $R_b, R_c$ respectively, lines $P_bP_c, P_bP_a$ intersect with $C'A'$ at $S_c, S_a$ respectively. and lines $P_cP_a, P_cP_b$ intersect with $A'B'$ at $T_a, T_b$, respectively. Given that $S_c,S_a, T_a, T_b$ are all on a circle centered at $O$.

Show that $OR_b=OR_c$.
\end{problem}
\begin{problem}[5066939379306191291]
Let $ABC$ be an acute triangle with circumcenter $O$ and circumcircle $\Omega$. Choose points $D, E$ from sides $AB, AC$, respectively, and let $\ell$ be the line passing through $A$ and perpendicular to $DE$. Let $\ell$ intersect the circumcircle of triangle $ADE$ and $\Omega$ again at points $P, Q$, respectively. Let $N$ be the intersection of $OQ$ and $BC$, $S$ be the intersection of $OP$ and $DE$, and $W$ be the orthocenter of triangle $SAO$.

Prove that the points $S$, $N$, $O$, $W$ are concyclic.
\end{problem}
\begin{problem}[402654566950359]
Let $a_1$, $a_2$, $\ldots$ be an infinite sequence of positive integers. Suppose that there is an integer $N > 1$ such that, for each $n \geq N$, the number
$$\frac{a_1}{a_2} + \frac{a_2}{a_3} + \cdots + \frac{a_{n-1}}{a_n} + \frac{a_n}{a_1}$$is an integer. Prove that there is a positive integer $M$ such that $a_m = a_{m+1}$ for all $m \geq M$.
\end{problem}
\begin{problem}[132497611943266]
Suppose that $a,b,c,d$ are positive real numbers satisfying $(a+c)(b+d)=ac+bd$. Find the smallest possible value of
$$\frac{a}{b}+\frac{b}{c}+\frac{c}{d}+\frac{d}{a}.$$
\end{problem}
\begin{problem}[8895719454292056765]
Given a non-right triangle $ABC$ with $BC>AC>AB$. Two points $P_1 \neq P_2$ on the plane satisfy that, for $i=1,2$, if $AP_i, BP_i$ and $CP_i$ intersect the circumcircle of the triangle $ABC$ at $D_i, E_i$, and $F_i$, respectively, then $D_iE_i \perp D_iF_i$ and $D_iE_i = D_iF_i \neq 0$. Let the line $P_1P_2$ intersects the circumcircle of $ABC$ at $Q_1$ and $Q_2$. The Simson lines of $Q_1$, $Q_2$ with respect to $ABC$ intersect at $W$.

Prove that $W$ lies on the nine-point circle of $ABC$.
\end{problem}
\begin{problem}[1580707630770476037]
Two triangles intersect to form seven finite disjoint regions, six of which are triangles with area 1. The last region is a hexagon with area \(A\). Compute the minimum possible value of \(A\).
\end{problem}
\begin{problem}[876239022447910]
	Let $ABCC_1B_1A_1$ be a convex hexagon such that $AB=BC$, and suppose that the line segments $AA_1, BB_1$, and $CC_1$ have the same perpendicular bisector. Let the diagonals $AC_1$ and $A_1C$ meet at $D$, and denote by $\omega$ the circle $ABC$. Let $\omega$ intersect the circle $A_1BC_1$ again at $E \neq B$. Prove that the lines $BB_1$ and $DE$ intersect on $\omega$.
\end{problem}
\begin{problem}[723258861624579]
Let $n\geq 2$ be an integer and let $a_1, a_2, \ldots, a_n$ be positive real numbers with sum $1$. Prove that$$\sum_{k=1}^n \frac{a_k}{1-a_k}(a_1+a_2+\cdots+a_{k-1})^2 < \frac{1}{3}.$$
\end{problem}
\begin{problem}[733773583946080]
$AB$ and $AC$ are tangents to a circle $\omega$ with center $O$ at $B,C$ respectively. Point $P$ is a variable point on minor arc $BC$. The tangent at $P$ to $\omega$ meets $AB,AC$ at $D,E$ respectively. $AO$ meets $BP,CP$ at $U,V$ respectively. The line through $P$ perpendicular to $AB$ intersects $DV$ at $M$, and the line through $P$ perpendicular to $AC$ intersects $EU$ at $N$. Prove that as $P$ varies, $MN$ passes through a fixed point.
\end{problem}
\begin{problem}[5867489266334805897]
	Let $ABCDE$ be a pentagon inscribed in a circle $\Omega$. A line parallel to the segment $BC$ intersects $AB$ and $AC$ at points $S$ and $T$, respectively. Let $X$ be the intersection of the line $BE$ and $DS$, and $Y$ be the intersection of the line $CE$ and $DT$.

Prove that, if the line $AD$ is tangent to the circle $\odot(DXY)$, then the line $AE$ is tangent to the circle $\odot(EXY)$.
\end{problem}
\begin{problem}[3906812380515301028]
Given a triangle $ \triangle ABC $. Denote its incenter and orthocenter by $ I, H $, respectively. If there is a point $ K $ with$$ AH+AK = BH+BK = CH+CK $$Show that $ H, I, K $ are collinear.
\end{problem}
\begin{problem}[290912955085727393]
Let $n \geqslant 3$ be a positive integer and let $\left(a_{1}, a_{2}, \ldots, a_{n}\right)$ be a strictly increasing sequence of $n$ positive real numbers with sum equal to 2. Let $X$ be a subset of $\{1,2, \ldots, n\}$ such that the value of
\[
\left|1-\sum_{i \in X} a_{i}\right|
\]is minimised. Prove that there exists a strictly increasing sequence of $n$ positive real numbers $\left(b_{1}, b_{2}, \ldots, b_{n}\right)$ with sum equal to 2 such that
\[
\sum_{i \in X} b_{i}=1.
\]
\end{problem}
\begin{problem}[6193947856984766386]
Let $ABCD$ be a cyclic quadrilateral. Assume that the points $Q, A, B, P$ are collinear in this order, in such a way that the line $AC$ is tangent to the circle $ADQ$, and the line $BD$ is tangent to the circle $BCP$. Let $M$ and $N$ be the midpoints of segments $BC$ and $AD$, respectively. Prove that the following three lines are concurrent: line $CD$, the tangent of circle $ANQ$ at point $A$, and the tangent to circle $BMP$ at point $B$.
\end{problem}
\begin{problem}[215375559035207]
$ABC$ is an isosceles triangle, with $AB=AC$. $D$ is a moving point such that $AD\parallel BC$, $BD>CD$. Moving point $E$ is on the arc of $BC$ in circumcircle of $ABC$ not containing $A$, such that $EB<EC$. Ray $BC$ contains point $F$ with $\angle ADE=\angle DFE$. If ray $FD$ intersects ray $BA$ at $X$, and intersects ray $CA$ at $Y$, prove that $\angle XEY$ is a fixed angle.
\end{problem}
\begin{problem}[2252133047011954512]
On a flat plane in Camelot, King Arthur builds a labyrinth $\mathfrak{L}$ consisting of $n$ walls, each of which is an infinite straight line. No two walls are parallel, and no three walls have a common point. Merlin then paints one side of each wall entirely red and the other side entirely blue.

At the intersection of two walls there are four corners: two diagonally opposite corners where a red side and a blue side meet, one corner where two red sides meet, and one corner where two blue sides meet. At each such intersection, there is a two-way door connecting the two diagonally opposite corners at which sides of different colours meet.

After Merlin paints the walls, Morgana then places some knights in the labyrinth. The knights can walk through doors, but cannot walk through walls.

Let $k(\mathfrak{L})$ be the largest number $k$ such that, no matter how Merlin paints the labyrinth $\mathfrak{L},$ Morgana can always place at least $k$ knights such that no two of them can ever meet. For each $n,$ what are all possible values for $k(\mathfrak{L}),$ where $\mathfrak{L}$ is a labyrinth with $n$ walls?
\end{problem}
\begin{problem}[161342796381450]
For each integer $n\ge 1,$ compute the smallest possible value of\[\sum_{k=1}^{n}\left\lfloor\frac{a_k}{k}\right\rfloor\]over all permutations $(a_1,\dots,a_n)$ of $\{1,\dots,n\}.$
\end{problem}
\begin{problem}[796349431725149]
An acute, non-isosceles triangle $ABC$ is inscribed in a circle with centre $O$. A line go through $O$ and midpoint $I$ of $BC$ intersects $AB, AC$ at $E, F$ respectively. Let $D, G$ be reflections to $A$ over $O$ and circumcentre of $(AEF)$, respectively. Let $K$ be the reflection of $O$ over circumcentre of $(OBC)$.
$a)$ Prove that $D, G, K$ are collinear.
$b)$ Let $M, N$ are points on $KB, KC$ that $IM\perp AC$, $IN\perp AB$. The midperpendiculars of $IK$ intersects $MN$ at $H$. Assume that $IH$ intersects $AB, AC$ at $P, Q$ respectively. Prove that the circumcircle of $\triangle APQ$ intersects $(O)$ the second time at a point on $AI$.
\end{problem}
\begin{problem}[5514383858686655851]
Determine all functions $f:(0,\infty)\to\mathbb{R}$ satisfying$$\left(x+\frac{1}{x}\right)f(y)=f(xy)+f\left(\frac{y}{x}\right)$$for all $x,y>0$.
\end{problem}
\begin{problem}[7494618588207758150]
	An anti-Pascal triangle is an equilateral triangular array of numbers such that, except for the numbers in the bottom row, each number is the absolute value of the difference of the two numbers immediately below it. For example, the following is an anti-Pascal triangle with four rows which contains every integer from $1$ to $10$.
\[\begin{array}{
c@{\hspace{4pt}}c@{\hspace{4pt}}
c@{\hspace{4pt}}c@{\hspace{2pt}}c@{\hspace{2pt}}c@{\hspace{4pt}}c
} \vspace{4pt}
 & & & 4 & & &  \\\vspace{4pt}
 & & 2 & & 6 & &  \\\vspace{4pt}
 & 5 & & 7 & & 1 & \\\vspace{4pt}
 8 & & 3 & & 10 & & 9 \\\vspace{4pt}
\end{array}\]Does there exist an anti-Pascal triangle with $2018$ rows which contains every integer from $1$ to $1 + 2 + 3 + \dots + 2018$?

\end{problem}
\begin{problem}[3813623497653179264]
	The real numbers $a, b, c, d$ are such that $a\geq b\geq c\geq d>0$ and $a+b+c+d=1$. Prove that
\[(a+2b+3c+4d)a^ab^bc^cd^d<1\]
\end{problem}
\begin{problem}[70043882336455]
Let $A$ be a point in the plane, and $\ell$ a line not passing through $A$. Evan does not have a straightedge, but instead has a special compass which has the ability to draw a circle through three distinct noncollinear points. (The center of the circle is not marked in this process.) Additionally, Evan can mark the intersections between two objects drawn, and can mark an arbitrary point on a given object or on the plane.

(i) Can Evan construct* the reflection of $A$ over $\ell$?

(ii) Can Evan construct the foot of the altitude from $A$ to $\ell$?

*To construct a point, Evan must have an algorithm which marks the point in finitely many steps.
\end{problem}
\begin{problem}[2134021625648303394]
The infinite sequence $a_0,a _1, a_2, \dots$ of (not necessarily distinct) integers has the following properties: $0\le a_i \le i$ for all integers $i\ge 0$, and\[\binom{k}{a_0} + \binom{k}{a_1} + \dots + \binom{k}{a_k} = 2^k\]for all integers $k\ge 0$. Prove that all integers $N\ge 0$ occur in the sequence (that is, for all $N\ge 0$, there exists $i\ge 0$ with $a_i=N$).
\end{problem}
\begin{problem}[6654677204410680146]
In the plane, there are $n \geqslant 6$ pairwise disjoint disks $D_{1}, D_{2}, \ldots, D_{n}$ with radii $R_{1} \geqslant R_{2} \geqslant \ldots \geqslant R_{n}$. For every $i=1,2, \ldots, n$, a point $P_{i}$ is chosen in disk $D_{i}$. Let $O$ be an arbitrary point in the plane. Prove that\[O P_{1}+O P_{2}+\ldots+O P_{n} \geqslant R_{6}+R_{7}+\ldots+R_{n}.\](A disk is assumed to contain its boundary.)
\end{problem}
\begin{problem}[4892352754475215646]
	We say that a set $S$ of integers is rootiful if, for any positive integer $n$ and any $a_0, a_1, \cdots, a_n \in S$, all integer roots of the polynomial $a_0+a_1x+\cdots+a_nx^n$ are also in $S$. Find all rootiful sets of integers that contain all numbers of the form $2^a - 2^b$ for positive integers $a$ and $b$.
\end{problem}
\begin{problem}[318208660266829737]
Let $ABC$ be an acute-angled triangle with $AB \ne AC$, and let $I$ and $O$ be its incenter and circumcenter, respectively. Let the incircle touch $BC, CA$ and $AB$ at $D, E$ and $F$, respectively. Assume that the line through $I$ parallel to $EF$, the line through $D$ parallel to$ AO$, and the altitude from $A$ are concurrent. Prove that the concurrency point is the orthocenter of the triangle $ABC$.
\end{problem}
\begin{problem}[8799177804774743019]
In each square of a garden shaped like a $2022 \times 2022$ board, there is initially a tree of height $0$. A gardener and a lumberjack alternate turns playing the following game, with the gardener taking the first turn:
The gardener chooses a square in the garden. Each tree on that square and all the surrounding squares (of which there are at most eight) then becomes one unit taller.
The lumberjack then chooses four different squares on the board. Each tree of positive height on those squares then becomes one unit shorter.
We say that a tree is majestic if its height is at least $10^6$. Determine the largest $K$ such that the gardener can ensure there are eventually $K$ majestic trees on the board, no matter how the lumberjack plays.
\end{problem}
\begin{problem}[436681276656848]
For the quadrilateral $ABCD$, let $AC$ and $BD$ intersect at $E$, $AB$ and $CD$ intersect at $F$, and $AD$ and $BC$ intersect at $G$. Additionally, let $W, X, Y$, and $Z$ be the points of symmetry to $E$ with respect to $AB, BC, CD,$ and $DA$ respectively. Prove that one of the intersection points of $\odot(FWY)$ and $\odot(GXZ)$ lies on the line $FG$.
\end{problem}
\begin{problem}[528504335909385]
Given a triangle $ \triangle{ABC} $ whose incenter is $ I $ and $ A $-excenter is $ J $. $ A' $ is point so that $ AA' $ is a diameter of $ \odot\left(\triangle{ABC}\right) $. Define $ H_{1}, H_{2} $ to be the orthocenters of $ \triangle{BIA'} $ and $ \triangle{CJA'} $. Show that $ H_{1}H_{2} \parallel BC $
\end{problem}
\begin{problem}[80567267310692]
Let $n$ be a positive integer. Given is a subset $A$ of $\{0,1,...,5^n\}$ with $4n+2$ elements. Prove that there exist three elements $a<b<c$ from $A$ such that $c+2a>3b$.
\end{problem}
\begin{problem}[660403976209529]
A number is called Norwegian if it has three distinct positive divisors whose sum is equal to $2022$. Determine the smallest Norwegian number.
(Note: The total number of positive divisors of a Norwegian number is allowed to be larger than $3$.)
\end{problem}
\begin{problem}[208441124738479]
Let $f : \{ 1, 2, 3, \dots \} \to \{ 2, 3, \dots \}$ be a function such that $f(m + n) | f(m) + f(n) $ for all pairs $m,n$ of positive integers. Prove that there exists a positive integer $c > 1$ which divides all values of $f$.
\end{problem}
\begin{problem}[931951248564234]
Let $n > 3$ be a positive integer. Suppose that $n$ children are arranged in a circle, and $n$ coins are distributed between them (some children may have no coins). At every step, a child with at least 2 coins may give 1 coin to each of their immediate neighbors on the right and left. Determine all initial distributions of the coins from which it is possible that, after a finite number of steps, each child has exactly one coin.
\end{problem}
\begin{problem}[852531542088551]
	Given a triangle $ABC$ for which $\angle BAC \neq 90^{\circ}$, let $B_1, C_1$ be variable points on $AB,AC$, respectively. Let $B_2,C_2$ be the points on line $BC$ such that a spiral similarity centered at $A$ maps $B_1C_1$ to $C_2B_2$. Denote the circumcircle of $AB_1C_1$ by $\omega$. Show that if $B_1B_2$ and $C_1C_2$ concur on $\omega$ at a point distinct from $B_1$ and $C_1$, then $\omega$ passes through a fixed point other than $A$.
\end{problem}
\begin{problem}[239934686230450]
Let triangle$ABC(AB<AC)$ with incenter $I$ circumscribed in $\odot O$. Let $M,N$ be midpoint of arc $\widehat{BAC}$ and $\widehat{BC}$, respectively. $D$ lies on $\odot O$ so that $AD//BC$, and $E$ is tangency point of $A$-excircle of $\bigtriangleup ABC$. Point $F$ is in $\bigtriangleup ABC$ so that $FI//BC$ and $\angle BAF=\angle EAC$. Extend $NF$ to meet $\odot O$ at $G$, and extend $AG$ to meet line $IF$ at L. Let line $AF$ and $DI$ meet at $K$. Proof that $ML\bot NK$.
\end{problem}
\begin{problem}[5897111412933990257]
Let $ABC$ be a triangle with circumcircle $\Gamma$, and points $E$ and $F$ are chosen from sides $CA$, $AB$, respectively. Let the circumcircle of triangle $AEF$ and $\Gamma$ intersect again at point $X$. Let the circumcircles of triangle $ABE$ and $ACF$ intersect again at point $K$. Line $AK$ intersect with $\Gamma$ again at point $M$ other than $A$, and $N$ be the reflection point of $M$ with respect to line $BC$. Let $XN$ intersect with $\Gamma$ again at point $S$ other that $X$.

Prove that $SM$ is parallel to $BC$.
\end{problem}
\begin{problem}[591652153716935]
	Let $M$ be the midpoint of $BC$ of triangle $ABC$. The circle with diameter $BC$, $\omega$, meets $AB,AC$ at $D,E$ respectively. $P$ lies inside $\triangle ABC$ such that $\angle PBA=\angle PAC, \angle PCA=\angle PAB$, and $2PM\cdot DE=BC^2$. Point $X$ lies outside $\omega$ such that $XM\parallel AP$, and $\frac{XB}{XC}=\frac{AB}{AC}$. Prove that $\angle BXC +\angle BAC=90^{\circ}$.
\end{problem}
\begin{problem}[3159161448000677570]
Let $a > 1$ be a positive integer and $d > 1$ be a positive integer coprime to $a$. Let $x_1=1$, and for $k\geq 1$, define
$$x_{k+1} = \begin{cases}
x_k + d &\text{if } a \text{ does not divide } x_k \\
x_k/a & \text{if } a \text{ divides } x_k
\end{cases}$$Find, in terms of $a$ and $d$, the greatest positive integer $n$ for which there exists an index $k$ such that $x_k$ is divisible by $a^n$.
\end{problem}
\begin{problem}[819328919046836]
Which positive integers $n$ make the equation\[\sum_{i=1}^n \sum_{j=1}^n \left\lfloor \frac{ij}{n+1} \right\rfloor=\frac{n^2(n-1)}{4}\]true?
\end{problem}
\begin{problem}[797215984506934]
	Let $ABC$ be a triangle. Circle $\Gamma$ passes through $A$, meets segments $AB$ and $AC$ again at points $D$ and $E$ respectively, and intersects segment $BC$ at $F$ and $G$ such that $F$ lies between $B$ and $G$. The tangent to circle $BDF$ at $F$ and the tangent to circle $CEG$ at $G$ meet at point $T$. Suppose that points $A$ and $T$ are distinct. Prove that line $AT$ is parallel to $BC$.
\end{problem}
\begin{problem}[37921131297270]
	You are given a set of $n$ blocks, each weighing at least $1$; their total weight is $2n$. Prove that for every real number $r$ with $0 \leq r \leq 2n-2$ you can choose a subset of the blocks whose total weight is at least $r$ but at most $r + 2$.
\end{problem}
\begin{problem}[1168447466971762345]
	Let $I, O, \omega, \Omega$ be the incenter, circumcenter, the incircle, and the circumcircle, respectively, of a scalene triangle $ABC$. The incircle $\omega$ is tangent to side $BC$ at point $D$. Let $S$ be the point on the circumcircle $\Omega$ such that $AS, OI, BC$ are concurrent. Let $H$ be the orthocenter of triangle $BIC$. Point $T$ lies on $\Omega$ such that $\angle ATI$ is a right angle. Prove that the points $D, T, H, S$ are concyclic.
\end{problem}
\begin{problem}[719467452801051]
Let $ABC$ be a triangle with circumcircle $\Omega$ and incentre $I$. A line $\ell$ intersects the lines $AI$, $BI$, and $CI$ at points $D$, $E$, and $F$, respectively, distinct from the points $A$, $B$, $C$, and $I$. The perpendicular bisectors $x$, $y$, and $z$ of the segments $AD$, $BE$, and $CF$, respectively determine a triangle $\Theta$. Show that the circumcircle of the triangle $\Theta$ is tangent to $\Omega$.
\end{problem}
\begin{problem}[3417358984411200361]
Let $ABC$ be a triangle with circumcircle $\Omega$, circumcenter $O$ and orthocenter $H$. Let $S$ lie on $\Omega$ and $P$ lie on $BC$ such that $\angle ASP=90^\circ$, line $SH$ intersects the circumcircle of $\triangle APS$ at $X\neq S$. Suppose $OP$ intersects $CA,AB$ at $Q,R$, respectively, $QY,RZ$ are the altitude of $\triangle AQR$. Prove that $X,Y,Z$ are collinear.
\end{problem}
\begin{problem}[409146991986056]
For each prime $p$, construct a graph $G_p$ on $\{1,2,\ldots p\}$, where $m\neq n$ are adjacent if and only if $p$ divides $(m^{2} + 1-n)(n^{2} + 1-m)$. Prove that $G_p$ is disconnected for infinitely many $p$
\end{problem}
\begin{problem}[908587245178389]
Let $I$ be the incenter of triangle $ABC$, and $\ell$ be the perpendicular bisector of $AI$. Suppose that $P$ is on the circumcircle of triangle $ABC$, and line $AP$ and $\ell$ intersect at point $Q$. Point $R$ is on $\ell$ such that $\angle IPR = 90^{\circ}$.Suppose that line $IQ$ and the midsegment of $ABC$ that is parallel to $BC$ intersect at $M$. Show that $\angle AMR = 90^{\circ}$

(Note: In a triangle, a line connecting two midpoints is called a midsegment.)
\end{problem}
\begin{problem}[669395675904242]
	Two squirrels, Bushy and Jumpy, have collected 2021 walnuts for the winter. Jumpy numbers the walnuts from 1 through 2021, and digs 2021 little holes in a circular pattern in the ground around their favourite tree. The next morning Jumpy notices that Bushy had placed one walnut into each hole, but had paid no attention to the numbering. Unhappy, Jumpy decides to reorder the walnuts by performing a sequence of 2021 moves. In the $k$-th move, Jumpy swaps the positions of the two walnuts adjacent to walnut $k$.

Prove that there exists a value of $k$ such that, on the $k$-th move, Jumpy swaps some walnuts $a$ and $b$ such that $a<k<b$.
\end{problem}
\begin{problem}[8811824418974048155]
$ABCDE$ is a cyclic pentagon, with circumcentre $O$. $AB=AE=CD$. $I$ midpoint of $BC$. $J$ midpoint of $DE$. $F$ is the orthocentre of $\triangle ABE$, and $G$ the centroid of $\triangle AIJ$.$CE$ intersects $BD$ at $H$, $OG$ intersects $FH$ at $M$. Show that $AM\perp CD$.
\end{problem}
\begin{problem}[6306108494297192985]
Carl is given three distinct non-parallel lines $\ell_1, \ell_2, \ell_3$ and a circle $\omega$ in the plane. In addition to a normal straightedge, Carl has a special straightedge which, given a line $\ell$ and a point $P$, constructs a new line passing through $P$ parallel to $\ell$. (Carl does not have a compass.) Show that Carl can construct a triangle with circumcircle $\omega$ whose sides are parallel to $\ell_1,\ell_2,\ell_3$ in some order.
\end{problem}
\begin{problem}[3245291910836201005]
	Let $P$ be a point inside triangle $ABC$. Let $AP$ meet $BC$ at $A_1$, let $BP$ meet $CA$ at $B_1$, and let $CP$ meet $AB$ at $C_1$. Let $A_2$ be the point such that $A_1$ is the midpoint of $PA_2$, let $B_2$ be the point such that $B_1$ is the midpoint of $PB_2$, and let $C_2$ be the point such that $C_1$ is the midpoint of $PC_2$. Prove that points $A_2, B_2$, and $C_2$ cannot all lie strictly inside the circumcircle of triangle $ABC$.
\end{problem}
\begin{problem}[1222382895728709073]
	Given a triangle $ABC$, a circle $\Omega$ is tangent to $AB,AC$ at $B,C,$ respectively. Point $D$ is the midpoint of $AC$, $O$ is the circumcenter of triangle $ABC$. A circle $\Gamma$ passing through $A,C$ intersects the minor arc $BC$ on $\Omega$ at $P$, and intersects $AB$ at $Q$. It is known that the midpoint $R$ of minor arc $PQ$ satisfies that $CR \perp AB$. Ray $PQ$ intersects line $AC$ at $L$, $M$ is the midpoint of $AL$, $N$ is the midpoint of $DR$, and $X$ is the projection of $M$ onto $ON$. Prove that the circumcircle of triangle $DNX$ passes through the center of $\Gamma$.
\end{problem}
\begin{problem}[2003233604438068678]
Given a triangle $ABC$ and a point $O$ on a plane. Let $\Gamma$ be the circumcircle of $ABC$. Suppose that $CO$ intersects with $AB$ at $D$, and $BO$ and $CA$ intersect at $E$. Moreover, suppose that $AO$ intersects with $\Gamma$ at $A,F$. Let $I$ be the other intersection of $\Gamma$ and the circumcircle of $ADE$, and $Y$ be the other intersection of $BE$ and the circumcircle of $CEI$, and $Z$ be the other intersection of $CD$ and the circumcircle of $BDI$. Let $T$ be the intersection of the two tangents of $\Gamma$ at $B,C$, respectively. Lastly, suppose that $TF$ intersects with $\Gamma$ again at $U$, and the reflection of $U$ w.r.t. $BC$ is $G$.

Show that $F,I,G,O,Y,Z$ are concyclic.
\end{problem}
\begin{problem}[448881061747528]
A magician intends to perform the following trick. She announces a positive integer $n$, along with $2n$ real numbers $x_1 < \dots < x_{2n}$, to the audience. A member of the audience then secretly chooses a polynomial $P(x)$ of degree $n$ with real coefficients, computes the $2n$ values $P(x_1), \dots , P(x_{2n})$, and writes down these $2n$ values on the blackboard in non-decreasing order. After that the magician announces the secret polynomial to the audience. Can the magician find a strategy to perform such a trick?
\end{problem}
\begin{problem}[7220404010846068686]
	Let $ABC$ be a acute, non-isosceles triangle. $D,\ E,\ F$ are the midpoints of sides $AB,\ BC,\ AC$, resp. Denote by $(O),\ (O')$ the circumcircle and Euler circle of $ABC$. An arbitrary point $P$ lies inside triangle $DEF$ and $DP,\ EP,\ FP$ intersect $(O')$ at $D',\ E',\ F'$, resp. Point $A'$ is the point such that $D'$ is the midpoint of $AA'$. Points $B',\ C'$ are defined similarly.
a. Prove that if $PO=PO'$ then $O\in(A'B'C')$;
b. Point $A'$ is mirrored by $OD$, its image is $X$. $Y,\ Z$ are created in the same manner. $H$ is the orthocenter of $ABC$ and $XH,\ YH,\ ZH$ intersect $BC, AC, AB$ at $M,\ N,\ L$ resp. Prove that $M,\ N,\ L$ are collinear.
\end{problem}
\begin{problem}[5873161915777778529]
In the acute-angled triangle $ABC$, the point $F$ is the foot of the altitude from $A$, and $P$ is a point on the segment $AF$. The lines through $P$ parallel to $AC$ and $AB$ meet $BC$ at $D$ and $E$, respectively. Points $X \ne A$ and $Y \ne A$ lie on the circles $ABD$ and $ACE$, respectively, such that $DA = DX$ and $EA = EY$.
Prove that $B, C, X,$ and $Y$ are concyclic.
\end{problem}
\begin{problem}[16776483958513]
Find all pairs $(k,n)$ of positive integers such that\[ k!=(2^n-1)(2^n-2)(2^n-4)\cdots(2^n-2^{n-1}). \]
\end{problem}
\begin{problem}[844684477828422]
Let point $H$ be the orthocenter of a scalene triangle $ABC$. Line $AH$ intersects with the circumcircle $\Omega$ of triangle $ABC$ again at point $P$. Line $BH, CH$ meets with $AC,AB$ at point $E$ and $F$, respectively. Let $PE, PF$ meet $\Omega$ again at point $Q,R$, respectively. Point $Y$ lies on $\Omega$ so that lines $AY,QR$ and $EF$ are concurrent. Prove that $PY$ bisects $EF$.
\end{problem}
\begin{problem}[8152181601565653036]
Let \(D\) be a point on segment \(PQ\). Let \(\omega\) be a fixed circle passing through \(D\), and let \(A\) be a variable point on \(\omega\). Let \(X\) be the intersection of the tangent to the circumcircle of \(\triangle ADP\) at \(P\) and the tangent to the circumcircle of \(\triangle ADQ\) at \(Q\). Show that as \(A\) varies, \(X\) lies on a fixed line.
\end{problem}
\begin{problem}[4389998719836463980]
Let $ABCD$ be a parallelogram with $AC=BC.$ A point $P$ is chosen on the extension of ray $AB$ past $B.$ The circumcircle of $ACD$ meets the segment $PD$ again at $Q.$ The circumcircle of triangle $APQ$ meets the segment $PC$ at $R.$ Prove that lines $CD,AQ,BR$ are concurrent.
\end{problem}
\begin{problem}[8255863576892581507]
	Let $ABC$ be an acute triangle with orthocenter $H$, and let $P$ be a point on the nine-point circle of $ABC$. Lines $BH, CH$ meet the opposite sides $AC, AB$ at $E, F$, respectively. Suppose that the circumcircles $(EHP), (FHP)$ intersect lines $CH, BH$ a second time at $Q,R$, respectively. Show that as $P$ varies along the nine-point circle of $ABC$, the line $QR$ passes through a fixed point.
\end{problem}
\begin{problem}[9153191064326230951]
Let scalene triangle $ABC$ have altitudes $AD, BE, CF$ and circumcenter $O$. The circumcircles of $\triangle ABC$ and $\triangle ADO$ meet at $P \ne A$. The circumcircle of $\triangle ABC$ meets lines $PE$ at $X \ne P$ and $PF$ at $Y \ne P$. Prove that $XY \parallel BC$.
\end{problem}
\begin{problem}[409530198849693]
In a cyclic convex hexagon $ABCDEF$, $AB$ and $DC$ intersect at $G$, $AF$ and $DE$ intersect at $H$. Let $M, N$ be the circumcenters of $BCG$ and $EFH$, respectively. Prove that the $BE$, $CF$ and $MN$ are concurrent.
\end{problem}
\begin{problem}[282712203118607]
	Let $ABC$ be an acute-angled triangle with $AC > AB$, let $O$ be its circumcentre, and let $D$ be a point on the segment $BC$. The line through $D$ perpendicular to $BC$ intersects the lines $AO, AC,$ and $AB$ at $W, X,$ and $Y,$ respectively. The circumcircles of triangles $AXY$ and $ABC$ intersect again at $Z \ne A$.
Prove that if $W \ne D$ and $OW = OD,$ then $DZ$ is tangent to the circle $AXY.$
\end{problem}
\begin{problem}[1891712635906763103]
	Let $BM$ be a median in an acute-angled triangle $ABC$. A point $K$ is chosen on the line through $C$ tangent to the circumcircle of $\vartriangle BMC$ so that $\angle KBC = 90^\circ$. The segments $AK$ and $BM$ meet at $J$. Prove that the circumcenter of $\triangle BJK$ lies on the line $AC$.
\end{problem}
\begin{problem}[6302540840099076878]
Let $ABC$ be an isosceles triangle with $BC=CA$, and let $D$ be a point inside side $AB$ such that $AD< DB$. Let $P$ and $Q$ be two points inside sides $BC$ and $CA$, respectively, such that $\angle DPB = \angle DQA = 90^{\circ}$. Let the perpendicular bisector of $PQ$ meet line segment $CQ$ at $E$, and let the circumcircles of triangles $ABC$ and $CPQ$ meet again at point $F$, different from $C$.
Suppose that $P$, $E$, $F$ are collinear. Prove that $\angle ACB = 90^{\circ}$.
\end{problem}
\begin{problem}[8866273454792491736]
	Let $r>1$ be a rational number. Alice plays a solitaire game on a number line. Initially there is a red bead at $0$ and a blue bead at $1$. In a move, Alice chooses one of the beads and an integer $k \in \mathbb{Z}$. If the chosen bead is at $x$, and the other bead is at $y$, then the bead at $x$ is moved to the point $x'$ satisfying $x'-y=r^k(x-y)$.

Find all $r$ for which Alice can move the red bead to $1$ in at most $2021$ moves.
\end{problem}
\begin{problem}[227919487650283]
Let $ABC$ be an acute triangle with orthocenter $H$ and circumcircle $\Omega$. Let $M$ be the midpoint of side $BC$. Point $D$ is chosen from the minor arc $BC$ on $\Gamma$ such that $\angle BAD = \angle MAC$. Let $E$ be a point on $\Gamma$ such that $DE$ is perpendicular to $AM$, and $F$ be a point on line $BC$ such that $DF$ is perpendicular to $BC$. Lines $HF$ and $AM$ intersect at point $N$, and point $R$ is the reflection point of $H$ with respect to $N$.

Prove that $\angle AER + \angle DFR = 180^\circ$.
\end{problem}
\begin{problem}[493493847475466779]
Let $ABC$ be a triangle and let $H$ be the orthogonal projection of $A$ on the line $BC$. Let $K$ be a point on the segment $AH$ such that $AH = 3 KH$. Let $O$ be the circumcenter of triangle $ABC$ and let $M$ and $N$ be the midpoints of sides $AC$ and $AB$ respectively. The lines $KO$ and $MN$ meet at a point $Z$ and the perpendicular at $Z$ to $OK$ meets lines $AB, AC$ at $X$ and $Y$ respectively. Show that $\angle XKY = \angle CKB$.
\end{problem}
\begin{problem}[423911944927735]
In acute $\triangle ABC$, $O$ is the circumcenter, $I$ is the incenter. The incircle touches $BC,CA,AB$ at $D,E,F$. And the points $K,M,N$ are the midpoints of $BC,CA,AB$ respectively.

a) Prove that the lines passing through $D,E,F$ in parallel with $IK,IM,IN$ respectively are concurrent.

b) Points $T,P,Q$ are the middle points of the major arc $BC,CA,AB$ on $\odot ABC$. Prove that the lines passing through $D,E,F$ in parallel with $IT,IP,IQ$ respectively are concurrent.
\end{problem}
\begin{problem}[5835156231907738776]
Given triangle $ABC$ with $A$-excenter $I_A$, the foot of the perpendicular from $I_A$ to $BC$ is $D$. Let the midpoint of segment $I_AD$ be $M$, $T$ lies on arc $BC$(not containing $A$) satisfying $\angle BAT=\angle DAC$, $I_AT$ intersects the circumcircle of $ABC$ at $S\neq T$. If $SM$ and $BC$ intersect at $X$, the perpendicular bisector of $AD$ intersects $AC,AB$ at $Y,Z$ respectively, prove that $AX,BY,CZ$ are concurrent.
\end{problem}
\begin{problem}[275429739915708]
Consider a $100\times 100$ square unit lattice $\textbf{L}$ (hence $\textbf{L}$ has $10000$ points). Suppose $\mathcal{F}$ is a set of polygons such that all vertices of polygons in $\mathcal{F}$ lie in $\textbf{L}$ and every point in $\textbf{L}$ is the vertex of exactly one polygon in $\mathcal{F}.$ Find the maximum possible sum of the areas of the polygons in $\mathcal{F}.$
\end{problem}
\begin{problem}[443006607452241]
Let $x_1, x_2, \dots, x_n$ be different real numbers. Prove that
\[\sum_{1 \leqslant i \leqslant n} \prod_{j \neq i} \frac{1-x_{i} x_{j}}{x_{i}-x_{j}}=\left\{\begin{array}{ll}
0, & \text { if } n \text { is even; } \\
1, & \text { if } n \text { is odd. }
\end{array}\right.\]
\end{problem}
\begin{problem}[8700998965901287095]
Let \(ABC\) be an acute triangle with circumcircle \(\omega\). Let \(P\) be a variable point on the arc \(BC\) of \(\omega\) not containing \(A\). Squares \(BPDE\) and \(PCFG\) are constructed such that \(A\), \(D\), \(E\) lie on the same side of line \(BP\) and \(A\), \(F\), \(G\) lie on the same side of line \(CP\). Let \(H\) be the intersection of lines \(DE\) and \(FG\). Show that as \(P\) varies, \(H\) lies on a fixed circle.
\end{problem}
\begin{problem}[8609709793627283757]
	Define the sequence $a_0,a_1,a_2,\hdots$ by $a_n=2^n+2^{\lfloor n/2\rfloor}$. Prove that there are infinitely many terms of the sequence which can be expressed as a sum of (two or more) distinct terms of the sequence, as well as infinitely many of those which cannot be expressed in such a way.
\end{problem}
\begin{problem}[1837105952530316058]
Let $k\ge2$ be an integer. Find the smallest integer $n \ge k+1$ with the property that there exists a set of $n$ distinct real numbers such that each of its elements can be written as a sum of $k$ other distinct elements of the set.
\end{problem}
\begin{problem}[1965233157265405983]
Given a triangle $ \triangle{ABC} $. Denote its incircle and circumcircle by $ \omega, \Omega $, respectively. Assume that $ \omega $ tangents the sides $ AB, AC $ at $ F, E $, respectively. Then, let the intersections of line $ EF $ and $ \Omega $ to be $ P,Q $. Let $ M $ to be the mid-point of $ BC $. Take a point $ R $ on the circumcircle of $ \triangle{MPQ} $, say $ \Gamma $, such that $ MR \perp EF $. Prove that the line $ AR $, $ \omega $ and $ \Gamma $ intersect at one point.
\end{problem}
\begin{problem}[5299971832672937326]
Let $ABCD$ be a cyclic quadrilateral. Points $K, L, M, N$ are chosen on $AB, BC, CD, DA$ such that $KLMN$ is a rhombus with $KL \parallel AC$ and $LM \parallel BD$. Let $\omega_A, \omega_B, \omega_C, \omega_D$ be the incircles of $\triangle ANK, \triangle BKL, \triangle CLM, \triangle DMN$.

Prove that the common internal tangents to $\omega_A$, and $\omega_C$ and the common internal tangents to $\omega_B$ and $\omega_D$ are concurrent.
\end{problem}
\begin{problem}[1613309914397651478]
Let $ABCD$ be a convex quadrilateral with $\angle B < \angle A < 90^{o}$. Let $I$ be the midpoint of $AB$ and $S$ the intersection of $AD$ and $BC$. Let $R$ be a variable point inside the triangle $SAB$ such that $\angle ASR = \angle BSR$. On the straight lines $AR, BR$ , take the points $E, F$, respectively so that $BE , AF$ are parallel to $RS$. Suppose that $EF$ intersects the circumcircle of triangle $SAB$ at points $H, K$. On the segment $AB$, take points $M , N$ such that $\angle AHM =\angle BHI$ , $\angle BKN = \angle AKI$.

a) Prove that the center $J$ of the circumcircle of triangle $SMN$ lies on a fixed line.

b) On $BE, AF$ , take the points $P, Q$ respectively so that $CP$ is parallel to $SE$ and $DQ$ is parallel to $SF$. The lines $SE, SF$ intersect the circle $(SAB)$, respectively, at $U, V$. Let $G$ be the intersection of $AU$ and $BV$. Prove that the median of vertex $G$ of the triangle $GPQ$ always passes through a fixed point .
\end{problem}
\begin{problem}[627600286851318227]
Find all triples $(a,b,p)$ of positive integers with $p$ prime and\[ a^p=b!+p. \]
\end{problem}
\begin{problem}[4308913658510445082]
Let $ABCD$ be a convex quadrilateral, the incenters of $\triangle ABC$ and $\triangle ADC$ are $I,J$, respectively. It is known that $AC,BD,IJ$ concurrent at a point $P$. The line perpendicular to $BD$ through $P$ intersects with the outer angle bisector of $\angle BAD$ and the outer angle bisector $\angle BCD$ at $E,F$, respectively. Show that $PE=PF$.
\end{problem}
\begin{problem}[8851048763094130212]
	Let $ABCD$ be a quadrilateral inscribed in a circle $\Omega.$ Let the tangent to $\Omega$ at $D$ meet rays $BA$ and $BC$ at $E$ and $F,$ respectively. A point $T$ is chosen inside $\triangle ABC$ so that $\overline{TE}\parallel\overline{CD}$ and $\overline{TF}\parallel\overline{AD}.$ Let $K\ne D$ be a point on segment $DF$ satisfying $TD=TK.$ Prove that lines $AC,DT,$ and $BK$ are concurrent.
\end{problem}
\begin{problem}[599825051147866097]
Show that $n!=a^{n-1}+b^{n-1}+c^{n-1}$ has only finitely many solutions in positive integers.
\end{problem}
\begin{problem}[6783316811528119504]
Let $S$ be an infinite set of positive integers, such that there exist four pairwise distinct $a,b,c,d \in S$ with $\gcd(a,b) \neq \gcd(c,d)$. Prove that there exist three pairwise distinct $x,y,z \in S$ such that $\gcd(x,y)=\gcd(y,z) \neq \gcd(z,x)$.
\end{problem}
\begin{problem}[1440964279096111130]
	Let $a$ be a positive integer. We say that a positive integer $b$ is $a$-good if $\tbinom{an}{b}-1$ is divisible by $an+1$ for all positive integers $n$ with $an \geq b$. Suppose $b$ is a positive integer such that $b$ is $a$-good, but $b+2$ is not $a$-good. Prove that $b+1$ is prime.
\end{problem}
\begin{problem}[3859961452154270883]
	A deck of $n > 1$ cards is given. A positive integer is written on each card. The deck has the property that the arithmetic mean of the numbers on each pair of cards is also the geometric mean of the numbers on some collection of one or more cards.
For which $n$ does it follow that the numbers on the cards are all equal?
\end{problem}
\begin{problem}[7500559455615129254]
For every positive integer $N$, determine the smallest real number $b_{N}$ such that, for all real $x$,
\[
\sqrt[N]{\frac{x^{2 N}+1}{2}} \leqslant b_{N}(x-1)^{2}+x .
\]
\end{problem}
\begin{problem}[5990443173263547430]
	Given a fixed circle $(O)$ and two fixed points $B, C$ on that circle, let $A$ be a moving point on $(O)$ such that $\triangle ABC$ is acute and scalene. Let $I$ be the midpoint of $BC$ and let $AD, BE, CF$ be the three heights of $\triangle ABC$. In two rays $\overrightarrow{FA}, \overrightarrow{EA}$, we pick respectively $M,N$ such that $FM = CE, EN = BF$. Let $L$ be the intersection of $MN$ and $EF$, and let $G \neq L$ be the second intersection of $(LEN)$ and $(LFM)$.

a) Show that the circle $(MNG)$ always goes through a fixed point.

b) Let $AD$ intersects $(O)$ at $K \neq A$. In the tangent line through $D$ of $(DKI)$, we pick $P,Q$ such that $GP \parallel AB, GQ \parallel AC$. Let $T$ be the center of $(GPQ)$. Show that $GT$ always goes through a fixed point.
\end{problem}
\begin{problem}[8053761138620448460]
	Let $ABC$ be a scalene triangle, and points $O$ and $H$ be its circumcenter and orthocenter, respectively. Point $P$ lies inside triangle $AHO$ and satisfies $\angle AHP = \angle POA$. Let $M$ be the midpoint of segment $\overline{OP}$. Suppose that $BM$ and $CM$ intersect with the circumcircle of triangle $ABC$ again at $X$ and $Y$, respectively.

Prove that line $XY$ passes through the circumcenter of triangle $APO$.
\end{problem}
\begin{problem}[8330669807899443473]
Let $ABC$ be an acute scalene triangle, and let $A_1, B_1, C_1$ be the feet of the altitudes from $A, B, C$. Let $A_2$ be the intersection of the tangents to the circle $ABC$ at $B, C$ and define $B_2, C_2$ similarly. Let $A_2A_1$ intersect the circle $A_2B_2C_2$ again at $A_3$ and define $B_3, C_3$ similarly. Show that the circles $AA_1A_3, BB_1B_3$, and $CC_1C_3$ all have two common points, $X_1$ and $X_2$ which both lie on the Euler line of the triangle $ABC$.
\end{problem}
\begin{problem}[518384374486289]
	Let $O$ be the center of the equilateral triangle $ABC$. Pick two points $P_1$ and $P_2$ other than $B$, $O$, $C$ on the circle $\odot(BOC)$ so that on this circle $B$, $P_1$, $P_2$, $O$, $C$ are placed in this order. Extensions of $BP_1$ and $CP_1$ intersects respectively with side $CA$ and $AB$ at points $R$ and $S$. Line $AP_1$ and $RS$ intersects at point $Q_1$. Analogously point $Q_2$ is defined. Let $\odot(OP_1Q_1)$ and $\odot(OP_2Q_2)$ meet again at point $U$ other than $O$.

Prove that $2\,\angle Q_2UQ_1 + \angle Q_2OQ_1 = 360^\circ$.

Remark. $\odot(XYZ)$ denotes the circumcircle of triangle $XYZ$.
\end{problem}
\begin{problem}[15595788767204175]
Let \(ABC\) be an acute scalene triangle with orthocenter \(H\). Line \(BH\) intersects \(\overline{AC}\) at \(E\) and line \(CH\) intersects \(\overline{AB}\) at \(F\). Let \(X\) be the foot of the perpendicular from \(H\) to the line through \(A\) parallel to \(\overline{EF}\). Point \(B_1\) lies on line \(XF\) such that \(\overline{BB_1}\) is parallel to \(\overline{AC}\), and point \(C_1\) lies on line \(XE\) such that \(\overline{CC_1}\) is parallel to \(\overline{AB}\). Prove that points \(B\), \(C\), \(B_1\), \(C_1\) are concyclic.
\end{problem}
\begin{problem}[9026100911884959358]
Let $n$ be a positive integer, and set $N=2^{n}$. Determine the smallest real number $a_{n}$ such that, for all real $x$,
\[
\sqrt[N]{\frac{x^{2 N}+1}{2}} \leqslant a_{n}(x-1)^{2}+x .
\]
\end{problem}
\begin{problem}[857598260795435]
Let $ ABCD $ be a rhombus with center $ O. $ $ P $ is a point lying on the side $ AB. $ Let $ I, $ $ J, $ and $ L $ be the incenters of triangles $ PCD, $ $ PAD, $ and $PBC, $ respectively. Let $ H $ and $ K $ be orthocenters of triangles $ PLB $ and $ PJA, $ respectively.

Prove that $ OI \perp HK. $
\end{problem}
\begin{problem}[569685816807741]
Determine all pairs $(n, k)$ of distinct positive integers such that there exists a positive integer $s$ for which the number of divisors of $sn$ and of $sk$ are equal.
\end{problem}
\begin{problem}[607556370102952]
Let $\Omega$ be the circumcircle of an acute triangle $ABC$. Points $D$, $E$, $F$ are the midpoints of the inferior arcs $BC$, $CA$, $AB$, respectively, on $\Omega$. Let $G$ be the antipode of $D$ in $\Omega$. Let $X$ be the intersection of lines $GE$ and $AB$, while $Y$ the intersection of lines $FG$ and $CA$. Let the circumcenters of triangles $BEX$ and $CFY$ be points $S$ and $T$, respectively. Prove that $D$, $S$, $T$ are collinear.
\end{problem}
\begin{problem}[8417327567048605288]
Let $ABCDE$ be a convex pentagon such that $BC=DE$. Assume that there is a point $T$ inside $ABCDE$ with $TB=TD,TC=TE$ and $\angle ABT = \angle TEA$. Let line $AB$ intersect lines $CD$ and $CT$ at points $P$ and $Q$, respectively. Assume that the points $P,B,A,Q$ occur on their line in that order. Let line $AE$ intersect $CD$ and $DT$ at points $R$ and $S$, respectively. Assume that the points $R,E,A,S$ occur on their line in that order. Prove that the points $P,S,Q,R$ lie on a circle.
\end{problem}
\begin{problem}[120381541018683]
Given any set $S$ of positive integers, show that at least one of the following two assertions holds:

(1) There exist distinct finite subsets $F$ and $G$ of $S$ such that $\sum_{x\in F}1/x=\sum_{x\in G}1/x$;

(2) There exists a positive rational number $r<1$ such that $\sum_{x\in F}1/x\neq r$ for all finite subsets $F$ of $S$.
\end{problem}
\begin{problem}[3353450172272500341]
Let $ABCD$ be a cyclic quadrilateral. Let $DA$ and $BC$ intersect at $E$ and let $AB$ and $CD$
intersect at $F$. Assume that $A, E, F$ all lie on the same side of $BD$. Let $P$ be on segment $DA$
such that $\angle CPD = \angle CBP$, and let $Q$ be on segment $CD$ such that $\angle DQA = \angle QBA$. Let $AC$ and $PQ$ meet at $X$. Prove that, if $EX = EP$, then $EF$ is perpendicular to $AC$.
\end{problem}
\begin{problem}[7948249970111159954]
	A $\pm 1$-sequence is a sequence of $2022$ numbers $a_1, \ldots, a_{2022},$ each equal to either $+1$ or $-1$. Determine the largest $C$ so that, for any $\pm 1$-sequence, there exists an integer $k$ and indices $1 \le t_1 < \ldots < t_k \le 2022$ so that $t_{i+1} - t_i \le 2$ for all $i$, and$$\left| \sum_{i = 1}^{k} a_{t_i} \right| \ge C.$$
\end{problem}
\begin{problem}[1121095467606378762]
	Let $\Gamma, \Gamma_1, \Gamma_2$ be mutually tangent circles. The three circles are also tangent to a line $l$. Let $\Gamma, \Gamma_1$ be tangent to each other at $B_1$, $\Gamma, \Gamma_2$ be tangent to each other at $B_2$, $\Gamma_1, \Gamma_2$ be tangent to each other at $C$. $\Gamma, \Gamma_1, \Gamma_2$ are tangent to $l$ at $A, A_1, A_2$ respectively, where $A$ is between $A_1,A_2$. Let $D_1 = A_1C \cap A_2B_2, D_2 = A_2C \cap A_1B_1$. Prove that $D_1D_2$ is parallel to $l$.
\end{problem}
\begin{problem}[6919176010062551987]
Find all positive integers $n>2$ such that
$$ n! \mid \prod_{ p<q\le n, p,q \, \text{primes}} (p+q)$$
\end{problem}
\begin{problem}[3923745101517032298]
	Let $a_0,a_1,a_2,\dots $ be a sequence of real numbers such that $a_0=0, a_1=1,$ and for every $n\geq 2$ there exists $1 \leq k \leq n$ satisfying\[ a_n=\frac{a_{n-1}+\dots + a_{n-k}}{k}. \]Find the maximum possible value of $a_{2018}-a_{2017}$.
\end{problem}
\begin{problem}[728988632553727]
Let $ABCD$ be a convex quadrilateral with $\angle ABC>90$, $CDA>90$ and $\angle DAB=\angle BCD$. Denote by $E$ and $F$ the reflections of $A$ in lines $BC$ and $CD$, respectively. Suppose that the segments $AE$ and $AF$ meet the line $BD$ at $K$ and $L$, respectively. Prove that the circumcircles of triangles $BEK$ and $DFL$ are tangent to each other.
\end{problem}
\begin{problem}[119129720704350]
Let $H$ be the orthocenter of a given triangle $ABC$. Let $BH$ and $AC$ meet at a point $E$, and $CH$ and $AB$ meet at $F$. Suppose that $X$ is a point on the line $BC$. Also suppose that the circumcircle of triangle $BEX$ and the line $AB$ intersect again at $Y$, and the circumcircle of triangle $CFX$ and the line $AC$ intersect again at $Z$.
Show that the circumcircle of triangle $AYZ$ is tangent to the line $AH$.
\end{problem}
\begin{problem}[8782897210450267045]
Let $\mathbb{Q}_{>0}$ denote the set of all positive rational numbers. Determine all functions $f:\mathbb{Q}_{>0}\to \mathbb{Q}_{>0}$ satisfying$$f(x^2f(y)^2)=f(x)^2f(y)$$for all $x,y\in\mathbb{Q}_{>0}$
\end{problem}
\begin{problem}[6025085618534905645]
	Let $ABCD$ be a cyclic quadrilateral whose sides have pairwise different lengths. Let $O$ be the circumcenter of $ABCD$. The internal angle bisectors of $\angle ABC$ and $\angle ADC$ meet $AC$ at $B_1$ and $D_1$, respectively. Let $O_B$ be the center of the circle which passes through $B$ and is tangent to $\overline{AC}$ at $D_1$. Similarly, let $O_D$ be the center of the circle which passes through $D$ and is tangent to $\overline{AC}$ at $B_1$.

Assume that $\overline{BD_1} \parallel \overline{DB_1}$. Prove that $O$ lies on the line $\overline{O_BO_D}$.
\end{problem}
\begin{problem}[296367141382799]
Given a triangle $ \triangle{ABC} $ with orthocenter $ H $. On its circumcenter, choose an arbitrary point $ P $ (other than $ A,B,C $) and let $ M $ be the mid-point of $ HP $. Now, we find three points $ D,E,F $ on the line $ BC, CA, AB $, respectively, such that $ AP \parallel HD, BP \parallel HE, CP \parallel HF $. Show that $ D, E, F, M $ are colinear.
\end{problem}
\begin{problem}[633974672407561]
Let $(a_n)_{n\geq 1}$ be a sequence of positive real numbers with the property that
$$(a_{n+1})^2 + a_na_{n+2} \leq a_n + a_{n+2}$$for all positive integers $n$. Show that $a_{2022}\leq 1$.
\end{problem}
\begin{problem}[47893544380608]
	Let $p$ be an odd prime, and put $N=\frac{1}{4} (p^3 -p) -1.$ The numbers $1,2, \dots, N$ are painted arbitrarily in two colors, red and blue. For any positive integer $n \leqslant N,$ denote $r(n)$ the fraction of integers $\{ 1,2, \dots, n \}$ that are red.
Prove that there exists a positive integer $a \in \{ 1,2, \dots, p-1\}$ such that $r(n) \neq a/p$ for all $n = 1,2, \dots , N.$
\end{problem}
\begin{problem}[639126468624733]
Let $ABCDEF$ be a hexagon inscribed in a circle $\Omega$ such that triangles $ACE$ and $BDF$ have the same orthocenter. Suppose that segments $BD$ and $DF$ intersect $CE$ at $X$ and $Y$, respectively. Show that there is a point common to $\Omega$, the circumcircle of $DXY$, and the line through $A$ perpendicular to $CE$.
\end{problem}
\begin{problem}[8528437132500966626]
	Let $ABC$ be an acute triangle with orthocenter $H$ and circumcircle $\Gamma$. Let $BH$ intersect $AC$ at $E$, and let $CH$ intersect $AB$ at $F$. Let $AH$ intersect $\Gamma$ again at $P \neq A$. Let $PE$ intersect $\Gamma$ again at $Q \neq P$. Prove that $BQ$ bisects segment $\overline{EF}$.
\end{problem}
\begin{problem}[712971117639738]
Let $\mathcal{A}$ denote the set of all polynomials in three variables $x, y, z$ with integer coefficients. Let $\mathcal{B}$ denote the subset of $\mathcal{A}$ formed by all polynomials which can be expressed as
\begin{align*}
(x + y + z)P(x, y, z) + (xy + yz + zx)Q(x, y, z) + xyzR(x, y, z)
\end{align*}with $P, Q, R \in \mathcal{A}$. Find the smallest non-negative integer $n$ such that $x^i y^j z^k \in \mathcal{B}$ for all non-negative integers $i, j, k$ satisfying $i + j + k \geq n$.
\end{problem}
\begin{problem}[8690567757444826166]
	A social network has $2019$ users, some pairs of whom are friends. Whenever user $A$ is friends with user $B$, user $B$ is also friends with user $A$. Events of the following kind may happen repeatedly, one at a time:
Three users $A$, $B$, and $C$ such that $A$ is friends with both $B$ and $C$, but $B$ and $C$ are not friends, change their friendship statuses such that $B$ and $C$ are now friends, but $A$ is no longer friends with $B$, and no longer friends with $C$. All other friendship statuses are unchanged.
Initially, $1010$ users have $1009$ friends each, and $1009$ users have $1010$ friends each. Prove that there exists a sequence of such events after which each user is friends with at most one other user.

\end{problem}
\begin{problem}[822921222405372]
Let $n\ge 3$ be a fixed integer. There are $m\ge n+1$ beads on a circular necklace. You wish to paint the beads using $n$ colors, such that among any $n+1$ consecutive beads every color appears at least once. Find the largest value of $m$ for which this task is $\emph{not}$ possible.
\end{problem}
\begin{problem}[9103148252094553273]
	The kingdom of Anisotropy consists of $n$ cities. For every two cities there exists exactly one direct one-way road between them. We say that a path from $X$ to $Y$ is a sequence of roads such that one can move from $X$ to $Y$ along this sequence without returning to an already visited city. A collection of paths is called diverse if no road belongs to two or more paths in the collection.

Let $A$ and $B$ be two distinct cities in Anisotropy. Let $N_{AB}$ denote the maximal number of paths in a diverse collection of paths from $A$ to $B$. Similarly, let $N_{BA}$ denote the maximal number of paths in a diverse collection of paths from $B$ to $A$. Prove that the equality $N_{AB} = N_{BA}$ holds if and only if the number of roads going out from $A$ is the same as the number of roads going out from $B$.
\end{problem}
\begin{problem}[117986541208663]
Given a triangle $ABC$. $D$ is a moving point on the edge $BC$. Point $E$ and Point $F$ are on the edge $AB$ and $AC$, respectively, such that $BE=CD$ and $CF=BD$. The circumcircle of $\triangle BDE$ and $\triangle CDF$ intersects at another point $P$ other than $D$. Prove that there exists a fixed point $Q$, such that the length of $QP$ is constant.
\end{problem}
\begin{problem}[915478364939250]
Consider the convex quadrilateral $ABCD$. The point $P$ is in the interior of $ABCD$. The following ratio equalities hold:
\[\angle PAD:\angle PBA:\angle DPA=1:2:3=\angle CBP:\angle BAP:\angle BPC\]Prove that the following three lines meet in a point: the internal bisectors of angles $\angle ADP$ and $\angle PCB$ and the perpendicular bisector of segment $AB$.
\end{problem}
\begin{problem}[183354438240037]
Let $I$, $O$, $H$, and $\Omega$ be the incenter, circumcenter, orthocenter, and the circumcircle of the triangle $ABC$, respectively. Assume that line $AI$ intersects with $\Omega$ again at point $M\neq A$, line $IH$ and $BC$ meets at point $D$, and line $MD$ intersects with $\Omega$ again at point $E\neq M$. Prove that line $OI$ is tangent to the circumcircle of triangle $IHE$.
\end{problem}
\begin{problem}[651490142085731]
	Let $I$ be the incenter of triangle $ABC$, and let $\omega$ be its incircle. Let $E$ and $F$ be the points of tangency of $\omega$ with $CA$ and $AB$, respectively. Let $X$ and $Y$ be the intersections of the circumcircle of $BIC$ and $\omega$. Take a point $T$ on $BC$ such that $\angle AIT$ is a right angle. Let $G$ be the intersection of $EF$ and $BC$, and let $Z$ be the intersection of $XY$ and $AT$. Prove that $AZ$, $ZG$, and $AI$ form an isosceles triangle.
\end{problem}
\begin{problem}[6020628633767269011]
Let \(ABCDE\) be a regular pentagon. Let \(P\) be a variable point on the interior of segment \(AB\) such that \(PA\ne PB\). The circumcircles of \(\triangle PAE\) and \(\triangle PBC\) meet again at \(Q\). Let \(R\) be the circumcenter of \(\triangle DPQ\). Show that as \(P\) varies, \(R\) lies on a fixed line.
\end{problem}
\begin{problem}[8402748184217471405]
In $\Delta ABC$, $AD \perp BC$ at $D$. $E,F$ lie on line $AB$, such that $BD=BE=BF$. Let $I,J$ be the incenter and $A$-excenter. Prove that there exist two points $P,Q$ on the circumcircle of $\Delta ABC$ , such that $PB=QC$, and $\Delta PEI \sim \Delta QFJ$ .
\end{problem}
\begin{problem}[1427062131747349943]
Let $ABC$ be a triangle with circumcenter $O$ and orthocenter $H$ such that $OH$ is parallel to $BC$. Let $AH$ intersects again with the circumcircle of $ABC$ at $X$, and let $XB, XC$ intersect with $OH$ at $Y, Z$, respectively. If the projections of $Y,Z$ to $AB,AC$ are $P,Q$, respectively, show that $PQ$ bisects $BC$.
\end{problem}
\begin{problem}[302438226120877]
	Given triangle $ABC$. Let $BPCQ$ be a parallelogram ($P$ is not on $BC$). Let $U$ be the intersection of $CA$ and $BP$, $V$ be the intersection of $AB$ and $CP$, $X$ be the intersection of $CA$ and the circumcircle of triangle $ABQ$ distinct from $A$, and $Y$ be the intersection of $AB$ and the circumcircle of triangle $ACQ$ distinct from $A$.
Prove that $\overline{BU} = \overline{CV}$ if and only if the lines $AQ$, $BX$, and $CY$ are concurrent.
\end{problem}
\begin{problem}[210358073900610]
Let triangle $ABC$ have altitudes $BE$ and $CF$ which meet at $H$. The reflection of $A$ over $BC$ is $A'$. Let $(ABC)$ meet $(AA'E)$ at $P$ and $(AA'F)$ at $Q$. Let $BC$ meet $PQ$ at $R$. Prove that $EF \parallel HR$.
\end{problem}
\begin{problem}[5363953658134647103]
Let $ABC$ be a triangle with incenter $I$. The line through $I$, perpendicular to $AI$, intersects the circumcircle of $ABC$ at points $P$ and $Q$. It turns out there exists a point $T$ on the side $BC$ such that $AB + BT = AC + CT$ and $AT^2 =  AB \cdot AC$. Determine all possible values of the ratio $IP/IQ$.
\end{problem}
\begin{problem}[6209707374283278028]
Let $ABC$ be a triangle and $D$ be a point inside triangle $ABC$. $\Gamma$ is the circumcircle of triangle $ABC$, and $DB$, $DC$ meet $\Gamma$ again at $E$, $F$ , respectively. $\Gamma_1$, $\Gamma_2$ are the circumcircles of triangle $ADE$ and $ADF$ respectively. Assume $X$ is on $\Gamma_2$ such that $BX$ is tangent to $\Gamma_2$. Let $BX$ meets $\Gamma$ again at $Z$. Prove that the line $CZ$ is tangent to $\Gamma_1$ .
\end{problem}
\begin{problem}[308110166188097]
Let $A,B$ be two fixed points on the unit circle $\omega$, satisfying $\sqrt{2} < AB < 2$. Let $P$ be a point that can move on the unit circle, and it can move to anywhere on the unit circle satisfying $\triangle ABP$ is acute and $AP>AB>BP$. Let $H$ be the orthocenter of $\triangle ABP$ and $S$ be a point on the minor arc $AP$ satisfying $SH=AH$. Let $T$ be a point on the minor arc $AB$ satisfying $TB || AP$. Let $ST\cap BP = Q$.

Show that (recall $P$ varies) the circle with diameter $HQ$ passes through a fixed point.
\end{problem}
\begin{problem}[3838489129977355762]
Two triangles $ABC$ and $A'B'C'$ are on the plane. It is known that each side length of triangle $ABC$ is not less than $a$, and each side length of triangle $A'B'C'$ is not less than $a'$. Prove that we can always choose two points in the two triangles respectively such that the distance between them is not less than $\sqrt{\dfrac{a^2+a'^2}{3}}$.
\end{problem}
\begin{problem}[6566259136811987209]
	Let $\Omega$ be the $A$-excircle of triangle $ABC$, and suppose that $\Omega$ is tangent to lines $BC$, $CA$, and $AB$ at points $D$, $E$, and $F$, respectively. Let $M$ be the midpoint of segment $EF$. Two more points $P$ and $Q$ are on $\Omega$ such that $EP$ and $FQ$ are both parallel to $DM$. Let $BP$ meet $CQ$ at point $X$. Prove that the line $AM$ is the angle bisector of $\angle XAD$.
\end{problem}
\begin{problem}[8639636622304457736]
Let $\triangle ABC$ be a triangle, and let $S$ and $T$ be the midpoints of the sides $BC$ and $CA$, respectively. Suppose $M$ is the midpoint of the segment $ST$ and the circle $\omega$ through $A, M$ and $T$ meets the line $AB$ again at $N$. The tangents of $\omega$ at $M$ and $N$ meet at $P$. Prove that $P$ lies on $BC$ if and only if the triangle $ABC$ is isosceles with apex at $A$.
\end{problem}
\begin{problem}[1872712387771032593]
Let $H$ be the orthocenter of triangle $ABC$, and $AD$, $BE$, $CF$ be the three altitudes of triangle $ABC$. Let $G$ be the orthogonal projection of $D$ onto $EF$, and $DD'$ be the diameter of the circumcircle of triangle $DEF$. Line $AG$ and the circumcircle of triangle $ABC$ intersect again at point $X$. Let $Y$ be the intersection of $GD'$ and $BC$, while $Z$ be the intersection of $AD'$ and $GH$. Prove that $X$, $Y$, and $Z$ are collinear.
\end{problem}
\begin{problem}[7017112574129036660]
	Let $ABC$ be a triangle with $AB<AC$, and let $I_a$ be its $A$-excenter. Let $D$ be the projection of $I_a$ to $BC$. Let $X$ be the intersection of $AI_a$ and $BC$, and let $Y,Z$ be the points on $AC,AB$, respectively, such that $X,Y,Z$ are on a line perpendicular to $AI_a$. Let the circumcircle of $AYZ$ intersect $AI_a$ again at $U$. Suppose that the tangent of the circumcircle of $ABC$ at $A$ intersects $BC$ at $T$, and the segment $TU$ intersects the circumcircle of $ABC$ at $V$. Show that $\angle BAV=\angle DAC$.
\end{problem}
\begin{problem}[7268978143074030034]
Given two circles $\omega_1$ and $\omega_2$ where $\omega_2$ is inside $\omega_1$. Show that there exists a point $P$ such that for any line $\ell$ not passing through $P$, if $\ell$ intersects circle $\omega_1$ at $A,B$ and $\ell$ intersects circle $\omega_2$ at $C,D$, where $A,C,D,B$ lie on $\ell$ in this order, then $\angle APC=\angle BPD$.
\end{problem}
\begin{problem}[684771433215596]
In triangle $ABC$, point $A_1$ lies on side $BC$ and point $B_1$ lies on side $AC$. Let $P$ and $Q$ be points on segments $AA_1$ and $BB_1$, respectively, such that $PQ$ is parallel to $AB$. Let $P_1$ be a point on line $PB_1$, such that $B_1$ lies strictly between $P$ and $P_1$, and $\angle PP_1C=\angle BAC$. Similarly, let $Q_1$ be the point on line $QA_1$, such that $A_1$ lies strictly between $Q$ and $Q_1$, and $\angle CQ_1Q=\angle CBA$.

Prove that points $P,Q,P_1$, and $Q_1$ are concyclic.
\end{problem}
\begin{problem}[240654526717277]
Let $\Gamma$ be a circle with centre $I$, and $A B C D$ a convex quadrilateral such that each of the segments $A B, B C, C D$ and $D A$ is tangent to $\Gamma$. Let $\Omega$ be the circumcircle of the triangle $A I C$. The extension of $B A$ beyond $A$ meets $\Omega$ at $X$, and the extension of $B C$ beyond $C$ meets $\Omega$ at $Z$. The extensions of $A D$ and $C D$ beyond $D$ meet $\Omega$ at $Y$ and $T$, respectively. Prove that\[A D+D T+T X+X A=C D+D Y+Y Z+Z C.\]
\end{problem}
\begin{problem}[1790114062253914451]
	Given a triangle $ \triangle{ABC} $ and a point $ O $. $ X $ is a point on the ray $ \overrightarrow{AC} $. Let $ X' $ be a point on the ray $ \overrightarrow{BA} $ so that $ \overline{AX} = \overline{AX_{1}} $ and $ A $ lies in the segment $ \overline{BX_{1}} $. Then, on the ray $ \overrightarrow{BC} $, choose $ X_{2} $ with $ \overline{X_{1}X_{2}} \parallel \overline{OC} $.

Prove that when $ X $ moves on the ray $ \overrightarrow{AC} $, the locus of circumcenter of $ \triangle{BX_{1}X_{2}} $ is a part of a line.
\end{problem}
\begin{problem}[221552874820768]
The incircle of a scalene triangle $ABC$ touches the sides $BC, CA$, and $AB$ at points $D, E$, and $F$, respectively. Triangles $APE$ and $AQF$ are constructed outside the triangle so that\[AP =PE, AQ=QF, \angle APE=\angle ACB,\text{ and }\angle AQF =\angle ABC.\]Let $M$ be the midpoint of $BC$. Find $\angle QMP$ in terms of the angles of the triangle $ABC$.
\end{problem}
\begin{problem}[8670333331361701457]
	Let $n$ be a given positive integer. Sisyphus performs a sequence of turns on a board consisting of $n + 1$ squares in a row, numbered $0$ to $n$ from left to right. Initially, $n$ stones are put into square $0$, and the other squares are empty. At every turn, Sisyphus chooses any nonempty square, say with $k$ stones, takes one of these stones and moves it to the right by at most $k$ squares (the stone should say within the board). Sisyphus' aim is to move all $n$ stones to square $n$.
Prove that Sisyphus cannot reach the aim in less than
\[ \left \lceil \frac{n}{1} \right \rceil + \left \lceil \frac{n}{2} \right \rceil + \left \lceil \frac{n}{3} \right \rceil + \dots + \left \lceil \frac{n}{n} \right \rceil \]turns. (As usual, $\lceil x \rceil$ stands for the least integer not smaller than $x$. )
\end{problem}
\begin{problem}[1336030836839904136]
Let $ABCDE$ be a convex pentagon with $CD= DE$ and $\angle EDC \ne 2 \cdot \angle ADB$.
Suppose that a point $P$ is located in the interior of the pentagon such that $AP =AE$ and $BP= BC$.
Prove that $P$ lies on the diagonal $CE$ if and only if area $(BCD)$ + area $(ADE)$ = area $(ABD)$ + area $(ABP)$.
\end{problem}
\begin{problem}[2265193939454652363]
	A circle $\omega$ with radius $1$ is given. A collection $T$ of triangles is called good, if the following conditions hold:
each triangle from $T$ is inscribed in $\omega$;
no two triangles from $T$ have a common interior point.
Determine all positive real numbers $t$ such that, for each positive integer $n$, there exists a good collection of $n$ triangles, each of perimeter greater than $t$.
\end{problem}
\begin{problem}[2139114147569608698]
Let $O$ be the circumcenter of an acute triangle $ABC$. Line $OA$ intersects the altitudes of $ABC$ through $B$ and $C$ at $P$ and $Q$, respectively. The altitudes meet at $H$. Prove that the circumcenter of triangle $PQH$ lies on a median of triangle $ABC$.
\end{problem}
\begin{problem}[233559801569582]
Let $n$ be a positive integer. Find the number of permutations $a_1$, $a_2$, $\dots a_n$ of the
sequence $1$, $2$, $\dots$ , $n$ satisfying
$$a_1 \le 2a_2\le 3a_3 \le \dots \le na_n$$
\end{problem}
\begin{problem}[571373387028298]
Let $ABC$ be a triangle with $\angle BAC > 90 ^{\circ}$, and let $O$ be its circumcenter and $\omega$ be its circumcircle. The tangent line of $\omega$ at $A$ intersects the tangent line of $\omega$ at $B$ and $C$ respectively at point $P$ and $Q$. Let $D,E$ be the feet of the altitudes from $P,Q$ onto $BC$, respectively. $F,G$ are two points on $\overline{PQ}$ different from $A$, so that $A,F,B,E$ and $A,G,C,D$ are both concyclic. Let M be the midpoint of $\overline{DE}$. Prove that $DF,OM,EG$ are concurrent.
\end{problem}

\end{document}
