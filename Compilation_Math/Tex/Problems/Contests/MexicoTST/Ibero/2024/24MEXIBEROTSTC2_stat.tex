\begin{problem}
    [M\'exico Ibero TST 2024/C2]
    Sean $a$ y $m$ enteros positivos y sea $f(n)=an^2+n$ una funci\'on de los enteros positivos a los enteros positivos. 
    Demuestra que el conjunto de residuos que deja $f(n)$ al dividir por $m$ es completo (es decir, contiene a todos los 
    residuos de $0$ a $m-1$), si y solo si $m$ tiene la forma $p_1^{a_1}p_2^{a_k}\cdots p_k^{a_k}$ donde $p_1, p_2, \ldots,
    p_k$ son los divisores primos de $a$ y $a_1, a_2, \ldots, a_k$ son n\'umeros enteros no negativos.
    

    \label{24MEXIBEROTSTC2}
\end{problem}