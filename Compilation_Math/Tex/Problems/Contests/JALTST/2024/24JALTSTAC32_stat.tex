\begin{problem}
    [Jalisco TST 2024/Ac3.2]
    Daniel dibuja algunos grafos en una pizarra. Comienza con el grafo $G(0)$, que consiste en un solo vertice sin
    ninguna arista. Para dibujar al grafo $G(n+1)$ Daniel realiza lo siguiente:
    \begin{enumerate}
        \item Numera todos los vertices del grafo $G(n)$.
        \item Hace una copia del grafo $G(n)$, y a las copias de los vertices numerados les agrega un apostrofe.
        \item Agrega una arista que une a los vertices originales con sus copias por parejas (1 con 1', 2 con 2', $\ldots$)
    \end{enumerate}
    A continuaci\'on se muestran los primeros grafos:

    \begin{center}
       \begin{asy}
        import graph; 
        size(11.543033701901319cm); real lsf=0.5; pen dps=linewidth(0.7)+fontsize(10); defaultpen(dps); pen ds=black; real xmin=-2.01734812695763,xmax=9.525685574943688,ymin=-1.9539353617289232,ymax=4.57168653701735; 
pair A=(0.,0.), B=(1.,0.), C=(2.,0.), D=(3.,0.), F=(4.,1.), G=(3.,1.), H=(5.,0.), I=(6.,0.), J=(6.,1.), K=(5.,1.), L=(6.5,0.5), M=(6.5,1.5), O=(5.5,0.5); 
draw(B--C,linewidth(2.)); draw(D--(4.,0.),linewidth(2.)); draw(F--(4.,0.),linewidth(2.)); draw(G--F,linewidth(2.)); draw(G--D,linewidth(2.)); draw(H--I,linewidth(2.)); draw(I--J,linewidth(2.)); draw(K--J,linewidth(2.)); draw(K--H,linewidth(2.)); draw(I--L,linewidth(2.)); draw(L--M,linewidth(2.)); draw(J--M,linewidth(2.)); draw(M--(5.5,1.5),linewidth(2.)); draw((5.5,1.5)--K,linewidth(2.)); draw(H--O,linewidth(2.)); draw(O--L,linewidth(2.)); draw((5.5,1.5)--O,linewidth(2.)); label("$G(0)$",(-0.2552808865638971,0.674232725637997),SE*lsf); label("$G(1)$",(1.252929209027349,0.674232725637997),SE*lsf); label("$G(2)$",(3.253920820999893,1.674728531624268),SE*lsf); label("$G(3)$",(5.747693949353735,2.182442821229241),SE*lsf); 
dot(A,ds); dot(B,ds); dot(C,ds); dot(D,ds); dot((4.,0.),ds); dot(F,ds); dot(G,ds); dot(H,ds); dot(I,ds); dot(J,ds); dot(K,ds); dot(L,ds); dot(M,ds); dot((5.5,1.5),ds); dot(O,ds); 
clip((xmin,ymin)--(xmin,ymax)--(xmax,ymax)--(xmax,ymin)--cycle);
       \end{asy}
        
    \end{center}
    
    Para el grafo $G(n)$ con $n>0$, Daniel decide colorear todos los vertices con $n$ colores de tal forma que cada 
    vertice esta conectado por medio de aristas con al menos un vertice de cada color. Muestre que solo podria hacer 
    esto si $n=2^k$ con $k$ un entero positivo. \\
    Nota: No es necesario encontrar la estrategia para colorear los vertices.
\end{problem}