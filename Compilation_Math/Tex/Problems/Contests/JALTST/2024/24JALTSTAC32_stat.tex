\begin{problem}
    [Jalisco TST 2024/Ac3.2]
    Daniel dibuja algunos grafos en una pizarra. Comienza con el grafo $G(0)$, que consiste en un solo vertice sin
    ninguna arista. Para dibujar al grafo $G(n+1)$ Daniel realiza lo siguiente:
    \begin{enumerate}
        \item Numera todos los vertices del grafo $G(n)$.
        \item Hace una copia del grafo $G(n)$, y a las copias de los vertices numerados les agrega un apostrofe.
        \item Agrega una arista que une a los vertices originales con sus copias por parejas ($1$ con $1'$, $2$ con $2'$, $\ldots$)
    \end{enumerate}
    A continuaci\'on se muestran los primeros grafos:

    \begin{center}
        \asyinclude{24JALTSTAC32_stat.asy}
    \end{center}
    
    Para el grafo $G(n)$ con $n>0$, Daniel decide colorear todos los vertices con $n$ colores de tal forma que cada 
    vertice esta conectado por medio de aristas con al menos un vertice de cada color. Muestre que solo podria hacer 
    esto si $n=2^k$ con $k$ un entero positivo. \\
    Nota: No es necesario encontrar la estrategia para colorear los vertices.
\end{problem}