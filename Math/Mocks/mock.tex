\documentclass[12pt]{article}
\usepackage{config}
\usepackage{jampm}

\begin{document}
\begin{examen}{Nivel 11}{}
    \begin{problema}
 	Let $ABC$ be a triangle, $I$ its incenter, and $\Gamma$ its circumcircle. Let $D$ be the second point of intersection of $AI$ with $\Gamma$. The line parallel to $BC$ through $I$ intersects $AB$ and $AC$ at $P$ and $Q$, respectively. The lines $PD$ and $QD$ intersect $BC$ at $E$ and $F$, respectively. Prove that triangles $IEF$ and $ABC$ are similar.
    \end{problema}
    
    \begin{problema}
        Sea $n\geq 2$ un entero positivo. Se tienen $2n$ floreros acomodados en un circulo. ¿De cuantas formas se pueden escoger $n-1$ de ellos de tal modo que no se eligan dos adyacentes?
    \end{problema}

     \begin{problema}
  	Let $n \ge 2$ be an integer, and let $A_n$ be the set\[A_n = \{2^n  - 2^k\mid k \in \mathbb{Z},\, 0 \le k < n\}.\]Determine the largest positive integer that cannot be written as the sum of one or more (not necessarily distinct) elements of $A_n$ .
    \end{problema}
    
    
\end{examen}

\begin{examen}{Nivel 15}{}
    \begin{problema}
        	A set of five different positive integers is called virtual if the greatest common divisor of any three of its elements is greater than $1$, but the greatest common divisor of any four of its elements is equal to $1$. Prove that, in any virtual set, the product of its elements has at least $2020$ distinct positive divisors.
    \end{problema}    

    \begin{problema}
        Sea $x>1$ un número real que no es entero y $\{ x\}$ su parte decimal, $\lfloor x \rfloor$ su funcion piso. Muestra que 
        \[ \left( \frac{x+\{ x\}}{\lfloor x \rfloor}- \frac{\lfloor x \rfloor}{x+\{ x\}} \right) + \left( \frac{x+\lfloor x\rfloor}{\{ x \}}- \frac{\{ x \}}{x+\lfloor x\rfloor}\right) \geq \frac{16}{3} \]
    \end{problema}

    \begin{problema}
        Let $ABC$ be an acute triangle with orthocenter $H$. The circle through $B, H$, and $C$ intersects lines $AB$ and $AC$ at $D$ and $E$ respectively, and segment $DE$ intersects $HB$ and $HC$ at $P$ and $Q$ respectively. Two points $X$ and $Y$, both different from $A$, are located on lines $AP$ and $AQ$ respectively such that $X, H, A, B$ are concyclic and $Y, H, A, C$ are concyclic. Show that lines $XY$ and $BC$ are parallel.
    \end{problema}
\end{examen}

\begin{examen}{Nivel 19}{}
      \begin{problema}
        Sea $n\geq 2$ un entero positivo. Se tienen $2n$ floreros acomodados en un circulo. ¿De cuantas formas se pueden escoger $n-1$ de ellos de tal modo que no se eligan dos adyacentes?
    \end{problema}
    
       \begin{problema}
        	Show that $r = 2$ is the largest real number $r$ which satisfies the following condition:

If a sequence $a_1$, $a_2$, $\ldots$ of positive integers fulfills the inequalities
\[a_n \leq a_{n+2} \leq\sqrt{a_n^2+ra_{n+1}}\]for every positive integer $n$, then there exists a positive integer $M$ such that $a_{n+2} = a_n$ for every $n \geq M$.
    \end{problema}

     \begin{problema}
        Let $ABC$ be an acute triangle with orthocenter $H$. The circle through $B, H$, and $C$ intersects lines $AB$ and $AC$ at $D$ and $E$ respectively, and segment $DE$ intersects $HB$ and $HC$ at $P$ and $Q$ respectively. Two points $X$ and $Y$, both different from $A$, are located on lines $AP$ and $AQ$ respectively such that $X, H, A, B$ are concyclic and $Y, H, A, C$ are concyclic. Show that lines $XY$ and $BC$ are parallel.
    \end{problema}
\end{examen}

\begin{examen}{Si no voy a la IMO voy a chillar, Nivel 25}{}
    \begin{problema}
  	Let $n \ge 2$ be an integer, and let $A_n$ be the set\[A_n = \{2^n  - 2^k\mid k \in \mathbb{Z},\, 0 \le k < n\}.\]Determine the largest positive integer that cannot be written as the sum of one or more (not necessarily distinct) elements of $A_n$ .
    \end{problema}
    
    \begin{problema}
        	Show that $r = 2$ is the largest real number $r$ which satisfies the following condition:

If a sequence $a_1$, $a_2$, $\ldots$ of positive integers fulfills the inequalities
\[a_n \leq a_{n+2} \leq\sqrt{a_n^2+ra_{n+1}}\]for every positive integer $n$, then there exists a positive integer $M$ such that $a_{n+2} = a_n$ for every $n \geq M$.
    \end{problema}
    
    \begin{problema}
     Let $ABC$ be an acute-angled triangle with $AC > AB$, let $O$ be its circumcentre, and let $D$ be a point on the segment $BC$. The line through $D$ perpendicular to $BC$ intersects the lines $AO, AC,$ and $AB$ at $W, X,$ and $Y,$ respectively. The circumcircles of triangles $AXY$ and $ABC$ intersect again at $Z \ne A$.
Prove that if $W \ne D$ and $OW = OD,$ then $DZ$ is tangent to the circle $AXY.$
    \end{problema}
    
\end{examen}

\begin{examen}{Si no voy a la IMO voy a chillar, Nivel 26}{}
\begin{problema}
    Determina todos los enteros que pueden ser escritos de la forma 
    \[ \frac{(x+y+z)^2}{xyz}\]
    donde $x,y,z$ son enteros positivos.
\end{problema}

\begin{problema}
    
    Encuentra todos los enteros $n\geq 3$ tales que existe un polígono convexto de $n$ lados $A_1A_2\dots A_n$ que tenga las siguientes características:
 \begin{itemize} 
 \item  Todos los ángulos internos de $A_1A_2\dots A_n$ son iguales. 
 \item  No todos los lados de $A_1A_2\dots A_n$ son iguales.
 \item  Existe un triángulo $T$ y un punto $O$ en el interior de $A_1A_2\dots A_n$ tal que los $n$ triángulos $OA_1A_2, OA_2A_3,\dots OA_nA_1$ son todos semejantes a $T$.
 \end{itemize} 
    
\end{problema}

\begin{problema}
Let $n > 3$ be a positive integer. Suppose that $n$ children are arranged in a circle, and $n$ coins are distributed between them (some children may have no coins). At every step, a child with at least 2 coins may give 1 coin to each of their immediate neighbors on the right and left. Determine all initial distributions of the coins from which it is possible that, after a finite number of steps, each child has exactly one coin.
\end{problema}
\end{examen}


\begin{examen}{Nivel 6}{}
\begin{problema}
    Considere la siguiente ecuación de cuarto grado:
    \[x^4-ax^3+bx^2-cx+d=0\]
    Si se sabe que sus soluciones son todas n\'umeros primos distintos entre si, determine el m\'aximo valor que puede tomar $\frac cd$.
\end{problema}

\begin{problema}
    Sea $ABCD$ un cuadrilatero convexo, y sea $l$ una recta paralela a $AC$. La recta $l$ corta a las rectas $AD,BC,AB,CD$ en $X,Y,Z,T$ respectivamente. Los circuncirculos de $XYB$ y $ZTB$ se cortan por segunda vez en $R$. Muestra que $R$ esta sobre $BD$.
\end{problema}

\begin{problema}
    There is a row with $2024$ cells. Ana and Beto take turns playing, with Ana going first. On each turn, the player selects an empty cell and places a digit in that space. Once all $2024$ cells are filled, the number obtained from reading left to right is considered, ignoring any leading zeros. Beto wins if the resulting number is a multiple of $99$, otherwise Ana wins. Determine which of the two players has a winning strategy and describe it.
\end{problema}

\end{examen}

\begin{examen}{Si no voy a la IMO voy a chillar, Nivel 25}{}
    \begin{problema}
        Let $ABC$ be an isosceles triangle with $BC=CA$, and let $D$ be a point inside side $AB$ such that $AD< DB$. Let $P$ and $Q$ be two points inside sides $BC$ and $CA$, respectively, such that $\angle DPB = \angle DQA = 90^{\circ}$. Let the perpendicular bisector of $PQ$ meet line segment $CQ$ at $E$, and let the circumcircles of triangles $ABC$ and $CPQ$ meet again at point $F$, different from $C$.
Suppose that $P$, $E$, $F$ are collinear. Prove that $\angle ACB = 90^{\circ}$.
    \end{problema}
    
    \begin{problema}
        Let $S = \{1,2,\dots,2014\}$. For each non-empty subset $T \subseteq S$, one of its members is chosen as its representative. Find the number of ways to assign representatives to all non-empty subsets of $S$ so that if a subset $D \subseteq S$ is a disjoint union of non-empty subsets $A, B, C \subseteq S$, then the representative of $D$ is also the representative of one of $A$, $B$, $C$.
    \end{problema}

    \begin{problema}
        Call a rational number $r$ powerful if $r$ can be expressed in the form $\dfrac{p^k}{q}$ for some relatively prime positive integers $p, q$ and some integer $k >1$. Let $a, b, c$ be positive rational numbers such that $abc = 1$. Suppose there exist positive integers $x, y, z$ such that $a^x + b^y + c^z$ is an integer. Prove that $a, b, c$ are all powerful.
    \end{problema}
\end{examen}

\begin{examen}{Nivel 12}{}
    \begin{problema}
        Show that from a set of $11$ square integers one can select six numbers $a^2,b^2,c^2,d^2,e^2,f^2$ such that $a^2+b^2+c^2 \equiv d^2+e^2+f^2\pmod{12}$.
    \end{problema}

    \begin{problema}
        Sean $a,b$ y $c$ enteros positivos tales que $b\leq c$ y que cumplen 
        \[ \frac{(ab-1)(ac-1)}{bc}=2023\]
        Encuentra todos los posibles valores de $c$. 
    \end{problema}
    
    \begin{problema}
       Determine all positive integers $ n\geq 2$ that satisfy the following condition: for all $ a$ and $ b$ relatively prime to $ n$ we have\[a \equiv b \pmod n\qquad\text{if and only if}\qquad ab\equiv 1 \pmod n.\]
        
    \end{problema}
\end{examen}


\begin{examen}{Nivel 13}{}

\begin{problema}
  A sequence $x_1, x_2, \dots$ of integers satisfies $x_1 \in \{5,7\}$
  and $x_{k+1} \in \{5^{x_k}, 7^{x_k}\}$ for each $k \ge 1$.
  What are the possible remainders when $x_{2012}$ is divided by $100$?
\end{problema}

\begin{problema}
Let \(x\) and \(y\) be positive real numbers satisfying the following system of equations:
\[
\begin{cases}
\sqrt{x}\left(2 + \dfrac{5}{x+y}\right) = 3 \\\\
\sqrt{y}\left(2 - \dfrac{5}{x+y}\right) = 2
\end{cases}
\]Find the maximum value of \(x + y\).
\end{problema}

\begin{problema}
Let $ABCDE$ be a convex pentagon such that $\angle ABC = \angle AED = 90^\circ$. Suppose that the midpoint of $CD$ is the circumcenter of triangle $ABE$. Let $O$ be the circumcenter of triangle $ACD$.

Prove that line $AO$ passes through the midpoint of segment $BE$.
\end{problema}


\end{examen}

\begin{examen}{Nivel 16}{}

\begin{problema}
Dado un triangulo equilatero $ABC$. Sean $M,N,L$ los puntos medios de los lados $AB,AC$ y $BC$ respectivamente. Con centro en $A$ se traza una circunferencia que pasa por $M$. Sobre el arco $MN$ e interior al triangulo $MNL$ se toma un punto $P$ distinto de $M$ y $N$. Se trazan los segmentos $PM,PN, PL$. Probar que $PL^2=PM^2+PN2$, con independencia de donde se escoga el punto $P$ en ese arco.
\end{problema}

\begin{problema}
For each positive integer $n$, let $d(n)$ be the number of positive integer divisors of $n$.
Prove that for all pairs of positive integers $(a,b)$ we have that:
\[ d(a)+d(b) \le d(\gcd(a,b))+d(\text{lcm}(a,b)) \]and determine all pairs of positive integers $(a,b)$ where we have equality case.
\end{problema}

\begin{problema}
	Let $S$ be an infinite set of positive integers, such that there exist four pairwise distinct $a,b,c,d \in S$ with $\gcd(a,b) \neq \gcd(c,d)$. Prove that there exist three pairwise distinct $x,y,z \in S$ such that $\gcd(x,y)=\gcd(y,z) \neq \gcd(z,x)$.
\end{problema}


\end{examen}

\begin{examen}{Nivel 17}{}

\begin{problema}
Pete tiene una baraja de 1001 cartas; los n\'umeros $1,2,\ldots, 1001$ estan escritos en esas cartas con una pluma azul, un n\'umero por carta. Pete acomoda las cartas en un c\'irculo con los n\'umeros azules en la cara inferior. Luego, para cada carta $C$, Pete considera las 500 cartas que siguen de $C$ en sentido horario y cuenta el n\'umero $f(C)$ de cartas con n\'umeros azules mayores al n\'umero azul de $C$. Pete escribe el n\'umero $f(C)$ en la cara superior de $C$ con pluma roja. Demuestra que Basil, quien puede ver solo los n\'umeros rojos en las cartas, puede determinar los n\'umeros azules de todas las cartas.
\end{problema}

\begin{problema}
Let acute scalene triangle $ABC$ have orthocenter $H$ and altitude $AD$ with $D$ on side $BC$. Let $M$ be the midpoint of side $BC$, and let $D'$ be the reflection of $D$ over $M$. Let $P$ be a point on line $D'H$ such that lines $AP$ and $BC$ are parallel, and let the circumcircles of $\triangle AHP$ and $\triangle BHC$ meet again at $G \neq H$. Prove that $\angle MHG = 90^\circ$.
\end{problema}


\begin{problema}
Let $k\ge2$ be an integer. Find the smallest integer $n \ge k+1$ with the property that there exists a set of $n$ distinct real numbers such that each of its elements can be written as a sum of $k$ other distinct elements of the set.
\end{problema}


\end{examen}

\begin{examen}{Si no voy a la IMO voy a chillar, Nivel 26}{}

\begin{problema}
Let $S$ be an infinite set of positive integers, such that there exist four pairwise distinct $a,b,c,d \in S$ with $\gcd(a,b) \neq \gcd(c,d)$. Prove that there exist three pairwise distinct $x,y,z \in S$ such that $\gcd(x,y)=\gcd(y,z) \neq \gcd(z,x)$.
\end{problema}

\begin{problema}
Let $a > 1$ be a positive integer and $d > 1$ be a positive integer coprime to $a$. Let $x_1=1$, and for $k\geq 1$, define
$$x_{k+1} = \begin{cases}
x_k + d &\text{if } a \text{ does not divide } x_k \\
x_k/a & \text{if } a \text{ divides } x_k
\end{cases}$$Find, in terms of $a$ and $d$, the greatest positive integer $n$ for which there exists an index $k$ such that $x_k$ is divisible by $a^n$.	
\end{problema}

\begin{problema}
Find all positive integers $d$ for which there exists a degree $d$ polynomial $P$ with real coefficients such that there are at most $d$ different values among $P(0),P(1),P(2),\cdots,P(d^2-d)$ .
\end{problema}

\end{examen}


\end{document}
