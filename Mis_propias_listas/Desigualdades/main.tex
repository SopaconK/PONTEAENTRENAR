\documentclass[11pt]{scrartcl}

\usepackage[sexy]{evan}
\usepackage{pgfplots}
\pgfplotsset{compat=1.15}
\usepackage{mathrsfs}
\usetikzlibrary{arrows}
\usepackage{graphics}
\usepackage{tikz}
\usepackage{ amssymb }
\usepackage[dvipsnames]{xcolor}
\definecolor{red1}{RGB}{255, 153, 153}
\definecolor{green1}{RGB}{204, 255, 204}
\definecolor{blue1}{RGB}{204, 255, 255}
\definecolor{yellow1}{RGB}{255, 247, 160}

\definecolor{red2}{RGB}{255, 102, 102}
\definecolor{green2}{RGB}{108, 255, 108}
\definecolor{blue2}{RGB}{94, 204, 255}
\definecolor{yellow2}{RGB}{255, 250, 104}

\definecolor{red2.5}{RGB}{255,76,76}
\definecolor{green2.5}{RGB}{54, 247, 54}
\definecolor{blue2.5}{RGB}{51, 189, 255}
\definecolor{yellow2.5}{RGB}{255, 242, 52}


\definecolor{red3}{RGB}{255, 51, 51}
\definecolor{green3}{RGB}{0, 240, 0}
\definecolor{blue3}{RGB}{9, 175, 255}
\definecolor{yellow3}{RGB}{255, 234, 0}

\definecolor{red3.5}{RGB}{229, 25, 25}
\definecolor{green3.5}{RGB}{0, 194, 0}
\definecolor{blue3.5}{RGB}{4, 143, 209}
\definecolor{yellow3.5}{RGB}{255,220,0}

\definecolor{red4}{RGB}{204, 0, 0}
\definecolor{green4}{RGB}{0, 149, 0}
\definecolor{blue4}{RGB}{0, 111, 164}
\definecolor{yellow4}{RGB}{255, 206, 0}


\title {Introducci\'on a Desigualdades}

\author{Emmanuel Buenrostro, David L\'opez}


\begin{document}
\maketitle
\section{Teor\'ia}
Comenzaremos con una desigualdad b\'asica: Sea $x$ cualquier n\'umero real, entonces se cumple que
$$x^2 \geq 0.$$
A partir de este resultado es posible construir varios teoremas. 

\subsection{AM-GM}

\begin{theorem} [AM-GM para dos variables]
Para $a,b$ reales no negativos se tiene que 
$$\frac{a+b}{2} \geq \sqrt{ab}$$
donde la igualdad ocurre si y solo si $a=b$.
\end{theorem}
\begin{proof}
Para demostrarlo tenemos que 
\begin{eqnarray*}
a+b &\geq& 2\sqrt{ab} \\
\Leftrightarrow a-2\sqrt{ab}+b &\geq& 0 \\
\Leftrightarrow (\sqrt{a}-\sqrt{b})^2 &\geq& 0.
\end{eqnarray*}
Lo cual es cierto, (Note que la condici\'on $a,b$ no negativos es necesaria para evitar n\'umeros imaginarios en $\sqrt{a}$ y $\sqrt{b}$).
\end{proof}
\begin{example}
Sean \(a\), \(b\), \(c\), reales positivos tales que: \(a+b+c=6\) y \(abc=3\), demuestra que:
$${1 \over a(b+c)}  + {1 \over b(a+c)} + {1 \over c(a+b)} \leq 1$$
y encuentra cuándo se da la igualdad.
\label{BCTST2015-5}
\end{example}
Este ejemplo ilustra claramente d\'onde es posible utilizar AM-GM, además de presentar ideas muy \'utiles para resolver desigualdades. Para este caso, vemos que es posible multiplicar por $abc$ (este movimiento es frecuente cuando se restringe $abc=1$, aunque en este caso tambi\'en resulta \'util), y llegamos a una expresi\'on m\'as manejable.
\begin{soln}
Multiplicamos por $abc$ y se convierte en 
$$\frac{bc}{b+c}+\frac{ca}{c+a}+\frac{ab}{a+b} \leq 3.$$
Ahora es momento de aplicar AM-GM. La intuici\'on detr\'as de esto proviene de que "arriba" (numerador) tenemos la parte menor de AM-GM (la multiplicicaci\'on), y "abajo" (denominador) tenemos la parte mayor (la suma), por lo que ser\'a posible simplificar t\'erminos.\\

\noindent
Entonces, por AM-GM:
\begin{eqnarray*}
\sqrt{ab} &\leq& \frac{a+b}{2} \\
\Leftrightarrow ab &\leq& \frac{(a+b)^2}{4}.
\end{eqnarray*}
An\'alogamente para $bc$ y $ac$. Entonces, tenemos que 
\begin{eqnarray*}
\frac{bc}{b+c}+\frac{ca}{c+a}+\frac{ab}{a+b} &\leq& \frac{\frac{(b+c)^2}{4}}{b+c}+\frac{\frac{(a+c)^2}{4}}{a+c}+\frac{\frac{(a+b)^2}{4}}{a+b} \\
&=& \frac{b+c}{4}+\frac{a+c}{4}+\frac{a+b}{4} \\
&=& \frac{2(a+b+c)}{4} \\
&=& \frac{12}{4} \\
&=& 3.
\end{eqnarray*}
Concluyendo el ejemplo.
\end{soln}
\noindent
Ahora bien, es posible generalizar AM-GM para $n$ variables. 
\begin{theorem} [AM-GM]
Para $a_1,a_2, \ldots, a_n$ reales no negativos se tiene que 
$$\frac{a_1+a_2+\ldots+a_n}{n} \geq \sqrt[n]{a_1a_2\cdots a_n}$$
Con la igualdad si y solo si $a_1=a_2=\ldots=a_n$.
\end{theorem}

\noindent
Esta desigualdad nos da la oportunidad de jugar con que valores insertar en la multiplicaci\'on para que te quede lo que necesites.

\begin{example}
    \label{JBMOSL22-A3}
    Sean $a,b,c$ reales positivos con $a+b+c=1$. Demuestra lo siguiente
    \[a \sqrt[3]{\frac{b}{a}} + b \sqrt[3]{\frac{c}{b}} + c \sqrt[3]{\frac{a}{c}} \le ab + bc + ca + \frac{2}{3}.\]
\end{example}

\begin{soln}

\noindent
Reescribiendo el lado izquierdo nos queda que queremos demostrar esto:
\[ \sqrt[3]{a^2b}+\sqrt[3]{b^2c}+\sqrt[3]{c^2a} \le ab+bc+ca+\frac 23\]
Y eso nos da el indicio de que queremos jugar con como repartir un AM-GM con $\sqrt[3]{a^2b}$ para obtener algo que te convenga del lado derecho. \\
Una manera muy com\'un de repartir ese producto es la siguiente:
\[ \sqrt[3]{a^2b}=\sqrt[3]{ab \cdot a\cdot 1 } \leq \frac{ab+a+1}{3}\]
Pero al juntar las 3 raices podemos notar que no te queda lo que necesitas, ya que el $(ab+bc+ca)$ termina dividido entre 3. Entonces suena \'optimo poner $3ab$ en lugar de $ab$ en el AM-GM, pero ese 3 lo tenemos que quitar de algun lado, en este caso vamos a cambiar el 1 por $\frac 13$, quedando:
\[
\sqrt[3]{a^2b}=\sqrt[3]{3ab\cdot a \cdot \frac 13}\leq \frac{3ab+a+\frac13}{3}
\]
Entonces cuando sumas las 3 raices cubicas te queda 
\begin{eqnarray*}
     \sqrt[3]{a^2b}+\sqrt[3]{b^2c}+\sqrt[3]{c^2a} &\le& \frac{3ab+3bc+3ca+a+b+c+1}{3} \\
     &=& ab+bc+ca+\frac{a+b+c+1}{3} \\
     &=& ab+bc+ca+\frac 23
\end{eqnarray*}


\end{soln}

\vspace{0.5cm}

\subsection{QM-AM-GM-HM}

\noindent
Existen al menos 4 medias que son fundamentales para la resoluci\'on de diversos problemas. Se resumen en el siguiente teorema:

\begin{theorem} [QM-AM-GM-HM]
Sean $a_1,a_2, \ldots, a_n$ reales positivos, entonces:
$$\sqrt{\frac{a_1^2+a_2^2+\ldots+a_n^2}{n}} \geq \frac{a_1+a_2+\ldots+a_n}{n} \geq \sqrt[n]{a_1a_2\cdots a_n} \geq \frac{n}{\frac{1}{a_1}+\frac{1}{a_2}+\ldots+\frac{1}{a_n}}$$
Con la igualdad si y solo si $a_1=a_2=\ldots=a_n$.
\end{theorem}

\noindent
Estas medias se conocen como cu\'adratica, aritm\'etica, geom\'etrica y arm\'onica, respectivamente. Note que al tenerlas todas ordenas podemos utilizarlas por parejas gracias a la transitividad de las desigualdades. Veamos un ejemplo de c\'omo utilizar QM-AM para atacar un problema.\\
\begin{example}
\label{JALTST2023-1.3}
Se tienen 27 puntos con coordenadas y distancias entre cualesquiera dos de ellos enteras. Demuestra que al menos 27 de esas distancias son divisibles entre 3. 
\end{example}
Si bien \'este problema podr\'ia pensarse en t\'erminos de combinatoria por su parecido a problemas c\'asicos del Principio de Casillas, existe una solución m\'as directa utilizando QM-AM. 
\begin{soln}
Vamos a iniciar con la idea que se suele usar en estos problemas con casillas. Estudiamos los posibles valores mod 3 de cada coordenada. Esto nos deja 9 posibles tipos de puntos $(0,0),(0,1),(0,2),(1,0),(1,1), (1,2),(2,0),(2,1),(2,2)$. \\

\noindent
Ahora, la distancia entre dos puntos $(x_1,y_1), (x_2,y_2)$ es $\sqrt{(x_1-x_2)^2+(y_1-y_2)^2}$. Estamos interesados en que eso sea m\'ultiplo de 3, entonces, como $1^2 \equiv 2^2 \equiv 1$ mod 3, vemos que si $x_1 \not \equiv x_2$ mod 3 y/o $y_1 \not \equiv y_2$ mod 3, la distancia ser\'ia 1 o 2 mod 3 (tendr\'iamos $ 0+1,1+0,$ o $1+1$ porque al menos una diferencia no es 0).  \\
Entonces queremos que $x_1\equiv x_2 $ mod 3 y $y_1\equiv y_2$ mod 3, por lo que son dos puntos del mismo tipo. \\

\noindent
Entonces si definimos que para cada tipo de punto tenemos $a_i$ puntos, la cantidad de distancias que cumplen la condici\'on tomando puntos de ese tipo es
$$\sum_{i=1}^9 \binom{a_i}{2}$$
esto es, las formas de tomar cualesquiera dos puntos de ese tipo. Aqu\'i es donde a terminado la parte de combi del problema y es donde empezamos con QM-AM.\\

\noindent
Para esto vamos a expandir. 
$$\sum_{i=1}^9 \binom{a_i}{2}= \sum_{i=1}^9 \frac{a_i^2-a_i}{2}.$$ 
Y podemos separar en dos sumas. 
$$ \sum_{i=1}^9 \frac{a_i^2-a_i}{2}= \frac{\sum_{i=1}^9 a_i^2- \sum_{i=1}^9 a_i }{2}.$$ 
Note que ya conocemos el valor de la suma de los $a_i$ que es la cantidad de puntos, 27.  \\
Entonces ahora queremos probar que 
$$\frac{\sum_{i=1}^9 a_i^2- 27}{2} \geq 27 \Leftrightarrow \sum_{i=1}^9 a_i^2 \geq 81.$$
Entonces solo basta probar que una desigualdad con la suma de los cuadrados es mayor o igual a una constante. Como ya conocemos el valor de la suma, podemos usar QM-AM. \\

\noindent
Entonces por QM-AM 
\begin{eqnarray*}
 \sqrt{\frac{\sum_{i=1}^9 a_i^2}{9}} &\geq& \frac{\sum_{i=1}^9 a_i}{9}=\frac{27}{9}=3 \\
\Leftrightarrow \sum_{i=1}^9 a_i^2 &\geq& 3^2\cdot 9 =81.
\end{eqnarray*}
Concluyendo el ejemplo.
\end{soln}

\subsection{Reacomodo}
\noindent
En esta secci\'on estudiaremos otro tipo de desigualdad que resulta muy pr\'actica en la Olimpiada. Consideremos que tenemos dos conjuntos ordenados $A$, $B$ con tres n\'umeros de la siguiente forma:
\begin{eqnarray*}
    a_3 \geq a_2 \geq a_1 \in A\\
    b_3 \geq b_2 \geq b_1 \in B.
\end{eqnarray*}
Supongamos que debemos emparejar cada elemento del conjunto $A$ con uno del conjunto $B$. Luego, para cada pareja podemos calcular el producto de sus elementos; y para cada arreglo de parejas podemos obtener la suma $S$ los productos por parejas. 
\begin{eqnarray*}
    P_6 = \left\{\left(a_1, b_1\right), \left(a_2, b_2\right), \left(a_3, b_3\right)\right\} \implies S_6 = a_1b_1 + a_2b_2 + a_3b_3\\
    P_5 = \left\{\left(a_1, b_1\right), \left(a_2, b_3\right), \left(a_3, b_2\right)\right\} \implies S_5 = a_1b_1 + a_2b_3 + a_3b_2\\
    P_4 = \left\{\left(a_1, b_2\right), \left(a_2, b_1\right), \left(a_3, b_3\right)\right\} \implies S_4 = a_1b_2 + a_2b_1 + a_3b_3\\
    P_3 = \left\{\left(a_1, b_2\right), \left(a_2, b_3\right), \left(a_3, b_1\right)\right\} \implies S_3 = a_1b_2 + a_2b_3 + a_3b_1\\
    P_2 = \left\{\left(a_1, b_3\right), \left(a_2, b_1\right), \left(a_3, b_2\right)\right\} \implies S_2 = a_1b_3 + a_2b_1 + a_3b_2\\
    P_1 = \left\{\left(a_1, b_3\right), \left(a_2, b_2\right), \left(a_3, b_1\right)\right\} \implies S_1 = a_1b_3 + a_2b_2 + a_3b_1\\
\end{eqnarray*}
Una pregunta interesante ser\'ia que podemos decir de estas sumas sabiendo que los elementos en los conjuntos est\'an ordenados. Resulta que es posible mostrar que:
\begin{equation*}
    S_6 \geq S_k \geq S_1 \text{ con } 6 \geq k \geq 1.
\end{equation*}
Es posible demostrar el lado izquierdo considerando alguno de los $S_k$ con $k \neq 6$ y aplicando de forma an\'aloga para los otros casos. Veamos el caso $k = 5$
\begin{eqnarray*}
    S_6 \geq S_5 \iff a_1b_1 + a_2b_2 + a_3b_3 &\geq& a_1b_3 + a_2b_2 + a_3b_1\\
    \iff a_1b_1 + a_3b_3 &\geq& a_1b_3 + a_3b_1\\
    \iff a_1\left(b_1 - b_3\right) + a_3\left(b_3 - b_1\right) &\geq& 0\\
    \iff a_3\left(b_3 - b_1\right) - a_1\left(b_3 - b_1\right) &\geq& 0\\
    \iff \left(a_3 - a_1\right)\left(b_3 - b_1\right) &\geq& 0 \text{, como } a_3 \geq a_1, b_3 \geq b_1 \text{, se cumple.}
\end{eqnarray*}
Aplicando para el resto algo similar, es posible convencerse de que la desigualdad se sostiente. Lo aprendido puede generalizarse para cualquier par de conjuntos ordenados con $n$ elementos.

\begin{theorem} [Reacomodo]
Sean $a_1 \leq a_2 \leq \dots \leq a_n$ y $b_1 \leq b_2 \leq \dots \leq b_n$ dos sucesiones de n\'umeros reales. Si $a'_1, a'_2, \dots, a'_n$ es cualquier permutuaci\'on de $a_1, a_2, \dots, a_n$, entonces
$$\sum_{j=1}^{n} a_jb_{n+1-j} \leq \sum_{j=1}^{n} a'_jb_{j} \leq \sum_{j=1}^{n} a_jb_{j}.$$
\end{theorem}

\noindent
La desdigualdad del reacomodo puede utilizarse para resolver una infinidad de problemas y demostrar muchas otras desigualdades incluyendo la ya explorada QM-AM-GM-HM y la desigualdad de Cauchy que se revisar\'a m\'as adelante. A manera de ejercicio, la utilizaremos para mostrar la desigualdad de Tchebyshev:

\begin{theorem} [Tchebyshev]
Sean $a_1 \leq a_2 \leq \dots \leq a_n$ y $b_1 \leq b_2 \leq \dots \leq b_n$ dos sucesiones de n\'umeros reales. Entonces,
$$\frac{a_1b_1+a_2b_2 + \dots + a_nb_n}{n} \geq \left(\frac{a_1+a_2+\dots+a_n}{n}\right)\left(\frac{b_1+b_2+\dots+b_n}{n}\right)$$
\end{theorem}

\begin{proof}
Para demostrarlo tenemos que aplicar varias la desigualdad del reacomodo: 
\begin{eqnarray*}
    a_1b_1 + a_2b_2 + \dots a_nb_n &=& a_1b_1 + a_2b_2 + \dots a_nb_n \\
    a_1b_1 + a_2b_2 + \dots a_nb_n &\geq& a_1b_2 + a_2b_3 + \dots a_nb_1 \\
    a_1b_1 + a_2b_2 + \dots a_nb_n &\geq& a_1b_3 + a_2b_4 + \dots a_nb_2 \\
    &\vdots& \\
    a_1b_1 + a_2b_2 + \dots a_nb_n &\geq& a_1b_n + a_2b_1 + \dots a_nb_{n-1}
\end{eqnarray*}
Sumando estas desiguladades:
\begin{equation*}
    n\left(a_1b_1 + a_2b_2 + \dots a_nb_n\right) = \left(a_1 + a_2 + \dots a_n\right)\left(b_1 + b_2 + \dots b_n\right)
\end{equation*}
Y dividiendo ambos lados por $n^2$:
\begin{equation*}
    \frac{a_1b_1 + a_2b_2 + \dots a_nb_n}{n} = \left(\frac{a_1 + a_2 + \dots a_n}{n}\right)\left(\frac{b_1 + b_2 + \dots b_n}{n}\right)
\end{equation*}
\end{proof}

\begin{example}
\label{IMO1995-P2}
Sean $a, b, c$ n\'umeros reales positivos tales que $abc = 1$. Demuestra que
\begin{equation*}
    \frac{1}{a^3\left(b+c\right)} + \frac{1}{b^3\left(a+c\right)} + \frac{1}{c^3\left(a+b\right)} \geq \frac{3}{2}
\end{equation*}
\end{example}
\begin{soln}
Como el lado izquierdo de la desigualdad es sim\'etrico, podemos asumir, sin p\'erdida de la generalidad, $a \geq b \geq c$. Sean $x = 1/a, y = 1/b, z = 1/c$ (esto solo para reducir trabajo de escritura). Note adem\'as que como $abc = 1 \implies xyz = 1$. Reescribiendo la ecuaci\'on:
\begin{eqnarray*}
    \frac{1}{a^3\left(b+c\right)} + \frac{1}{b^3\left(a+c\right)} + \frac{1}{c^3\left(a+b\right)} &=& \frac{x^3}{\frac{1}{y} + \frac{1}{z}} + \frac{y^3}{\frac{1}{x} + \frac{1}{z}} + \frac{z^3}{\frac{1}{x} + \frac{1}{y}} \\
    &=& \frac{x^2}{y+z} + \frac{y^2}{x+z} + \frac{z^2}{x+y}
\end{eqnarray*}
Ahora, c\'omo $c \leq b \leq a$, se sigue que $x \leq y \leq z$, y esto implica que $\frac{x}{y+z} \leq \frac{y}{x+z} \leq \frac{z}{x+y}$ (esto se puede demostrar estableciendo las desigualdades por parejas y reduciendo al producto de n\'umeros positivos es mayor que cero). Aplicando la desiguldad del reacomodo con las \'ultimas dos desigualdades dos veces:
\begin{eqnarray*}
    \frac{x^2}{y+z} + \frac{y^2}{x+z} + \frac{z^2}{x+y} &\geq& \frac{xy}{y+z} + \frac{yz}{x+z} + \frac{xz}{y+z}\\
    \frac{x^2}{y+z} + \frac{y^2}{x+z} + \frac{z^2}{x+y} &\geq& \frac{xz}{y+z} + \frac{xy}{x+z} + \frac{yz}{y+z}
\end{eqnarray*}
Sumando ambas desigualdades:
\begin{eqnarray*}
    2\left(\frac{1}{a^3\left(b+c\right)} + \frac{1}{b^3\left(a+c\right)} + \frac{1}{c^3\left(a+b\right)}\right) = 2\left(\frac{x^2}{y+z} + \frac{y^2}{x+z} + \frac{z^2}{x+y}\right) \\
    \geq \frac{xy}{y+z} + \frac{yz}{x+z} + \frac{xz}{y+z} + \frac{xz}{y+z} + \frac{xy}{x+z} + \frac{yz}{y+z}\\
    \geq \frac{x\left(y+z\right)}{y+z} + \frac{y\left(x+z\right)}{x+z} + \frac{z\left(x+y\right)}{x+y} = x + y + z\\
    \geq 3\sqrt[3]{xyz} = 3 \text{ (Note el uso de AM-GM)}
\end{eqnarray*}
Dividiendo entre dos ambos lados conclu\'imos la demostraci\'on.
\end{soln}

\subsection{Cauchy}
\begin{theorem} [Desigualdad de Cauchy]
    Para cualquiera n\'umeros reales $x_1,x_2,\ldots, x_n, y_1,\ldots,y_n$ la siguiente desigualdad se cumple
    \[ \left( \sum_{i=1}^n x_iy_i \right)^2 \leq \left(\sum_{i=1}^n x_i^2 \right)\left(\sum_{i=1}^n y_i^2\right)\]
    Y la igualdad se da si y solo si existe una $\lambda \in \RR$ con $x_i=\lambda y_i$ para todo $i$.  
\end{theorem}
\begin{proof}
Si $x_1=x_2=\ldots=x_n=0$ o $y_1=y_2=\ldots=y_n=0$ ambos lados son 0 y se cumple.\\
Ahora, sean $S=\sqrt{\sum_{i=1}^n x_i^2}, T=\sqrt{\sum_{i=1}^n y_i^2}$ y sean $a_i=\frac{x_i}{S}$ para $i$ entre 1 y $n$, $a_i=\frac{y_{n-i}}{T}$ para $i$ entre $n+1$ y $2n$. \\
Por reacomodo tenemos que 
\begin{eqnarray*}
\sum_{i=1}^n a_ia_{n+i} + \sum_{i=1}^n a_{n+i}a_i &\leq& \sum_{i=1}^n a_i^2 \\
&=& \sum_{i=1}^n \left(\frac{x_i^2}{S^2}+\frac{y_i^2}{T^2}\right) \\
&=& 2
\end{eqnarray*}
Por lo tanto
\begin{eqnarray*}
    2 &\geq& \sum_{i=1}^n a_ia_{n+i} + \sum_{i=1}^n a_{n+i}a_i\\
    &=& 2\frac{\sum_{i=1}^n x_iy_i}{ST}
\end{eqnarray*}
Entonces 
\[ ST \geq \sum_{i=1}^n x_iy_i\]
y elevar al cuadrado obtenemos el resultado buscado.
\end{proof}

\noindent
Tambi\'en existe otra versi\'on de esta desigualdad, la cu\'al es conocida popularmente como "\'Util", o como Cauchy en forma de Engel o Lema de Titu. 
\begin{theorem} [\'Util]
    Para cualquiera n\'umeros reales $a_1,a_2,\ldots, a_n, b_1,\ldots,b_n$ con $b_i \geq 0$ para todo $i$. Se tiene que 
    \[\sum_{i=1}^n \frac{a_i^2}{b_i} \geq \frac{(a_1+a_2+\ldots+a_n)^2}{b_1+b_2+\ldots+b_n}\]
    Con igualdad si y solo si
    \[\frac{a_1}{b_1}=\frac{a_2}{b_2}=\ldots=\frac{a_n}{b_n}\]
\end{theorem}
\begin{proof}
Por Cauchy con $x_i=\frac{a_i}{\sqrt{b_i}}$ (note que ah\'i viene la necesidad de $b_i>0$), $y_i=\sqrt{b_i}$. Obtenemos:
\[\left(\sum_{i=1}^n \frac{a_i^2}{b_i}\right)\left(\sum_{i=1}^n b_i\right) \geq \left(\sum_{i=1}^n a_i\right)^2\]
Y al dividir entre la suma de los $b_i$ obtenemos lo que est\'abamos buscando.
\end{proof}
\begin{example}
    Sean $a,b,c,d$ reales positivos con $(a+c)(b+d)=1$, demuestra que
    \[\frac{a^3}{b+c+d}+\frac{b^3}{c+d+a}+\frac{c^3}{d+a+b}+\frac{d^3}{a+b+c}\geq \frac 13\]
    \label{IMOSL90-24}
\end{example}
\begin{soln}
    Primero el primer factor lo multiplicamos por $a$, el segundo por $b$, y as\'i para tener cuadrados en los n\'umeradores. Y usando Titu
    \[\frac{a^4}{ab+ac+ad}+\frac{b^4}{bc+bd+ba}+\frac{c^4}{cd+ca+cb}+\frac{d^4}{da+db+dc}\]\[\geq \frac{(a^2+b^2+c^2+d^2)^2}{ab+ac+ad+ba+bc+bd+ca+cb+cd+da+db+dc}\]
    Por reacomodo 
    \[ab+ac+ad+ba+bc+bd+ca+cb+cd+da+db+dc\leq 3(a^2+b^2+c^2+d^2)\]
    Entonces 
    \[\frac{(a^2+b^2+c^2+d^2)^2}{ab+ac+ad+ba+bc+bd+ca+cb+cd+da+db+dc} \geq \frac{a^2+b^2+c^2+d^2}{3}\]
    Y por QM-AM-GM
    \begin{eqnarray*}
        a^2+c^2+b^2+d^2 &\geq& \frac{(a+c)^2+(b+d)^2}{2} \\
        &\geq& (a+c)(b+d)\\
        &=& 1
    \end{eqnarray*}
    entonces 
    \[\frac{a^2+b^2+c^2+d^2}{3}\geq \frac 13\]
    Terminando el problema.
\end{soln}
\newpage
\section{Problemas}
\subsection{Introductorios}
\begin{exercise}
    Muestra que para cualquier real positivo $x$ se cumple que $x+\frac 1x \geq 2$.
    \label{INEQOMMBOOK-EJ1.19}
\end{exercise}
\begin{exercise}
    Demuestra que para cualesquiera reales positivos $x,y$ se tiene que $\frac xy + \frac yx \geq 2$
    \label{INEQOMMBOOK-EJ1.24}
\end{exercise}
\begin{exercise}
    Si $x,y$ son reales, demuestra que $x^2+y^2+xy \geq 0$
\end{exercise}
\begin{exercise}
    Para $a,b,c$ reales positivos demuestra que 
    \[\frac{1}{a+2b}+\frac{1}{b+2c}+\frac{1}{c+2a} \geq \frac{3}{a+b+c}\]
\end{exercise}
\begin{exercise}
    Para reales positivos $a_1,a_2,\ldots,a_n$ demuestra que 
    \[(a_1+a_2+\ldots+a_n)\left(\frac{1}{a_1}+\frac{1}{a_2}+\ldots+\frac{1}{a_n}\right)\geq n^2\]
    \label{INEQOMMBOOK-EJ1.36}
\end{exercise}
\begin{exercise}
    Demuestra que para reales positivos $a,b,c$ se tiene que $a^2+b^2+c^2 \geq ab+bc+ca$ 
\end{exercise}
\begin{exercise}
    Demuestra que para $a,b$ reales positivos se tiene que 
    $a^5+b^5 \geq a^3b^2+b^3a^2$
    \label{MHDDEB}
\end{exercise}
\begin{exercise}
Muestra que para cualquier permutaci\'on $\left(a'_1, a'_2, \dots, a'_n\right)$ de $\left(a_1, a_2, \dots, a_n\right)$, se tiene que:
\begin{equation*}
    a^2_1 + a^2_2 + \dots + a^2_n \geq a_1a'_1 + a_2a'_2 + \dots + a_na'_n.
\end{equation*}
\label{INEQOMMBOOK-COR1.4.1}
\end{exercise}
\begin{exercise}
Probar que si para reales positivos $a,b,c$ con $a^2+b^2+c^2=3$ entonces
\[ \frac{1}{1+ab}+\frac{1}{1+bc}+\frac{1}{1+ca} \geq \frac32\]
\label{MEXTST24-8-3}
\end{exercise}

\begin{exercise}
Sean $a, b, c$ n\'umeros reales positivos. Demostrar que
\begin{equation*}
    \frac{a}{b+c} + \frac{b}{a+c} + \frac{c}{a+b} \geq \frac{3}{2}
\end{equation*}
\label{NESBITT}
\end{exercise}


\subsection{No tan introductorios}
\begin{problem}
    Para reales positivos $a,b,c$ con suma 1. Demuestra que 
    \[\frac{b^2+c^2}{1+a}+\frac{c^2+a^2}{1+b}+\frac{a^2+b^2}{1+c}\geq \half\]
\end{problem}
\begin{problem}
\label{IMO1978-P5}
Sean $x_1, x_2, \dots, x_n$ enteros postivos distintos. Muestre que:
\begin{equation*}
    \frac{x_1}{1^2} + \frac{x_2}{2^2} + \dots \frac{x_n}{n^2} \geq \frac{1}{1} + \frac{1}{2} + \dots + \frac{1}{n}
\end{equation*}
\end{problem}

\begin{problem}
\label{OMCC2020-5}
Sea $P(x)$ un polinomio con coeficientes reales no negativos. Sea $k$ un entero positivo y sean $x_1,x_2,\ldots,x_k$ números reales tales que $x_1x_2\cdots x_k=1$. Demuestre que
$$P(x_1)+P(x_2)+\ldots+P(x_k)\geq kP(1)$$

\end{problem}
\begin{problem}
Sean reales $a,b,c>0$. Probar que 
\[ \left(a^2+b^2+c^2\right)\left(\frac1a+\frac1b+\frac1c\right) \geq 3(a+b+c)\]
\end{problem}
\begin{problem}
\label{IMOLL1967B2}
Prueba que 
$$\frac 1 3 n^2 +\half n +\frac 1 6 \geq (n!)^{\frac 2 n} $$
para $n$ un entero positivo, y encuentra el caso donde se da la igualdad.
\end{problem}
\begin{problem}
    
    Sean $a$, $b$ y $c$ números reales positivos tales que $a + b + c = 3$. Muestra que 
\[ \frac{a^2}{a + \sqrt[3]{bc}} + \frac{b^2}{b + \sqrt[3]{ca}} + \frac{c^2}{c + \sqrt[3]{ab}} \geq \frac{3}{2} \]
y determina para qué números $a$, $b$ y $c$ se alcanza la igualdad.

    \label{OMM14-3}
\end{problem}
\begin{problem}
    
    Sea $n \geq 3$ un número entero y $a_1,a_2,...,a_n$ números reales positivos tales que $m$ es el menor y $M$ el mayor de estos números. Se sabe que para cualesquiera enteros distintos $1 \leq i,j,k \leq n$, si $a_i \leq a_j \leq a_k$ entonces $a_ia_k \leq a_j^2$. Demostrar que
\[ a_1a_2 \cdots a_n \geq m^2M^{n-2} \]
y determinar cuando la igualdad se mantiene.

    \label{OMCC 2021/5}
\end{problem}
\begin{problem}
    Demuestra que para reales positivos $a,b,c$ con $abc=1$ se tiene que
    \[ \frac{ab}{ab + a^5 + b^5} + \frac{bc}{bc + b^5 + c^5} + \frac{ca}{ca + c^5 + a^5} \leq 1. \]
    \label{IMOSL96A1}
\end{problem}
\begin{problem}
    
    Sean $a, b$ y $c$ números reales positivos tales que $a b+b c+c a=1$. Demostrar que
\[
\frac{a^{3}}{a^{2}+3 b^{2}+3 a b+2 b c}+\frac{b^{3}}{b^{2}+3 c^{2}+3 b c+2 c a}+\frac{c^{3}}{c^{2}+3 a^{2}+3 c a+2 a b}>\frac{1}{6\left(a^{2}+b^{2}+c^{2}\right)^{2}} .\]
    \label{OMCC23-3}
\end{problem}
\begin{problem}
    Sean $a, b, c$ n\'umeros reales positivos. Muestre que
    \begin{equation*}
        \left(1 + \frac{a}{b}\right)\left(1 + \frac{b}{c}\right)\left(1 + \frac{c}{a}\right) \geq 2\left(1+\frac{a+b+c}{\sqrt[3]{abc}}\right).
    \end{equation*}
\label{APMO1998-P3}
\end{problem}

\section{Fuentes}
%\textcolor{white}{
\begin{itemize}
    \item \ref{BCTST2015-5} BC TST 2015/5 
    \item \ref{JBMOSL22-A3} JBMO SL 2022/A3
    \item \ref{JALTST2023-1.3} Jalisco TST 2023/Acm 1.3 
    \item \ref{IMO1995-P2} IMO 1995/2
    \item \ref{IMOSL90-24} IMO SL 1990/24
    \item \ref{INEQOMMBOOK-EJ1.19} Ejercicio 1.19. Libro de desigualdades de la OMM
    \item \ref{INEQOMMBOOK-EJ1.24} Ejercicio 1.24. Libro de desigualdades de la OMM
    \item \ref{MHDDEB} Caso particular Muirhead
    \item \ref{INEQOMMBOOK-EJ1.19} Ejercicio 1.36. Libro de desigualdades de la OMM
    \item \ref{INEQOMMBOOK-COR1.4.1} Corolario 1.4.1. Libro de desigualdades de la OMM
    \item \ref{MEXTST24-8-3} MEX TST 2024 8.3
    \item \ref{NESBITT} Desigualdad de Nesbitt
    \item \ref{IMO1978-P5} IMO 1978/P5 Refraseado
    \item \ref{OMCC2020-5} OMCC 2020/5 
    \item \ref{IMOLL1967B2} IMO LL 1967/Bulgaria 2 
    \item \ref{OMM14-3} OMM 2014/3
    \item \ref{OMCC 2021/5} OMCC 2021/5
    \item \ref{IMOSL96A1} IMO SL 1996/A1
    \item \ref{OMCC23-3} OMCC 2023/3
    \item \ref{APMO1998-P3} APMO 1998/3
\end{itemize}    



\end{document}