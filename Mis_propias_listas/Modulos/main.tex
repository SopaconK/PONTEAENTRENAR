\documentclass[11pt]{scrartcl}

\usepackage[sexy]{evan}
\usepackage{pgfplots}
\pgfplotsset{compat=1.15}
\usepackage{mathrsfs}
\usetikzlibrary{arrows}
\usepackage{graphics}
\usepackage{tikz}
\usepackage{ amssymb }
\usepackage[dvipsnames]{xcolor}
\definecolor{red1}{RGB}{255, 153, 153}
\definecolor{green1}{RGB}{204, 255, 204}
\definecolor{blue1}{RGB}{204, 255, 255}
\definecolor{yellow1}{RGB}{255, 247, 160}

\definecolor{red2}{RGB}{255, 102, 102}
\definecolor{green2}{RGB}{108, 255, 108}
\definecolor{blue2}{RGB}{94, 204, 255}
\definecolor{yellow2}{RGB}{255, 250, 104}

\definecolor{red2.5}{RGB}{255,76,76}
\definecolor{green2.5}{RGB}{54, 247, 54}
\definecolor{blue2.5}{RGB}{51, 189, 255}
\definecolor{yellow2.5}{RGB}{255, 242, 52}


\definecolor{red3}{RGB}{255, 51, 51}
\definecolor{green3}{RGB}{0, 240, 0}
\definecolor{blue3}{RGB}{9, 175, 255}
\definecolor{yellow3}{RGB}{255, 234, 0}

\definecolor{red3.5}{RGB}{229, 25, 25}
\definecolor{green3.5}{RGB}{0, 194, 0}
\definecolor{blue3.5}{RGB}{4, 143, 209}
\definecolor{yellow3.5}{RGB}{255,220,0}

\definecolor{red4}{RGB}{204, 0, 0}
\definecolor{green4}{RGB}{0, 149, 0}
\definecolor{blue4}{RGB}{0, 111, 164}
\definecolor{yellow4}{RGB}{255, 206, 0}


\title{Modulos}
\subtitle{Entrenamientos Nacionales Diciembre 2022}
\author{Emmanuel Buenrostro}
\date{December 2022}

\begin{document}

\maketitle.


\section{Principios}
\textbf{Definición:} \\
\begin{center}
$a \equiv b$ mod $n$ si $n|a-b$ 
\end{center}
\textbf{ Propiedades:}
Si $a\equiv b$ mod $n$ y $c \equiv d$ mod $n$:
\begin{enumerate}
\item $a+c\equiv b+d$ mod $n$
\item $a-c \equiv b-d$ mod $n$
\item $ac \equiv bd$ mod $n$
\item $a^x\equiv b^x$ mod $n$
\item Se puede dividir? $\frac a c \equiv \frac b d$ mod $n$ es cierto si  $c | a, d|b$ y que $(c,n)=1 \Rightarrow (d,n)=1$.
\item  En caso de que no sean primos relativos 
    $$10 \equiv 6 \text{ mod } 4$$
    Dividimos entre 2
    $$5 \equiv 3 \text{ mod } 2$$
\end{enumerate}
\textbf{Cuando $a^c\equiv b^d$ mod $n$:} \\
\begin{center}
$a^c\equiv b^d$ mod $n$ funciona si $c\equiv d$ mod $\phi (n)$
\end{center}
\section{Pequeño Teorema Fermat}


\begin{theorem}[Pequeño Teorema de Fermat].
 Si $p$ es primo y $a\in \mathbb{Z}$ entonces 

$$a^p \equiv a \text{ mod } p$$
\end{theorem}
Tambien generalmente conocido como 

\begin{theorem}[Pequeño Teorema de Fermat].
 Si $p$ es primo y $a\in \mathbb{Z}$ y $(a,p)=1$ entonces 
$$a^{p-1} \equiv 1 \text{ mod } p$$
\end{theorem}

\underline{\textbf{Prueba por inducción.}} \\
Notemos que $0^p\equiv 0$ mod $p$. y $1^p\equiv 1$ mod $p$. Son nuestros casos base, ahora supongamos que para algun $a$ se tiene que 
$$a^p\equiv a \text{ mod } p$$
entonces tenemos que por el binomio de newton
$$(a+1)^p=a^p+pa^{p-1}+\binom{p}{2}a^{p-2}+\ldots+\binom{p}{p-1}a+1$$
Pero $\binom{p}{k}=\frac{p!}{k!(p-k)!}$ asi que para $1\leq k \leq p-1$ se tiene que $p | \binom{p}{k}$, porque como $k<p$ entonces no hay ningun factor $p$ en $k!$ y como $p-k\leq p-1<p$ entonces tampoco tiene un factor $p$ y no hay ningun factor $p$ en el denominador pero si en el numerador porque es $p!$.

Entonces $\binom{p}{k}a^{p-k}\equiv 0$ mod $p$ para $k\geq 1$ (porque es multiplo de $p$) entonces queda que 
$$(a+1)^p\equiv a^p+1 \equiv a+1 \text{ mod } p$$
probando lo que queremos mediante inducción.


\underline{\textbf{Prueba mas de números.}} \\

Notemos que para $a$ tal que $p|a$ si cumple porque $a^p\equiv 0 \equiv a$ mod $p$.

Para los $a$ con $(p,a)=1$. se tiene que 
$$\{a,2a,3a, \ldots, (p-1)a\}$$
es una permutación de 
$$\{1,2,3,\ldots p-1\}$$
en mod $p$. 

\underline{\textbf{Prueba.}} Todos los números $a,2a,3a,\ldots, (p-1)a$ tiene distintos modulos $p$, ya que si hay dos iguales $ia$ y $ja$ con $i\neq j$ entonces 
$$ia-ja= a(i-j) \equiv 0 \text{ mod } p$$
y como $(a,p)=1$ entonces $i-j \equiv 0$ mod $p \Rightarrow i\equiv j$ mod $p$, pero $1\leq i,j \leq p-1$ entonces $i=j$, una contradicción.
Entonces si es una permutación.  \\

Asi que si multiplicamos todos se tiene que 

$$a\cdot 2a\cdot 3a \cdots (p-1)a \equiv 1\cdot 2 \cdots (p-1) \text{ mod } p$$
entonces
$$ \Rightarrow(p-1)!a^{p-1}\equiv (p-1)! \text{ mod } p $$
$$ \Rightarrow a^{p-1}\equiv 1 \text{ mod }p$$
Probando el teorema.

\section{ Teorema de Euler}
\begin{theorem}[Teorema de Euler]
 Si $a$ es primo relativo con $n$ entonces $a^{\phi(n)}\equiv 1$ mod $n$
\end{theorem}

\underline{\textbf{Prueba parecida a la 2da de Fermat.}}
Sea $a$ un entero tal que $(a,n)=1$.

Vamos a considerar el conjunto $S=\{ja: 1\leq j \leq n \text{ y } (j,n)=1\}$ y vamos a probar que es una permutación de los primos relativos a $n$.
\begin{itemize}
\item Los números $ja$ pertenicientes a $S$ son primos relativos a $n$, ya que es la multiplicacion de dos primos relativos a $n$, y entonces su multiplicación no comparte ningun factor con $n$.
\item Hay $\phi (n)$ distintos modulos primos relativos a $n$ por definición.
\item Si hay dos $1\leq j_1,j_2 \leq n$ distintos primos relativos a $n$ tal que $j_1a\equiv j_2 a$ mod $n$ entonces $(j_1-j_2)a\equiv 0$ mod $n$ entonces $n|(j_1-j_2)a$ pero como $(a,n)=1$ entonces $n|j_1-j_2$ y $j_1\equiv j_2$ mod $n$ pero como $1\leq j_1,j_2 \leq n$ entonces $j_1=j_2$ una contradicción.
\item Entonces los $\phi (n)$ $j$ son distintos y entonces hay $\phi (n)$ $ja$ diferentes y son primos relativos con $n$ entonces $S$ si es una permutacion de los primos relativos con $n$ en mod $n$.
\end{itemize}
Entonces 
$$\prod_{(j,n)=1}^{n} ja \equiv \prod_{(j,n)=1}^{n} j \text{ mod } n $$
$$\Rightarrow \prod_{(j,n)=1}^{n} ja = (\prod_{(j,n)=1}^{n}j) a^{\phi (n)} \equiv \prod_{(j,n)=1}^{n} j \text{ mod } n$$
y como $\prod_{(j,n)=1}^{n} j$ es la multiplicación de primos relativos con $n$ entonces es primo relativo con $n$ y 
$$a^{\phi(n)}\equiv 1 \text{ mod } n$$


\section{Wilson}
\begin{theorem}[Teorema de Wilson]
Si $p$ es primo entonces $(p-1)! \equiv -1$ mod $p$. 
\end{theorem}

\underline{\textbf{Prueba.}} \\

Sea $a$ un entero con $1\leq a \leq p-1$ y sea $b$ su inverso multiplicativo mod $p$. ($1\leq b \leq p-1$).

Tenemos que $ab\equiv 1$ mod $p$  y como es una ecuacion lineal sabemos que $ax\equiv 1$ mod $p$ tiene una solucion unica mod $p$, entonces el inverso de $a$ es $b$ y viceversa.

Entonces separamos los números del $1$ al $p-1$ en parejas de la forma $(a,b)$ donde $ab\equiv 1$ mod $p$ excepto cuando se tiene que $a=b$ es decir es el inverso de si mismo.
$$a^2\equiv 1 \text{ mod } p $$
$$(a+1)(a-1) \equiv 0 \text{ mod } p$$
$$a\equiv 1,-1 \text{ mod } p$$

Entonces todos los números del 2 al $p-2$ se emparejan y su producto es 1 mod $p$. 
Asi que 
$$(p-1)!=1\cdot(2\cdot3\cdots (p-2))\cdot (p-1)\equiv 1\cdot 1 \cdot (p-1)\equiv -1 \text{ mod } p$$
probando el teorema.



\section{ Algoritmo de Euclides}

Sea $a,b \in \mathbb{N}$ entonces podemos escribir
$$a=bq+r \text{  con } 0< r < b$$
$$b=rq_1+r_1 \text{  con } 0< r_1 < r$$
$$r=r_1q_2+r_2 \text{  con } 0< r_2<r_1$$
$$\vdots$$
Como $r_i>r_{i+1}$ siempre va decendiendo, entonces en algun punto va a llegar a ser 0. \\
$$r_{n-1}=r_nq_{n+1}+r_{n+1} \text{  con } 0< r_{n+1}<r_1$$
$$r_n=r_{n+1}q_{n+2}+0$$ 
Entonces se tiene que $(a,b)=r_{n+1}$

\section{Identidad de Bezout}
\begin{theorem}
    [Identidad de Bezout] Si $(a,b)=d$. Existen enteros $x,y$ tales que 
    $$ax+by=d$$
\end{theorem}
\begin{remark*}
En particular si $(a,b)=1$ existen enteros $x,y$ tales que 
$$ax+by=1$$
\end{remark*}

Esto se apoya con el Algoritmo de Euclides, veamos un ejemplo con $a=11, b=7$. 

\begin{example}
    $$11=7\cdot 1+4$$
    $$7=4\cdot 1+3$$
    $$4=3\cdot 1+1$$
    $$3=1\cdot 3+0$$

    De esta penultima se tiene que $(11,7)=1$ y podemos escribir
    $$1=4(1)-3(1)$$
    pero de la anterior tenemos que $3=7(1)-4(1)$ asi que sustituimos.
    $$1=4(1)-(7(1)-4(1))(1)=4(1)-7(1)+4(1)=4(2)-7(1)$$
    Y de la primera tenemos que $4=11(1)-7(1)$ asi que podemos sustituirlo
    $$1=4(2)-7(1)=(11(1)-7(1))(2)-7(1)=11(2)-7(2)-7(1)=11(2)-7(3)$$
    Entonces para $a=11, b=7$ tenemos que $11(2)+7(-3)=1$ cumpliendo la identidad.
\end{example}

Otro lema que se puede demostrar con Bezout es el Lema de Euclides
\begin{claim} [Lema de Euclides]
    Sea $p$ un primo, Si $p|ab \Rightarrow p|a$ o $p|b$.
\end{claim}
\underline{\textbf{Prueba.}} \\
Supongamos que $p|ab$ y $p\not | a \Rightarrow (a,p)=1$. Entonces por Bezout existen enteros $x,y$ con 
$$ax+py=1 \Rightarrow abx+pby=b$$
Y como $p|ab \Rightarrow p|abx$ y $p|pby$ entonces $p|abx+pby=b$ probando el Lema. \\

\section{Resolver Ecuaciones Modulares}

Primero veamos que para enteros $(a,n)=1$ sabemos que existen enteros $c$ y $d$ tales que $ac+dn=1 \Rightarrow ac=1-dn \Rightarrow ac\equiv 1$ mod $n$ 
Entonces $c$ es el inverso multiplicativo de $a$ mod $n$.\\

Por ejemplo anteriormente sacamos que $11(2)+7(-3)=1$ Entonces $7(-3)\equiv 1$ mod 11 y por lo que $c=-3$ es el inverso multiplicativo de 7 mod 11.

Asi que para resolver esta ecuacion: 
$$ax\equiv b \text{ mod } n$$
Multiplicamos por $c$ (el inverso de $a$).
$$acx \equiv bc \text{ mod } n$$
$$\Rightarrow x \equiv bc \text{ mod } n$$

Ya que $ac\equiv 1$ mod $n$ por definición.

Ahora que pasa si no son primos relativos, digamos $(a,n)=d$, entonces escribimos $a=da'$ y $n=dn'$ entonces tenemos que 
$$ax\equiv b \text{ mod }n \Rightarrow da'x \equiv b \text { mod } dn'$$
y 
$$dn' | da'x -b \Rightarrow d|da'x-b$$
 y como $d| da'x$ entonces $d|b$, de lo contrario no tiene solución el sistema. Entonces escribimos $b=db'$. \\
 Asi que la ecuacion queda:
$$ da'x \equiv db' \text{ mod } dn' \Rightarrow da'x-db'=(dn')k$$
para algun entero $k$ y seguimos a (dividiendo entre $d$):
$$a'x-b'=n'k \Rightarrow a'x\equiv b \text{ mod } n'$$
Y como $(a',n')=1$ porque quitamos todos sus factores en comun al dividir entre su maximo comun divisor, entonces esta ecuacion tiene una unica solución mod $n'$, y por lo tanto hay $d$ soluciones mod $n$.
Si la solucion mod $n'$ es $x\equiv x_0$ mod $n'$, entonces las $d$ soluciones son 
$$x_0, x_0+n', x_0+2n', \ldots , x_0+(d-1)n'$$
modulo $n$.

Tambien cuando el modulo es primo se vale pensar en fracciones, ejemplo con 7,11. \\
Sacamos anteriormente que $-3$ es el inverso multiplicativo de 7 mod 11, entonces $7(-3)\equiv 1$ mod 11, y entonces podemos pensar a $-3$ como $\frac{1}{7}$ mod 11.

\section{Problemas de Ejemplo}

\begin{example} [IMO 2005/4]
    Encuentra todos los enteros positivos que son primos relativos con todos los términos de la secuencia infinita $$ a_n=2^n+3^n+6^n -1,\ n\geq 1. $$
\end{example}

\underline{\textbf{Solución.}} \\
El unico número que cumple es el 1. \\
Supongamos que algun otro número $m$ cumpla con $m>1$, entonces tiene algun primo $p$ tal que $p|m$ y ese primo cumple.
Entonces vamos a probar que $p|a_n$ para algun $a_n$.
Primero para poder usar lo que hemos visto vamos a ver que pasa con $p=2,3$ porque son los factores q aparecen en 2,3 y 6 (todos los demas primos son primos relativos a estos números y podemos aplicar Fermat/Euler).
\begin{itemize}
    \item $p=2$ Notemos que $a_1=2+3+6-1=10$ y $2|10$ entonces si cumple.
    \item $p=3$ Notemos que $a_2=4+9+36-1=48$ y $3|48$ entonces si cumple.
\end{itemize}

Entonces ahora para $p>3$ queremos que $2^n+3^n+6^n-1\equiv 0$ mod $p$. \\
Primeramente lo que intentariamos seria usar Fermat (porque $p$ es primo), entonces intentamos con $n=p-1$ pero queda 
$$2^{p-1}+3^{p-1}+6^{p-1}-1 \equiv 1+1+1-1 \equiv 2 \text{ mod } p$$ 
entonces no cumple, ahora podemos usar lo de la idea de usar fracciones, y vemos que si $n=-1$ tenemos que 
$$\frac 1 2 +\frac 1 3 +\frac 1 3 -1 =0$$ 
entonces si cumple, pero como $n\geq 1$ no podemos usar esto, asi que tenemos que encontrar un número que sea masomenos esto, el cual como $x^{p-1}\equiv 1$ mod $p$, vamos a sumarle $p-1$ a el exponente y lo va a multiplicar por $1$, entonces con $n=p-2$ tenemos que 
$$2^{p-2}+3^{p-2}+6^{p-2}-1=2^{-1}\cdot 2^{p-1}+3^{-1}\cdot 3^{p-1}+6^{-1}\cdot 6^{p-1}-1\equiv \frac 1 2 \cdot 1 +\frac 1 3 \cdot 1 +\frac 1 6 \cdot 1 -1\equiv 0 \text{ mod } p $$
cumpliendo la condicion y demostrando que para cada número existe uno con el que no sea primo relativo.

\begin{example}  [PUTNAM 2022/A3]
    Sea $p$ un primo mayor a 5. Sea $f(p)$ el número de secuencias infinitas $a_1,a_2,a_3,\ldots$ que satisfaces
    \begin{center}
        \begin{enumerate}
            \item $a_n \in \{1,2, \ldots,p-1 \}$   $\forall n \geq 1$
            \item $a_na_{n+2}\equiv 1+a_{n+1}$ mod $p$ $\forall n\geq 1$
        \end{enumerate}
    \end{center}
\end{example}

\underline{\textbf{Solución.}} \\
Como $a_na_{n+2}\equiv 1+a_{n+1}$ mod $p$, y $(a_n,p)=1$ porque por definicion $1\leq a_n \leq p-1$ y entonces $p$ no lo puede dividir, asi que $a_{n+2}\equiv \frac{1+a_{n+1}}{a_n}$ mod $p$. \\
Entonces 
\begin{itemize}
    \item $a_3 \equiv \frac{1+a_2}{a_1}$ mod $p$.
    \item $a_4 \equiv \frac{1+a_3}{a_2} \equiv \frac{1+\frac{1+a_2}{a_1}}{a_2} \equiv \frac{1+a_1+a_2}{a_1a_2}$ mod $p$.
    \item $a_5 \equiv \frac{1+a_4}{a_3} \equiv \frac{1+\frac{1+a_1+a_2}{a_1a_2}}{a_3}\equiv \frac{1+a_1+a_2+a_1a_2}{a_1a_2a_3} \equiv \frac{(1+a_1)(1+a_2)}{a_2(1+a_2)}\equiv \frac{1+a_1}{a_2}$ mod $p$.
    \item $a_6 \equiv \frac{1+a_5}{a_4} \equiv \frac{1+\frac{1+a_1}{a_2}}{\frac{1+a_1+a_2}{a_1a_2}} \equiv \frac{1+a_1+a_2}{\frac{1+a_1+a_2}{a_1}} \equiv a_1$ mod $p$.
    \item $a_7 \equiv \frac{1+a_6}{a_5} \equiv \frac{1+a_1}{\frac{1+a_1}{a_2}}\equiv a_2$ mod $p$.
\end{itemize}
Y como la sucesion se va escribiendo en base a los dos anteriores llegamos a una periodicidad y entonces escribimos toda la sucesion en base de $a_1$ y $a_2$. (Como $1\leq a_n \leq p-1$ entonces $a_n$ van a ser a lo que son congruentes mod $p$. (Es decir $a_3=frac{1+a_2}{a_1}$)) \\
Pero nunca descartamos que no pueden ser 0 mod $p$ (es la unica congruencia mod $p$ que no se puede)
Como $a_3 \neq 0$ entonces $1+a_2\neq 0 \Rightarrow a_2 \neq -1$ mod $p$. \\
Como $a_5 \neq 0$ entonces $1+a_1 \neq 0 \Rightarrow a_1 \neq -1$ mod $p$. \\
Como $a_4 \neq 0$ entonces $1+a_1+a_2 \neq 0 \Rightarrow a_2 \neq -1-a_1$ mod $p$. \\
Asi que $f(p)=(p-2)(p-3)$, porque tenemos $p-2$ formas de escoger $a_1$ ya que tenemos $p-1$ opciones originalmente pero no puede estar $p-1$, y para $a_2$ tenemos originalmente $p-1$ opciones pero como no puede ser $-1$ mod $p$ y tampoco puede ser $-1-a_1$ y como $a_1$ no es ninguna de $0,-1$ entonces $-1-a_1$ no es ninguna de las dos opciones que no pueden ser por lo que descartamos otra opcion y solo quedan $p-3$ opciones.
Entonces 
$$f(p)=(p-2)(p-3)=p^2-5p+6 \equiv p^2+1 \text{ mod } 5 $$
Como $p>5$, $p \not \equiv 0$ mod 5. \\
Y $p^2$ mod 5 puede ser $1^2,2^2,3^2,4^2 \rightarrow 1,4,4,1$ mod 5, entonces $p^2+1$ puede ser congruente a $1+1\equiv 2$ mod 5 o $4+1\equiv 0$ mod 5, demostrando el problema.


\subsection{Ejercicios de Practica}
\begin{exercise}
Resuelve para $x$:
\begin{enumerate}
    \item $12x\equiv 1 \text{ mod } 23$
    \item $x^2\equiv 1 \text{ mod } 23$
    \item $x^2 \equiv 1 \text{ mod } 8$
    \item $x(x+5) \equiv 6 \text{ mod } 10$
\end{enumerate}
\end{exercise}
\begin{exercise}
Encuentra todos los primos $p$ tales que $13^{2p-1}+17$ es divisible por $p$.
\end{exercise}

\begin{problem}
    Prueba que  
    $$(x-1^2)(x-2^2)(x-3^2)(x-4^2)(x-5^2)(x-6^2) \equiv x^6-1 \text{ mod } 13$$
\end{problem}
\begin{problem}
Encuentra todas las tripletas de enteros positivos $(k,m,n)$ tal que $7^k=9^m+2^n$.
\end{problem}

\begin{problem} 
Prueba que si $p$ y $8p-1$ son ambos primos entonces $8p+1$ es compuesto.
\end{problem}

\begin{problem}
Sea $f(x_1,x_2,\ldots,x_n)$ un polinomio con coeficientes enteros. Prueba que 
$$f(x_1,x_2, \ldots ,x_n)^p \equiv f(x_1^p,x_2^p, \ldots, x_n^p) \text{ mod } p$$
para $p$ primo.
\end{problem}

\begin{problem} [IMO 1970/4]
Encuentra el conjunto de todos los enteros positivos $n$ con la propiedad que el conjunto $\{n,n+1,n+2,n+3,n+4,n+5\}$ puede ser dividido en dos conjuntos tal que el producto de los numeros en un conjunto sea igual al producto de los numeros del otro conjunto.
\end{problem}

\end{document}