\documentclass[11pt]{scrartcl}

\usepackage[sexy]{evan}
\usepackage{import}
\usepackage[table]{xcolor}
\usepackage{xfp}

\newcommand{\gad}{\textcolor{yellow}{$\bigstar$}}
\newcommand{\bad}{\textcolor{red}{$\bigstar$}}

\definecolor{dcol0}{HTML}{C8E6C9}
\definecolor{dcol1}{HTML}{D4E9B3}
\definecolor{dcol2}{HTML}{E5ED9A}
\definecolor{dcol3}{HTML}{FFF59D}
\definecolor{dcol4}{HTML}{FFE082}
\definecolor{dcol5}{HTML}{FFCC80}
\definecolor{dcol6}{HTML}{FFAB91}
\definecolor{dcol7}{HTML}{F49890}
\definecolor{dcol8}{HTML}{E57373}
\definecolor{dcol9}{HTML}{D32F2F}

\makeatletter
\newcommand{\getcolorname}[1]{dcol#1}
\makeatother

\newcommand{\dif}[1]{%
    \edef\colorindex{\number\fpeval{floor(#1)}}%
    \edef\fulltext{#1}%
    \colorbox{\getcolorname{\colorindex}}{%
        \ifnum\colorindex>8
            \textbf{\textcolor{white}{\,\fulltext\,}}%
        \else
            \textbf{\textcolor{black}{\,\fulltext\,}}%
        \fi
    }%
}
% Variable para dificultad (inicial 0)
\newcommand{\thmdifficulty}{0}

% Comando para asignar dificultad antes del problema
\newcommand{\problemdiff}[1]{\renewcommand{\thmdifficulty}{#1}}

% Estilo del problema que incluye dificultad antes del título
\declaretheoremstyle[
    headfont=\color{blue!40!black}\normalfont\bfseries,
    headformat={%
      \dif{\thmdifficulty}\quad \NAME~\NUMBER\ifx\relax\EMPTY\relax\else\ \NOTE\fi
    },
    postheadspace=1em,
    spaceabove=8pt,
    spacebelow=8pt,
    bodyfont=\normalfont
]{problemstyle}

    \declaretheorem[style=problemstyle,name=Problema,sibling=theorem]{problema}
    \declaretheorem[style=problemstyle,name=Problema,numbered=no]{problema*}

\title{Modulos}
\subtitle{OMMJAL}
\author{Emmanuel Buenrostro}

\begin{document}

\maketitle.


\section{Principios}
\subsection{Definici\'on}

\begin{definition}
$a \equiv b \pmod n$ si $n\mid a-b$ 
\end{definition}
\textbf{ Propiedades:}
Si $a\equiv b \pmod n$ y $c \equiv d \pmod n$:
\begin{enumerate}
\item $a+c\equiv b+d  \pmod n$
\item $a-c \equiv b-d \pmod n$
\item $ac \equiv bd \pmod n$
\item $a^x\equiv b^x \pmod n$
\end{enumerate}
Y en general cualquier operaci\'on con suma, resta, multiplicaci\'on se puede hacer.

\begin{remark*}
Podemos notar que no se menciona la divisi\'on, m\'as adelante tratamos con esta.
\end{remark*}

\subsection{Lema de la Division de Euclides}
Tomar $a$ mod $n$ en realidad lo que estamos haciendo es asignarle al entero $a$ algun valor de $\{0,1, \ldots, n-1\}$, este valor es el residuo que deja $a$ al dividirlo entre $n$, entonces vamos a demostrar que con la definici\'on que tenemos si asignamos \textbf{exactamente} un valor de  $\{0,1, \ldots, n-1\}$ \\

Vamos a considerarnos el conjunto de multiplos de $5$ no negativos.
\[\{ 0,5,10,15,20,25,30,35,40,45,50,55, \ldots \}\]
Entonces, ¿Qu\'e pasa con los n\'umeros como $32, 33$ que no estan en ese conjunto?, pues estan en medio entre $30$ y $35$, entonces podemos escribirlos como
\[32 = 5 \times 6 + 2\]
\[33 = 5 \times 6 + 3\]
\[34 = 5 \times 6 + 4\]
\[35 = 5 \times 6 + 5\]
\[36 = 5 \times 6 + 6\]

Pero $35$ y $36$ ya no estan en medio de $30$ y $35$ entonces mejores formas de escribirlos seria
\[35= 5\times 7+0\]
\[36= 5 \times 7 +1\]

Siguiendo esto obtenemos el lema:
\begin{lemma} [Lema de la Division de Euclides]
Para cualesquiera enteros $a,b$ podemos encontrar \textbf{\'unicos} enteros $q,r$ tales que 
\[b = aq+r\]
con $0 \leq r <a$, donde $q$ es el cociente y $r$ el residuo.
\end{lemma}

\begin{proof}
Para ver que existen, unicamente haz este proceso: Inicia con $r=b$ y $q=0$, entonces mientras $r\geq a$ restale $a$ a $r$ y sumale $1$ a $q$, esto sigue siendo igual a $b$:
\[b=aq+r = aq+a+r-a = a(q+1)+(r-a)\]
Entonces cuando finalmente $r<a$ tienes unos enteros $q,r$ que cumplan. \\

Ahora para ver que son \'unicos, haremos contradicci\'on, si asumes que tienes dos parejas $(q_1,r_1)$ y $(q_2, r_2)$ que cumplen, entonces 
\begin{align*}
aq_1+r_1&=b=aq_2+r_2 \\
aq_1-aq_2&=r_2-r_1 \\
a(q_1-q_2)&=r_2-r_1 
\end{align*}
Entonces $r_2-r_1$ es m\'ultiplo de $a$, pero como $0\leq r_1,r_2 < a$ entonces $-a < r_2-r_1  < a$, y la unica posibilidad de que sea multiplo de $a$ es que $r_2-r_1=0$ y $r_1=r_2$, entonces $aq_1=aq_2 \Rightarrow q_1=q_2$ y son la misma pareja, contradicci\'on.

\end{proof}

Entonces $b \equiv r \pmod a$ porque $a \mid aq = b-r$ , y como $r$ es \'unico, entonces si asignamos exactamente un valor a cada entero positivo $b$ modulo $a$. \\

En particular si $x=aq_1+r_1, y=aq_2+r_2$ entonces $x \equiv y \pmod a \iff r_1=r_2$.


\begin{exercise}
Prueba que para los negativos tambien sucede.
\end{exercise}
\begin{remark*}
Usar modulos negativos suele servir para operaci\'ones, por ejemplo $n-1 \equiv -1 \pmod n$, entonces $(n-1)^2 \equiv (-1)^2 \equiv 1 \pmod n$, algo que sin usar el modulo negativo ser\'ia bastante mas complejo.
\end{remark*}

\subsection{¿C\'omo se usa?}
Los modulos tienen distintos usos, creo que los dos m\'as usuales es cosas directamente relacionadas con la divisibilidad (o por ejemplo calcular algun modulo), o el poder resolver distintas ecuaciones diofantinas (con enteros) al acotar bastante en que casos se puede y no se puede. \\
Veamos unos problemas de ejemplo. 

\begin{example} 
Encuentra que valores de $n$ se tiene que $3 \mid n^2+1$.
\end{example}

\begin{soln}
Como queremos que $3 \mid n^2+1$ entonces queremos que $3 \mid n^2-(-1)$ y entonces quieres $n^2 \equiv -1 \pmod 3$.  \\
Otra forma de ver esto, es que si $3 \mid n^2+1$ entonces
\begin{align*}
n^2+1 &\equiv 0 \pmod 3 \\
n^2 &\equiv -1 \pmod 3
\end{align*}
Puedes moverlo como si fuera una ecuaci\'on normal. \\

Entonces ahora vamos a ver todos los posibles casos de $n^2 \pmod 3$. 
\begin{center}
\begin{tabular}{|c|c|}
\hline
$n \pmod 3$ &  $n^2\pmod 3$\\
\hline
0 &  0 \\
\hline
1 &  1\\
\hline
2 &  $ 4 \equiv 1 $\\
\hline
\end{tabular}
\end{center}

Entonces podemos notar que no hay ningun caso donde $n^2 \equiv -1 \pmod 3$ entonces no hay soluciones. \\
\end{soln}
\begin{example}
¿Cu\'al es el residuo de $2^{2025}$ al dividirlo entre 7?
\end{example}

\begin{soln}
Vamos a ver los primeros casos de potencias de 2 modulo 7. \\
\[2^1 \equiv 2, 2^2 \equiv 4, 2^3 \equiv 1, 2^4 \equiv 2\]
Entonces podemos ver que volvemos al 2, pero el residuo de una potencia de 2 depende totalmente del residuo anterior, entonces si se repite una se va a hacer un ciclo, en este caso el ciclo es
\[2 \rightarrow 4 \rightarrow 1 \]
Entonces ahora lo que nos importa es ver cuanto es 2025 modulo 3 (porque el ciclo es de tamaño 3), y como $2025 \equiv 0 \pmod 3$ entonces $2^{2025} \equiv 1 \pmod 7$.
\end{soln}






\newpage
\epigraph{"Aqui mataron mucha gente"
}{El lider de Per\'u en Tiananmen Square}
\section{Problemas}
\subsection{Calcular mods}
\problemdiff{0}
\begin{problema}
Obten tres n\'umeros que sean congruentes a $a \pmod m$ para:
\begin{itemize}
\item $a=2, m=3$
\item $a=-1, m=11$
\item $a=52, m=17$
\item $a=-16, m=6$
\end{itemize}
\end{problema}

\problemdiff{0}
\begin{problema}
Calcula $a$ donde $a \in \{0,1,\ldots, 6\}$ y: 
\[11 \cdot 18 \cdot 2322 \cdot 13 \cdot 19 \equiv a \pmod 7\]
\end{problema}

\problemdiff{0}
\begin{problema}
Demuestra que si $x \equiv 11 \pmod{24}$ entonces $3x \equiv 1 \pmod 8$
\end{problema}

 \problemdiff{0}
\begin{problema}
Si en este momento son las $10$ de la mañana, ¿Qu\'e hora sera en $2500$ horas?
\end{problema}

\problemdiff{0}
\begin{problema}
Encuentra el \'ultimo digito de $2\times 325 + 3\times 8^7 \times 5104+ 123^5$.
\end{problema}

\problemdiff{0}
\begin{problema}
Resuelve para $x$:
\begin{enumerate}
    \item $12x\equiv 1 \text{ mod } 23$
    \item $x^2\equiv 1 \text{ mod } 23$
    \item $x^2 \equiv 1 \text{ mod } 8$
    \item $x(x+5) \equiv 6 \text{ mod } 10$
\end{enumerate}
\end{problema}

\newpage
\subsection{Usar mods pt 1}
\problemdiff{0}
\begin{problema}
¿Para cu\'ales enteros $n$ se tiene que $4 \mid 3n^3+1$ ?
\end{problema}

\problemdiff{0}
\begin{problema}
Demuestra el criterio de divisibilidad del $3$, el cual dice que un n\'umero es divisible entre 3 si la suma de sus digitos es divisible entre 3.
\end{problema}

\problemdiff{0}
\begin{problema}
Demuestra el criterio de divisibilidad del $4$, el cual dice que un n\'umero es multiplo de 4 si el n\'umero formado por sus dos ultimos digitos es multiplo de 4.
\end{problema}

\problemdiff{0}
\begin{problema}
Demuestra el criterio de divisibilidad de $2^n$, el cual dice que que un numero es divisible por $2^n$ si el número formado por los ultimos $n$ digitos es multiplo de $2^n$.
\end{problema}

\problemdiff{0}
\begin{problema}
Demuestra que un numero es divisible por $5^n$ si el número formado por los ultimos $n$ digitos es multiplo de $5^n$.
\end{problema}

\problemdiff{0}
\begin{problema}
Encuentra los enteros $n$ tales que $n^2 \equiv 1 \pmod 8$.
\end{problema}

\problemdiff{0}
\begin{problema}
Encuentra los enteros $x$ tales que $12x \equiv 1 \pmod{23}$.
\end{problema}

\problemdiff{0}
\begin{problema}
Para un entero positivo $n$, sea $A(n)$ la suma de los digitos de $n$, por ejemplo $A(24135)= 5 +3 +1 +4 +2$. Prueba que $n \equiv A(n) \pmod{9}$.
\end{problema}

\problemdiff{0}
\begin{problema}
Se tienen $2003$ tarjetas n\'umeradas del $1$ al $2003$ y colocadas hacia abajo en orden en un m\'onton (la tarjeta con el $1$ aparece arriba). Sin mirar se quitan tres tarjetas consecutivas hasta que solo quedan dos tarjetas. ¿Es posible que haya quedado la tarjeta con el $1002$?
\end{problema}

\problemdiff{1}
\begin{problema}
¿Puede $222222$ ser un cuadrado perfecto?
\end{problema}

\newpage
\subsection{Usar mods pt2}

\problemdiff{1}
\begin{problema}
Demuestra que $a-b \mid a^n-b^n$ para $n$ entero no negativo. 
\end{problema}

\problemdiff{1}
\begin{problema}
Demuestra que los n\'umeros primos mayores a 3 son 1 o $5 \pmod 6$. 
\end{problema}

\problemdiff{1}
\begin{problema}
Demuestra que si para $x,y$ enteros entonces si $3\mid x^2+y^2$ se tiene que $3\mid x$ y $3\mid y$. 
\end{problema}

\problemdiff{1}
\begin{problema}
Demuestra que si $7 \mid x^3+y^3+z^3$ entonces $7 $ divide a alguno de $x,y,z$.
\end{problema}

\problemdiff{1}
\begin{problema}
Demuestra que si $a^2 \equiv b^2 \pmod p$ donde $p$ es primo entonces $a\equiv b \pmod p$ o $a\equiv -b \pmod p$.
\end{problema}

\problemdiff{1}
\begin{problema}
Demuestra que un número es multiplo de 7 si el número formado por los numeros excepto el ultimo digito - el doble de el ultimo digito es multiplo de 7.
\end{problema}

\problemdiff{1}
\begin{problema}
Encuentra todas las tripletas de enteros positivos $(k,m,n)$ tal que $7^k=9^m+2^n$.
\end{problema}

\problemdiff{1}
\begin{problema}
Para un entero positivo $n$, sea $A(n)$ la suma alternada de los digitos de $n$, por ejemplo $A(24135)= 5 -3 +1 -4 +2$. Prueba que $n \equiv A(n) \pmod{11}$.
\end{problema}

\problemdiff{1}
\begin{problema} 
Prueba que si $p$ y $8p-1$ son ambos primos entonces $8p+1$ es compuesto.
\end{problema}

\problemdiff{1}
 \begin{problema}
    Prueba que  
    $$(x-1^2)(x-2^2)(x-3^2)(x-4^2)(x-5^2)(x-6^2) \equiv x^6-1 \text{ mod } 13$$
\end{problema}

\problemdiff{1}
\begin{problema}
Sea $S$ la suma siguiente:
\[S=1+2+3+\ldots+2025\]
Encuentra el residuo cuando $S$ es dividido entre 7.
\end{problema}



\problemdiff{1}
\begin{problema}[AIME 2010/1]
Encuentra el residuo cuando $9 \times 99 \time 999\times \ldots \times 99\cdots9$ con $999$ $9$'s al final es dividido entre $1000$. 
\end{problema}


\problemdiff{2}
\begin{problema}
Demuestra que 
\[2025 \mid 1^{2025}+2^{2025}+3^{2025}+\ldots + 2025^{2025}\]
\end{problema}

\problemdiff{2}
\begin{problema}
Determina todas las soluciones enteras no negativas $(n_1,n_2, \ldots, n_{14})$ a
\[n_1^4+n_2^4+\ldots+n_{14}^4 =1599\]
\end{problema}

\problemdiff{2}
\begin{problema} 
[ORO 2021/3]
La secuencia de enteros positivos $a_1,a_2, \ldots$ esta definida de la siguiente forma: $a_1=2019, a_2=2020, a_3=2021$ y para todo $n\geq 1$
\[a_{n+3}=5a_{n+2}^6+3a_{n+1}^3+a_n^2\]
Prueba que la secuencia no contiene n\'umeros de la forma $m^6$ donde $m$ es un entero positivo.

\end{problema}

\newpage
\subsection{Mas problemas}
Puede que ocupes mas cosas (Como Fermat o Euler) aqui.

\problemdiff{1}
\begin{problema}
Encuentra todos los primos $p$ tales que $13^{2p-1}+17$ es divisible por $p$.
\end{problema}


\problemdiff{3.5}
\begin{problema}[PUMaC 2012 N A3]
Definimos la secuencia $\{x_n\}$ de la siguiente forma: $x_1 \in \{5,7\}$ y para todo $k\geq 1, x_{k+1} \in \{5^{x_k}, 7^{x_k}\}$. Por ejemplo los posibles valores de $x_3$ son: $5^{5^5}, 5^{5^7}, 5^{7^5}, 5^{7^7}, 7^{5^5}, 7^{5^7}, 7^{7^5}$. ¿Cu\'al es la suma de todos los posibles valores para los ultimos dos digitos de $x_{2012}$.
\end{problema}

\problemdiff{4}
\begin{problema}[IMOSL 2000 N1]
Encuentra todos los enteros positivos $n \geq 2$ que cumplen que para todos los enteros positivos coprimos con $n$ $a,b$ se cumple que 
\[a\equiv b \pmod n \text{ si y solo si } ab \equiv 1 \pmod n\]
\end{problema}


\newpage
\section{¿Y la divisi\'on?}
\subsection{Definici\'on de inverso}
Al momento de hacer la divisi\'on ocupamos la existencia de \textit{inversos}.  Al hacer divisi\'on en los reales, si queremos dividir $x$ entre $y$, lo que en realidad estamos haciendo es multiplicar $x$ por el \textit{inverso} de $y$ el cual es representado como $\frac 1y$.  \\

¿Y qu\'e cumplen los inversos? En los reales lo que cumplen es que el inverso de un real $a$, es aquel que cumple 
\[a \cdot a^{-1} = 1\]
Entonces eso justo coincide con lo que conocemos de $\frac 1a$. \\

Entonces si estamos trabajando mod $n$, un entero $a$ tiene inverso si existe un $a^{-1}$ con 
\[a \cdot a^{-1} \equiv  1 \pmod n \]
Y este existe si y solo si $\gcd(a,n)=1$.\\

Podemos ver un ejemplo cuando $n=7$.

\begin{center}
\begin{tabular}{|c|c|}
\hline
\textbf{Número} & \textbf{Inverso mod 7} \\
\hline
1 & 1 \\
2 & 4 \\
3 & 5 \\
4 & 2 \\
5 & 3 \\
6 & 6 \\
\hline
\end{tabular}
\end{center}
Mostrandonos que efectivamente en este caso si existen, entonces vamos a demostrar esto.

\subsection{Maquinaria a usar}






\newpage
\section{Teoremitas \'utiles}



\begin{theorem}[Pequeño Teorema de Fermat]
 Si $p$ es primo y $a\in \mathbb{Z}$ entonces 

$$a^p \equiv a \text{ mod } p$$
\end{theorem}
Tambien generalmente conocido como 

\begin{theorem*}[Pequeño Teorema de Fermat]
 Si $p$ es primo y $a\in \mathbb{Z}$ y $(a,p)=1$ entonces 
$$a^{p-1} \equiv 1 \text{ mod } p$$
\end{theorem*}

\begin{proof}
Notemos que para $a$ tal que $p|a$ si cumple porque $a^p\equiv 0 \equiv a$ mod $p$.

Para los $a$ con $(p,a)=1$. se tiene que 
$$\{a,2a,3a, \ldots, (p-1)a\}$$
es una permutación de 
$$\{1,2,3,\ldots p-1\}$$
en mod $p$. 

\underline{\textbf{Prueba.}} Todos los números $a,2a,3a,\ldots, (p-1)a$ tiene distintos modulos $p$, ya que si hay dos iguales $ia$ y $ja$ con $i\neq j$ entonces 
$$ia-ja= a(i-j) \equiv 0 \text{ mod } p$$
y como $(a,p)=1$ entonces $i-j \equiv 0$ mod $p \Rightarrow i\equiv j$ mod $p$, pero $1\leq i,j \leq p-1$ entonces $i=j$, una contradicción.
Entonces si es una permutación.  \\

Asi que si multiplicamos todos se tiene que 

$$a\cdot 2a\cdot 3a \cdots (p-1)a \equiv 1\cdot 2 \cdots (p-1) \text{ mod } p$$
entonces
$$ \Rightarrow(p-1)!a^{p-1}\equiv (p-1)! \text{ mod } p $$
$$ \Rightarrow a^{p-1}\equiv 1 \text{ mod }p$$
\end{proof}



\end{document}