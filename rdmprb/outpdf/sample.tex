\documentclass[11pt]{scrartcl}

\usepackage[sexy]{evan}
\usepackage{microtype}
\microtypesetup{expansion=false}
\usepackage{pgfplots}
\pgfplotsset{compat=1.15}
\usepackage{mathrsfs}
\usetikzlibrary{arrows}
\usepackage{graphics}
\usepackage{listings}
\usepackage{tikz}
\usepackage{ amssymb }
\usepackage[dvipsnames]{xcolor}
\definecolor{red1}{RGB}{255, 153, 153}
\definecolor{green1}{RGB}{204, 255, 204}
\definecolor{blue1}{RGB}{204, 255, 255}
\definecolor{yellow1}{RGB}{255, 247, 160}

\definecolor{red2}{RGB}{255, 102, 102}
\definecolor{green2}{RGB}{108, 255, 108}
\definecolor{blue2}{RGB}{94, 204, 255}
\definecolor{yellow2}{RGB}{255, 250, 104}

\definecolor{red2.5}{RGB}{255,76,76}
\definecolor{green2.5}{RGB}{54, 247, 54}
\definecolor{blue2.5}{RGB}{51, 189, 255}
\definecolor{yellow2.5}{RGB}{255, 242, 52}


\definecolor{red3}{RGB}{255, 51, 51}
\definecolor{green3}{RGB}{0, 240, 0}
\definecolor{blue3}{RGB}{9, 175, 255}
\definecolor{yellow3}{RGB}{255, 234, 0}

\definecolor{red3.5}{RGB}{229, 25, 25}
\definecolor{green3.5}{RGB}{0, 194, 0}
\definecolor{blue3.5}{RGB}{4, 143, 209}
\definecolor{yellow3.5}{RGB}{255,220,0}

\definecolor{red4}{RGB}{204, 0, 0}
\definecolor{green4}{RGB}{0, 149, 0}
\definecolor{blue4}{RGB}{0, 111, 164}
\definecolor{yellow4}{RGB}{255, 206, 0}

\newcommand{\camod}[1]{\frac{\ZZ}{#1 \ZZ}}
\newcommand{\modm}[1]{\text{ mod } #1}
\newcommand{\campm}[1]{\frac{\ZZ}{m\ZZ}}
\newcommand{\paren}[1]{\left(  #1 \right)}
\newcommand{\proc}[1]{Esto va a aparecer si no procrastino :S (Procrastinando desde #1)}


\title{quiero ser bueno }
\subtitle{PONTE A ENTRENAR}
\author{Emmanuel Buenrostro}


\begin{document}

\maketitle

\section{Problemas}
\begin{problem}
\label{931951248564234}
Let $n > 3$ be a positive integer. Suppose that $n$ children are arranged in a circle, and $n$ coins are distributed between them (some children may have no coins). At every step, a child with at least 2 coins may give 1 coin to each of their immediate neighbors on the right and left. Determine all initial distributions of the coins from which it is possible that, after a finite number of steps, each child has exactly one coin.
\end{problem}
\begin{problem}
\label{543318535845123}
Show that $r = 2$ is the largest real number $r$ which satisfies the following condition:

If a sequence $a_1$, $a_2$, $\ldots$ of positive integers fulfills the inequalities
\[a_n \leq a_{n+2} \leq\sqrt{a_n^2+ra_{n+1}}\]for every positive integer $n$, then there exists a positive integer $M$ such that $a_{n+2} = a_n$ for every $n \geq M$.
\end{problem}
\begin{problem}
\label{716406996122549}
Determine all functions $f: \mathbb{R} \rightarrow \mathbb{R}$ such that$$f(xf(x-y))+yf(x)=x+y+f(x^2),$$for all real numbers $x$ and $y.$
\end{problem}
\begin{problem}
\label{748681263295975}
We are given an acute triangle $ABC$. The angle bisector of $\angle BAC$ cuts $BC$ at $P$. Points $D$ and $E$ lie on segments $AB$ and $AC$, respectively, so that $BC \parallel  DE$. Points $K$ and $L$ lie on segments $PD$ and $PE$, respectively, so that points $A$, $D$, $E$, $K$, $L$ are concyclic. Prove that points $B$, $C$, $K$, $L$ are also concyclic.
\end{problem}
\begin{problem}
\label{12311699525330}
	Suppose $a_{1} < a_{2}< \cdots < a_{2024}$ is an arithmetic sequence of positive integers, and $b_{1} <b_{2} < \cdots <b_{2024}$ is a geometric sequence of positive integers. Find the maximum possible number of integers that could appear in both sequences, over all possible choices of the two sequences.
\end{problem}
\section{Fuentes}
\begin{itemize}
\item \ref{931951248564234} IMOSL 2022 C4
\item \ref{543318535845123} APMO 2020/2
\item \ref{716406996122549} Ibero 2020/5
\item \ref{748681263295975} IGO Advanced 2023/1
\item \ref{12311699525330} USA TST 2024/5
\end{itemize}

\end{document}
