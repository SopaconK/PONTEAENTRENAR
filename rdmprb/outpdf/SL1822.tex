\documentclass[11pt]{scrartcl}

\usepackage[sexy]{evan}
\usepackage{pgfplots}
\pgfplotsset{compat=1.15}
\usepackage{mathrsfs}
\usetikzlibrary{arrows}
\usepackage{graphics}
\usepackage{listings}
\usepackage{tikz}
\usepackage{ amssymb }
\usepackage[dvipsnames]{xcolor}
\definecolor{red1}{RGB}{255, 153, 153}
\definecolor{green1}{RGB}{204, 255, 204}
\definecolor{blue1}{RGB}{204, 255, 255}
\definecolor{yellow1}{RGB}{255, 247, 160}

\definecolor{red2}{RGB}{255, 102, 102}
\definecolor{green2}{RGB}{108, 255, 108}
\definecolor{blue2}{RGB}{94, 204, 255}
\definecolor{yellow2}{RGB}{255, 250, 104}

\definecolor{red2.5}{RGB}{255,76,76}
\definecolor{green2.5}{RGB}{54, 247, 54}
\definecolor{blue2.5}{RGB}{51, 189, 255}
\definecolor{yellow2.5}{RGB}{255, 242, 52}


\definecolor{red3}{RGB}{255, 51, 51}
\definecolor{green3}{RGB}{0, 240, 0}
\definecolor{blue3}{RGB}{9, 175, 255}
\definecolor{yellow3}{RGB}{255, 234, 0}

\definecolor{red3.5}{RGB}{229, 25, 25}
\definecolor{green3.5}{RGB}{0, 194, 0}
\definecolor{blue3.5}{RGB}{4, 143, 209}
\definecolor{yellow3.5}{RGB}{255,220,0}

\definecolor{red4}{RGB}{204, 0, 0}
\definecolor{green4}{RGB}{0, 149, 0}
\definecolor{blue4}{RGB}{0, 111, 164}
\definecolor{yellow4}{RGB}{255, 206, 0}

\newcommand{\camod}[1]{\frac{\ZZ}{#1 \ZZ}}
\newcommand{\modm}[1]{\text{ mod } #1}
\newcommand{\campm}[1]{\frac{\ZZ}{m\ZZ}}
\newcommand{\paren}[1]{\left(  #1 \right)}
\newcommand{\proc}[1]{Esto va a aparecer si no procrastino :S (Procrastinando desde #1)}


\title{SL1822}
\subtitle{PONTE A ENTRENAR}
\author{Emmanuel Buenrostro}


\begin{document}

\maketitle

\section{Problemas}
\begin{problem}
Find all triples $(a,b,p)$ of positive integers with $p$ prime and\[ a^p=b!+p. \]
\end{problem}
\begin{problem}
	Let $ABC$ be an acute-angled triangle with $AC > AB$, let $O$ be its circumcentre, and let $D$ be a point on the segment $BC$. The line through $D$ perpendicular to $BC$ intersects the lines $AO, AC,$ and $AB$ at $W, X,$ and $Y,$ respectively. The circumcircles of triangles $AXY$ and $ABC$ intersect again at $Z \ne A$.
Prove that if $W \ne D$ and $OW = OD,$ then $DZ$ is tangent to the circle $AXY.$
\end{problem}
\begin{problem}
Let $ABC$ be a triangle and $\ell_1,\ell_2$ be two parallel lines. Let $\ell_i$ intersects line $BC,CA,AB$ at $X_i,Y_i,Z_i$, respectively. Let $\Delta_i$ be the triangle formed by the line passed through $X_i$ and perpendicular to $BC$, the line passed through $Y_i$ and perpendicular to $CA$, and the line passed through $Z_i$ and perpendicular to $AB$. Prove that the circumcircles of $\Delta_1$ and $\Delta_2$ are tangent.
\end{problem}
\begin{problem}
	Two squirrels, Bushy and Jumpy, have collected 2021 walnuts for the winter. Jumpy numbers the walnuts from 1 through 2021, and digs 2021 little holes in a circular pattern in the ground around their favourite tree. The next morning Jumpy notices that Bushy had placed one walnut into each hole, but had paid no attention to the numbering. Unhappy, Jumpy decides to reorder the walnuts by performing a sequence of 2021 moves. In the $k$-th move, Jumpy swaps the positions of the two walnuts adjacent to walnut $k$.

Prove that there exists a value of $k$ such that, on the $k$-th move, Jumpy swaps some walnuts $a$ and $b$ such that $a<k<b$.
\end{problem}
\begin{problem}
Suppose that $a,b,c,d$ are positive real numbers satisfying $(a+c)(b+d)=ac+bd$. Find the smallest possible value of
$$\frac{a}{b}+\frac{b}{c}+\frac{c}{d}+\frac{d}{a}.$$
\end{problem}
\begin{problem}
	A social network has $2019$ users, some pairs of whom are friends. Whenever user $A$ is friends with user $B$, user $B$ is also friends with user $A$. Events of the following kind may happen repeatedly, one at a time:
Three users $A$, $B$, and $C$ such that $A$ is friends with both $B$ and $C$, but $B$ and $C$ are not friends, change their friendship statuses such that $B$ and $C$ are now friends, but $A$ is no longer friends with $B$, and no longer friends with $C$. All other friendship statuses are unchanged.
Initially, $1010$ users have $1009$ friends each, and $1009$ users have $1010$ friends each. Prove that there exists a sequence of such events after which each user is friends with at most one other user.

\end{problem}
\begin{problem}
Determine all pairs $(n, k)$ of distinct positive integers such that there exists a positive integer $s$ for which the number of divisors of $sn$ and of $sk$ are equal.
\end{problem}
\begin{problem}
Let $m,n \geqslant 2$ be integers, let $X$ be a set with $n$ elements, and let $X_1,X_2,\ldots,X_m$ be pairwise distinct non-empty, not necessary disjoint subset of $X$. A function $f \colon X \to \{1,2,\ldots,n+1\}$ is called nice if there exists an index $k$ such that\[\sum_{x \in X_k} f(x)>\sum_{x \in X_i} f(x) \quad \text{for all } i \ne k.\]Prove that the number of nice functions is at least $n^n$.
\end{problem}
\begin{problem}
Let $ABC$ be a triangle with circumcircle $\Omega$ and incentre $I$. A line $\ell$ intersects the lines $AI$, $BI$, and $CI$ at points $D$, $E$, and $F$, respectively, distinct from the points $A$, $B$, $C$, and $I$. The perpendicular bisectors $x$, $y$, and $z$ of the segments $AD$, $BE$, and $CF$, respectively determine a triangle $\Theta$. Show that the circumcircle of the triangle $\Theta$ is tangent to $\Omega$.
\end{problem}
\begin{problem}
ind all triples $(a, b, c)$ of positive integers such that $a^3 + b^3 + c^3 = (abc)^2$.
\end{problem}
\begin{problem}
A number is called Norwegian if it has three distinct positive divisors whose sum is equal to $2022$. Determine the smallest Norwegian number.
(Note: The total number of positive divisors of a Norwegian number is allowed to be larger than $3$.)
\end{problem}
\begin{problem}
Let $S$ be an infinite set of positive integers, such that there exist four pairwise distinct $a,b,c,d \in S$ with $\gcd(a,b) \neq \gcd(c,d)$. Prove that there exist three pairwise distinct $x,y,z \in S$ such that $\gcd(x,y)=\gcd(y,z) \neq \gcd(z,x)$.
\end{problem}
\begin{problem}
Let $ABCD$ be a parallelogram with $AC=BC.$ A point $P$ is chosen on the extension of ray $AB$ past $B.$ The circumcircle of $ACD$ meets the segment $PD$ again at $Q.$ The circumcircle of triangle $APQ$ meets the segment $PC$ at $R.$ Prove that lines $CD,AQ,BR$ are concurrent.
\end{problem}
\begin{problem}
Let $ABC$ be an acute-angled triangle and let $D, E$, and $F$ be the feet of altitudes from $A, B$, and $C$ to sides $BC, CA$, and $AB$, respectively. Denote by $\omega_B$ and $\omega_C$ the incircles of triangles $BDF$ and $CDE$, and let these circles be tangent to segments $DF$ and $DE$ at $M$ and $N$, respectively. Let line $MN$ meet circles $\omega_B$ and $\omega_C$ again at $P \ne M$ and $Q \ne N$, respectively. Prove that $MP = NQ$.
\end{problem}
\begin{problem}
Find all pairs $(k,n)$ of positive integers such that\[ k!=(2^n-1)(2^n-2)(2^n-4)\cdots(2^n-2^{n-1}). \]
\end{problem}
\begin{problem}
	A hunter and an invisible rabbit play a game on an infinite square grid. First the hunter fixes a colouring of the cells with finitely many colours. The rabbit then secretly chooses a cell to start in. Every minute, the rabbit reports the colour of its current cell to the hunter, and then secretly moves to an adjacent cell that it has not visited before (two cells are adjacent if they share an edge). The hunter wins if after some finite time either:
the rabbit cannot move; or
the hunter can determine the cell in which the rabbit started.
Decide whether there exists a winning strategy for the hunter.
\end{problem}
\begin{problem}
For every positive integer $N$, determine the smallest real number $b_{N}$ such that, for all real $x$,
\[
\sqrt[N]{\frac{x^{2 N}+1}{2}} \leqslant b_{N}(x-1)^{2}+x .
\]
\end{problem}
\begin{problem}
Let $a_1$, $a_2$, $\ldots$ be an infinite sequence of positive integers. Suppose that there is an integer $N > 1$ such that, for each $n \geq N$, the number
$$\frac{a_1}{a_2} + \frac{a_2}{a_3} + \cdots + \frac{a_{n-1}}{a_n} + \frac{a_n}{a_1}$$is an integer. Prove that there is a positive integer $M$ such that $a_m = a_{m+1}$ for all $m \geq M$.
\end{problem}
\begin{problem}
Which positive integers $n$ make the equation\[\sum_{i=1}^n \sum_{j=1}^n \left\lfloor \frac{ij}{n+1} \right\rfloor=\frac{n^2(n-1)}{4}\]true?
\end{problem}
\begin{problem}
	You are given a set of $n$ blocks, each weighing at least $1$; their total weight is $2n$. Prove that for every real number $r$ with $0 \leq r \leq 2n-2$ you can choose a subset of the blocks whose total weight is at least $r$ but at most $r + 2$.
\end{problem}
\begin{problem}
Let $n \geqslant 3$ be a positive integer and let $\left(a_{1}, a_{2}, \ldots, a_{n}\right)$ be a strictly increasing sequence of $n$ positive real numbers with sum equal to 2. Let $X$ be a subset of $\{1,2, \ldots, n\}$ such that the value of
\[
\left|1-\sum_{i \in X} a_{i}\right|
\]is minimised. Prove that there exists a strictly increasing sequence of $n$ positive real numbers $\left(b_{1}, b_{2}, \ldots, b_{n}\right)$ with sum equal to 2 such that
\[
\sum_{i \in X} b_{i}=1.
\]
\end{problem}
\begin{problem}
Let $n > 3$ be a positive integer. Suppose that $n$ children are arranged in a circle, and $n$ coins are distributed between them (some children may have no coins). At every step, a child with at least 2 coins may give 1 coin to each of their immediate neighbors on the right and left. Determine all initial distributions of the coins from which it is possible that, after a finite number of steps, each child has exactly one coin.
\end{problem}
\begin{problem}
Let $u_1, u_2, \dots, u_{2019}$ be real numbers satisfying\[u_{1}+u_{2}+\cdots+u_{2019}=0 \quad \text { and } \quad u_{1}^{2}+u_{2}^{2}+\cdots+u_{2019}^{2}=1.\]Let $a=\min \left(u_{1}, u_{2}, \ldots, u_{2019}\right)$ and $b=\max \left(u_{1}, u_{2}, \ldots, u_{2019}\right)$. Prove that
\[
a b \leqslant-\frac{1}{2019}.
\]
\end{problem}
\begin{problem}
Let $n$ and $k$ be two integers with $n>k\geqslant 1$. There are $2n+1$ students standing in a circle. Each student $S$ has $2k$ neighbors - namely, the $k$ students closest to $S$ on the left, and the $k$ students closest to $S$ on the right.

Suppose that $n+1$ of the students are girls, and the other $n$ are boys. Prove that there is a girl with at least $k$ girls among her neighbors.
\end{problem}
\begin{problem}
Let $n \geqslant 3$ be an integer, and let $x_1,x_2,\ldots,x_n$ be real numbers in the interval $[0,1]$. Let $s=x_1+x_2+\ldots+x_n$, and assume that $s \geqslant 3$. Prove that there exist integers $i$ and $j$ with $1 \leqslant i<j \leqslant n$ such that
\[2^{j-i}x_ix_j>2^{s-3}.\]
\end{problem}
\begin{problem}
	In a regular 100-gon, 41 vertices are colored black and the remaining 59 vertices are colored white. Prove that there exist 24 convex quadrilaterals $Q_{1}, \ldots, Q_{24}$ whose corners are vertices of the 100-gon, so that
the quadrilaterals $Q_{1}, \ldots, Q_{24}$ are pairwise disjoint, and
every quadrilateral $Q_{i}$ has three corners of one color and one corner of the other color.
\end{problem}
\begin{problem}
	An anti-Pascal triangle is an equilateral triangular array of numbers such that, except for the numbers in the bottom row, each number is the absolute value of the difference of the two numbers immediately below it. For example, the following is an anti-Pascal triangle with four rows which contains every integer from $1$ to $10$.
\[\begin{array}{
c@{\hspace{4pt}}c@{\hspace{4pt}}
c@{\hspace{4pt}}c@{\hspace{2pt}}c@{\hspace{2pt}}c@{\hspace{4pt}}c
} \vspace{4pt}
 & & & 4 & & &  \\\vspace{4pt}
 & & 2 & & 6 & &  \\\vspace{4pt}
 & 5 & & 7 & & 1 & \\\vspace{4pt}
 8 & & 3 & & 10 & & 9 \\\vspace{4pt}
\end{array}\]Does there exist an anti-Pascal triangle with $2018$ rows which contains every integer from $1$ to $1 + 2 + 3 + \dots + 2018$?

\end{problem}
\begin{problem}
Show that $n!=a^{n-1}+b^{n-1}+c^{n-1}$ has only finitely many solutions in positive integers.
\end{problem}
\begin{problem}
Let $f : \{ 1, 2, 3, \dots \} \to \{ 2, 3, \dots \}$ be a function such that $f(m + n) | f(m) + f(n) $ for all pairs $m,n$ of positive integers. Prove that there exists a positive integer $c > 1$ which divides all values of $f$.
\end{problem}
\begin{problem}
	Let $r>1$ be a rational number. Alice plays a solitaire game on a number line. Initially there is a red bead at $0$ and a blue bead at $1$. In a move, Alice chooses one of the beads and an integer $k \in \mathbb{Z}$. If the chosen bead is at $x$, and the other bead is at $y$, then the bead at $x$ is moved to the point $x'$ satisfying $x'-y=r^k(x-y)$.

Find all $r$ for which Alice can move the red bead to $1$ in at most $2021$ moves.
\end{problem}
\begin{problem}
Find all positive integers $n \geqslant 2$ for which there exist $n$ real numbers $a_1<\cdots< a_n$ and a real number $r>0$ such that the $\tfrac{1}{2}n(n-1)$ differences $a_j-a_i$ for $1 \leqslant i<j \leqslant n$ are equal, in some order, to the numbers $r^1,r^2,\ldots,r^{\frac{1}{2}n(n-1)}$.
\end{problem}
\begin{problem}
	Determine all functions $f$ defined on the set of all positive integers and taking non-negative integer values, satisfying the three conditions:
$(i)$ $f(n) \neq 0$ for at least one $n$;
$(ii)$ $f(x y)=f(x)+f(y)$ for every positive integers $x$ and $y$;
$(iii)$ there are infinitely many positive integers $n$ such that $f(k)=f(n-k)$ for all $k<n$.
\end{problem}
\begin{problem}
Let $n\ge 3$ be a fixed integer. There are $m\ge n+1$ beads on a circular necklace. You wish to paint the beads using $n$ colors, such that among any $n+1$ consecutive beads every color appears at least once. Find the largest value of $m$ for which this task is $\emph{not}$ possible.
\end{problem}
\begin{problem}
A point $T$ is chosen inside a triangle $ABC$. Let $A_1$, $B_1$, and $C_1$ be the reflections of $T$ in $BC$, $CA$, and $AB$, respectively. Let $\Omega$ be the circumcircle of the triangle $A_1B_1C_1$. The lines $A_1T$, $B_1T$, and $C_1T$ meet $\Omega$ again at $A_2$, $B_2$, and $C_2$, respectively. Prove that the lines $AA_2$, $BB_2$, and $CC_2$ are concurrent on $\Omega$.
\end{problem}
\begin{problem}
Consider a $100\times 100$ square unit lattice $\textbf{L}$ (hence $\textbf{L}$ has $10000$ points). Suppose $\mathcal{F}$ is a set of polygons such that all vertices of polygons in $\mathcal{F}$ lie in $\textbf{L}$ and every point in $\textbf{L}$ is the vertex of exactly one polygon in $\mathcal{F}.$ Find the maximum possible sum of the areas of the polygons in $\mathcal{F}.$
\end{problem}
\begin{problem}
Let $(a_n)_{n\geq 1}$ be a sequence of positive real numbers with the property that
$$(a_{n+1})^2 + a_na_{n+2} \leq a_n + a_{n+2}$$for all positive integers $n$. Show that $a_{2022}\leq 1$.
\end{problem}
\begin{problem}
	A site is any point $(x, y)$ in the plane such that $x$ and $y$ are both positive integers less than or equal to 20.

Initially, each of the 400 sites is unoccupied. Amy and Ben take turns placing stones with Amy going first. On her turn, Amy places a new red stone on an unoccupied site such that the distance between any two sites occupied by red stones is not equal to $\sqrt{5}$. On his turn, Ben places a new blue stone on any unoccupied site. (A site occupied by a blue stone is allowed to be at any distance from any other occupied site.) They stop as soon as a player cannot place a stone.

Find the greatest $K$ such that Amy can ensure that she places at least $K$ red stones, no matter how Ben places his blue stones.
\end{problem}
\begin{problem}
Let $k\ge2$ be an integer. Find the smallest integer $n \ge k+1$ with the property that there exists a set of $n$ distinct real numbers such that each of its elements can be written as a sum of $k$ other distinct elements of the set.
\end{problem}
\begin{problem}
Let $\mathbb{R}^+$ denote the set of positive real numbers. Find all functions $f: \mathbb{R}^+ \to \mathbb{R}^+$ such that for each $x \in \mathbb{R}^+$, there is exactly one $y \in \mathbb{R}^+$ satisfying$$xf(y)+yf(x) \leq 2$$
\end{problem}
\begin{problem}
Let $\Gamma$ be a circle with centre $I$, and $A B C D$ a convex quadrilateral such that each of the segments $A B, B C, C D$ and $D A$ is tangent to $\Gamma$. Let $\Omega$ be the circumcircle of the triangle $A I C$. The extension of $B A$ beyond $A$ meets $\Omega$ at $X$, and the extension of $B C$ beyond $C$ meets $\Omega$ at $Z$. The extensions of $A D$ and $C D$ beyond $D$ meet $\Omega$ at $Y$ and $T$, respectively. Prove that\[A D+D T+T X+X A=C D+D Y+Y Z+Z C.\]
\end{problem}
\begin{problem}
	Find all positive integers $n$ with the following property: the $k$ positive divisors of $n$ have a permutation $(d_1,d_2,\ldots,d_k)$ such that for $i=1,2,\ldots,k$, the number $d_1+d_2+\cdots+d_i$ is a perfect square.
\end{problem}
\begin{problem}
	The real numbers $a, b, c, d$ are such that $a\geq b\geq c\geq d>0$ and $a+b+c+d=1$. Prove that
\[(a+2b+3c+4d)a^ab^bc^cd^d<1\]
\end{problem}
\begin{problem}
Let $ABCD$ be a cyclic quadrilateral. Assume that the points $Q, A, B, P$ are collinear in this order, in such a way that the line $AC$ is tangent to the circle $ADQ$, and the line $BD$ is tangent to the circle $BCP$. Let $M$ and $N$ be the midpoints of segments $BC$ and $AD$, respectively. Prove that the following three lines are concurrent: line $CD$, the tangent of circle $ANQ$ at point $A$, and the tangent to circle $BMP$ at point $B$.
\end{problem}
\begin{problem}
For each integer $n\ge 1,$ compute the smallest possible value of\[\sum_{k=1}^{n}\left\lfloor\frac{a_k}{k}\right\rfloor\]over all permutations $(a_1,\dots,a_n)$ of $\{1,\dots,n\}.$
\end{problem}
\begin{problem}
Consider the convex quadrilateral $ABCD$. The point $P$ is in the interior of $ABCD$. The following ratio equalities hold:
\[\angle PAD:\angle PBA:\angle DPA=1:2:3=\angle CBP:\angle BAP:\angle BPC\]Prove that the following three lines meet in a point: the internal bisectors of angles $\angle ADP$ and $\angle PCB$ and the perpendicular bisector of segment $AB$.
\end{problem}
\begin{problem}
There is an integer $n > 1$. There are $n^2$ stations on a slope of a mountain, all at different altitudes. Each of two cable car companies, $A$ and $B$, operates $k$ cable cars; each cable car provides a transfer from one of the stations to a higher one (with no intermediate stops). The $k$ cable cars of $A$ have $k$ different starting points and $k$ different finishing points, and a cable car which starts higher also finishes higher. The same conditions hold for $B$. We say that two stations are linked by a company if one can start from the lower station and reach the higher one by using one or more cars of that company (no other movements between stations are allowed). Determine the smallest positive integer $k$ for which one can guarantee that there are two stations that are linked by both companies.
\end{problem}
\begin{problem}
Let $n \geqslant 100$ be an integer. Ivan writes the numbers $n, n+1, \ldots, 2 n$ each on different cards. He then shuffles these $n+1$ cards, and divides them into two piles. Prove that at least one of the piles contains two cards such that the sum of their numbers is a perfect square.
\end{problem}
\begin{problem}
The Bank of Oslo issues two types of coin: aluminum (denoted A) and bronze (denoted B). Marianne has $n$ aluminum coins and $n$ bronze coins arranged in a row in some arbitrary initial order. A chain is any subsequence of consecutive coins of the same type. Given a fixed positive integer $k \leq 2n$, Gilberty repeatedly performs the following operation: he identifies the longest chain containing the $k^{th}$ coin from the left and moves all coins in that chain to the left end of the row. For example, if $n=4$ and $k=4$, the process starting from the ordering $AABBBABA$ would be $AABBBABA \to BBBAAABA \to AAABBBBA \to BBBBAAAA \to ...$

Find all pairs $(n,k)$ with $1 \leq k \leq 2n$ such that for every initial ordering, at some moment during the process, the leftmost $n$ coins will all be of the same type.
\end{problem}
\begin{problem}
For each prime $p$, construct a graph $G_p$ on $\{1,2,\ldots p\}$, where $m\neq n$ are adjacent if and only if $p$ divides $(m^{2} + 1-n)(n^{2} + 1-m)$. Prove that $G_p$ is disconnected for infinitely many $p$
\end{problem}
\begin{problem}
Four positive integers $x,y,z$ and $t$ satisfy the relations
\[ xy - zt = x + y = z + t. \]Is it possible that both $xy$ and $zt$ are perfect squares?
\end{problem}
\begin{problem}
Let $n$ be a positive integer. Find the number of permutations $a_1$, $a_2$, $\dots a_n$ of the
sequence $1$, $2$, $\dots$ , $n$ satisfying
$$a_1 \le 2a_2\le 3a_3 \le \dots \le na_n$$
\end{problem}
\begin{problem}
Show that the inequality\[\sum_{i=1}^n \sum_{j=1}^n \sqrt{|x_i-x_j|}\leqslant \sum_{i=1}^n \sum_{j=1}^n \sqrt{|x_i+x_j|}\]holds for all real numbers $x_1,\ldots x_n.$
\end{problem}
\begin{problem}
	We say that a set $S$ of integers is rootiful if, for any positive integer $n$ and any $a_0, a_1, \cdots, a_n \in S$, all integer roots of the polynomial $a_0+a_1x+\cdots+a_nx^n$ are also in $S$. Find all rootiful sets of integers that contain all numbers of the form $2^a - 2^b$ for positive integers $a$ and $b$.
\end{problem}
\begin{problem}
In the acute-angled triangle $ABC$, the point $F$ is the foot of the altitude from $A$, and $P$ is a point on the segment $AF$. The lines through $P$ parallel to $AC$ and $AB$ meet $BC$ at $D$ and $E$, respectively. Points $X \ne A$ and $Y \ne A$ lie on the circles $ABD$ and $ACE$, respectively, such that $DA = DX$ and $EA = EY$.
Prove that $B, C, X,$ and $Y$ are concyclic.
\end{problem}
\begin{problem}
A magician intends to perform the following trick. She announces a positive integer $n$, along with $2n$ real numbers $x_1 < \dots < x_{2n}$, to the audience. A member of the audience then secretly chooses a polynomial $P(x)$ of degree $n$ with real coefficients, computes the $2n$ values $P(x_1), \dots , P(x_{2n})$, and writes down these $2n$ values on the blackboard in non-decreasing order. After that the magician announces the secret polynomial to the audience. Can the magician find a strategy to perform such a trick?
\end{problem}
\begin{problem}
	The Fibonacci numbers $F_0, F_1, F_2, . . .$ are defined inductively by $F_0=0, F_1=1$, and $F_{n+1}=F_n+F_{n-1}$ for $n \ge 1$. Given an integer $n \ge 2$, determine the smallest size of a set $S$ of integers such that for every $k=2, 3, . . . , n$ there exist some $x, y \in S$ such that $x-y=F_k$.
\end{problem}
\begin{problem}
In triangle $ABC$, point $A_1$ lies on side $BC$ and point $B_1$ lies on side $AC$. Let $P$ and $Q$ be points on segments $AA_1$ and $BB_1$, respectively, such that $PQ$ is parallel to $AB$. Let $P_1$ be a point on line $PB_1$, such that $B_1$ lies strictly between $P$ and $P_1$, and $\angle PP_1C=\angle BAC$. Similarly, let $Q_1$ be the point on line $QA_1$, such that $A_1$ lies strictly between $Q$ and $Q_1$, and $\angle CQ_1Q=\angle CBA$.

Prove that points $P,Q,P_1$, and $Q_1$ are concyclic.
\end{problem}
\begin{problem}
Given any set $S$ of positive integers, show that at least one of the following two assertions holds:

(1) There exist distinct finite subsets $F$ and $G$ of $S$ such that $\sum_{x\in F}1/x=\sum_{x\in G}1/x$;

(2) There exists a positive rational number $r<1$ such that $\sum_{x\in F}1/x\neq r$ for all finite subsets $F$ of $S$.
\end{problem}
\begin{problem}
	Let $a_0,a_1,a_2,\dots $ be a sequence of real numbers such that $a_0=0, a_1=1,$ and for every $n\geq 2$ there exists $1 \leq k \leq n$ satisfying\[ a_n=\frac{a_{n-1}+\dots + a_{n-k}}{k}. \]Find the maximum possible value of $a_{2018}-a_{2017}$.
\end{problem}
\begin{problem}
	For any odd prime $p$ and any integer $n,$ let $d_p (n) \in \{ 0,1, \dots, p-1 \}$ denote the remainder when $n$ is divided by $p.$ We say that $(a_0, a_1, a_2, \dots)$ is a p-sequence, if $a_0$ is a positive integer coprime to $p,$ and $a_{n+1} =a_n + d_p (a_n)$ for $n \geqslant 0.$
(a) Do there exist infinitely many primes $p$ for which there exist $p$-sequences $(a_0, a_1, a_2, \dots)$ and $(b_0, b_1, b_2, \dots)$ such that $a_n >b_n$ for infinitely many $n,$ and $b_n > a_n$ for infinitely many $n?$
(b) Do there exist infinitely many primes $p$ for which there exist $p$-sequences $(a_0, a_1, a_2, \dots)$ and $(b_0, b_1, b_2, \dots)$ such that $a_0 <b_0,$ but $a_n >b_n$ for all $n \geqslant 1?$
\end{problem}
\begin{problem}
	Let $a$ be a positive integer. We say that a positive integer $b$ is $a$-good if $\tbinom{an}{b}-1$ is divisible by $an+1$ for all positive integers $n$ with $an \geq b$. Suppose $b$ is a positive integer such that $b$ is $a$-good, but $b+2$ is not $a$-good. Prove that $b+1$ is prime.
\end{problem}
\begin{problem}
	Define the sequence $a_0,a_1,a_2,\hdots$ by $a_n=2^n+2^{\lfloor n/2\rfloor}$. Prove that there are infinitely many terms of the sequence which can be expressed as a sum of (two or more) distinct terms of the sequence, as well as infinitely many of those which cannot be expressed in such a way.
\end{problem}
\begin{problem}
Given a positive integer $k$ show that there exists a prime $p$ such that one can choose distinct integers $a_1,a_2\cdots, a_{k+3} \in \{1, 2, \cdots ,p-1\}$ such that p divides $a_ia_{i+1}a_{i+2}a_{i+3}-i$ for all $i= 1, 2, \cdots, k$.
\end{problem}
\begin{problem}
	Let $n\geqslant 3$ be an integer. Prove that there exists a set $S$ of $2n$ positive integers satisfying the following property: For every $m=2,3,...,n$ the set $S$ can be partitioned into two subsets with equal sums of elements, with one of subsets of cardinality $m$.
\end{problem}
\begin{problem}
Let $ABCD$ be a convex quadrilateral with $\angle ABC>90$, $CDA>90$ and $\angle DAB=\angle BCD$. Denote by $E$ and $F$ the reflections of $A$ in lines $BC$ and $CD$, respectively. Suppose that the segments $AE$ and $AF$ meet the line $BD$ at $K$ and $L$, respectively. Prove that the circumcircles of triangles $BEK$ and $DFL$ are tangent to each other.
\end{problem}
\begin{problem}
	Find all positive integers $n\geq1$ such that there exists a pair $(a,b)$ of positive integers, such that $a^2+b+3$ is not divisible by the cube of any prime, and$$n=\frac{ab+3b+8}{a^2+b+3}.$$
\end{problem}
\begin{problem}
Let $ABCD$ be a cyclic quadrilateral. Points $K, L, M, N$ are chosen on $AB, BC, CD, DA$ such that $KLMN$ is a rhombus with $KL \parallel AC$ and $LM \parallel BD$. Let $\omega_A, \omega_B, \omega_C, \omega_D$ be the incircles of $\triangle ANK, \triangle BKL, \triangle CLM, \triangle DMN$.

Prove that the common internal tangents to $\omega_A$, and $\omega_C$ and the common internal tangents to $\omega_B$ and $\omega_D$ are concurrent.
\end{problem}
\begin{problem}
Let $x_1, x_2, \dots, x_n$ be different real numbers. Prove that
\[\sum_{1 \leqslant i \leqslant n} \prod_{j \neq i} \frac{1-x_{i} x_{j}}{x_{i}-x_{j}}=\left\{\begin{array}{ll}
0, & \text { if } n \text { is even; } \\
1, & \text { if } n \text { is odd. }
\end{array}\right.\]
\end{problem}
\begin{problem}
	The Bank of Bath issues coins with an $H$ on one side and a $T$ on the other. Harry has $n$ of these coins arranged in a line from left to right. He repeatedly performs the following operation: if there are exactly $k>0$ coins showing $H$, then he turns over the $k$th coin from the left; otherwise, all coins show $T$ and he stops. For example, if $n=3$ the process starting with the configuration $THT$ would be $THT \to HHT  \to HTT \to TTT$, which stops after three operations.

(a) Show that, for each initial configuration, Harry stops after a finite number of operations.

(b) For each initial configuration $C$, let $L(C)$ be the number of operations before Harry stops. For example, $L(THT) = 3$ and $L(TTT) = 0$. Determine the average value of $L(C)$ over all $2^n$ possible initial configurations $C$.
\end{problem}
\begin{problem}
	A deck of $n > 1$ cards is given. A positive integer is written on each card. The deck has the property that the arithmetic mean of the numbers on each pair of cards is also the geometric mean of the numbers on some collection of one or more cards.
For which $n$ does it follow that the numbers on the cards are all equal?
\end{problem}
\begin{problem}
In each square of a garden shaped like a $2022 \times 2022$ board, there is initially a tree of height $0$. A gardener and a lumberjack alternate turns playing the following game, with the gardener taking the first turn:
The gardener chooses a square in the garden. Each tree on that square and all the surrounding squares (of which there are at most eight) then becomes one unit taller.
The lumberjack then chooses four different squares on the board. Each tree of positive height on those squares then becomes one unit shorter.
We say that a tree is majestic if its height is at least $10^6$. Determine the largest $K$ such that the gardener can ensure there are eventually $K$ majestic trees on the board, no matter how the lumberjack plays.
\end{problem}
\begin{problem}
Let $ABC$ be an isosceles triangle with $BC=CA$, and let $D$ be a point inside side $AB$ such that $AD< DB$. Let $P$ and $Q$ be two points inside sides $BC$ and $CA$, respectively, such that $\angle DPB = \angle DQA = 90^{\circ}$. Let the perpendicular bisector of $PQ$ meet line segment $CQ$ at $E$, and let the circumcircles of triangles $ABC$ and $CPQ$ meet again at point $F$, different from $C$.
Suppose that $P$, $E$, $F$ are collinear. Prove that $\angle ACB = 90^{\circ}$.
\end{problem}
\begin{problem}
Let $ABC$ be a triangle with $AB=AC$, and let $M$ be the midpoint of $BC$. Let $P$ be a point such that $PB<PC$ and $PA$ is parallel to $BC$. Let $X$ and $Y$ be points on the lines $PB$ and $PC$, respectively, so that $B$ lies on the segment $PX$, $C$ lies on the segment $PY$, and $\angle PXM=\angle PYM$. Prove that the quadrilateral $APXY$ is cyclic.
\end{problem}
\begin{problem}
On a flat plane in Camelot, King Arthur builds a labyrinth $\mathfrak{L}$ consisting of $n$ walls, each of which is an infinite straight line. No two walls are parallel, and no three walls have a common point. Merlin then paints one side of each wall entirely red and the other side entirely blue.

At the intersection of two walls there are four corners: two diagonally opposite corners where a red side and a blue side meet, one corner where two red sides meet, and one corner where two blue sides meet. At each such intersection, there is a two-way door connecting the two diagonally opposite corners at which sides of different colours meet.

After Merlin paints the walls, Morgana then places some knights in the labyrinth. The knights can walk through doors, but cannot walk through walls.

Let $k(\mathfrak{L})$ be the largest number $k$ such that, no matter how Merlin paints the labyrinth $\mathfrak{L},$ Morgana can always place at least $k$ knights such that no two of them can ever meet. For each $n,$ what are all possible values for $k(\mathfrak{L}),$ where $\mathfrak{L}$ is a labyrinth with $n$ walls?
\end{problem}
\begin{problem}
In the plane, there are $n \geqslant 6$ pairwise disjoint disks $D_{1}, D_{2}, \ldots, D_{n}$ with radii $R_{1} \geqslant R_{2} \geqslant \ldots \geqslant R_{n}$. For every $i=1,2, \ldots, n$, a point $P_{i}$ is chosen in disk $D_{i}$. Let $O$ be an arbitrary point in the plane. Prove that\[O P_{1}+O P_{2}+\ldots+O P_{n} \geqslant R_{6}+R_{7}+\ldots+R_{n}.\](A disk is assumed to contain its boundary.)
\end{problem}
\begin{problem}
	Let $k$ be a positive integer. The organising commitee of a tennis tournament is to schedule the matches for $2k$ players so that every two players play once, each day exactly one match is played, and each player arrives to the tournament site the day of his first match, and departs the day of his last match. For every day a player is present on the tournament, the committee has to pay $1$ coin to the hotel. The organisers want to design the schedule so as to minimise the total cost of all players' stays. Determine this minimum cost.
\end{problem}
\begin{problem}
Let $ABCDE$ be a convex pentagon with $CD= DE$ and $\angle EDC \ne 2 \cdot \angle ADB$.
Suppose that a point $P$ is located in the interior of the pentagon such that $AP =AE$ and $BP= BC$.
Prove that $P$ lies on the diagonal $CE$ if and only if area $(BCD)$ + area $(ADE)$ = area $(ABD)$ + area $(ABP)$.
\end{problem}
\begin{problem}
Find all functions $f:\mathbb Z_{>0}\to \mathbb Z_{>0}$ such that $a+f(b)$ divides $a^2+bf(a)$ for all positive integers $a$ and $b$ with $a+b>2019$.
\end{problem}
\begin{problem}
	Let $ABC$ be a triangle. Circle $\Gamma$ passes through $A$, meets segments $AB$ and $AC$ again at points $D$ and $E$ respectively, and intersects segment $BC$ at $F$ and $G$ such that $F$ lies between $B$ and $G$. The tangent to circle $BDF$ at $F$ and the tangent to circle $CEG$ at $G$ meet at point $T$. Suppose that points $A$ and $T$ are distinct. Prove that line $AT$ is parallel to $BC$.
\end{problem}
\begin{problem}
	A $\pm 1$-sequence is a sequence of $2022$ numbers $a_1, \ldots, a_{2022},$ each equal to either $+1$ or $-1$. Determine the largest $C$ so that, for any $\pm 1$-sequence, there exists an integer $k$ and indices $1 \le t_1 < \ldots < t_k \le 2022$ so that $t_{i+1} - t_i \le 2$ for all $i$, and$$\left| \sum_{i = 1}^{k} a_{t_i} \right| \ge C.$$
\end{problem}
\begin{problem}
Let $n$ be a positive integer, and set $N=2^{n}$. Determine the smallest real number $a_{n}$ such that, for all real $x$,
\[
\sqrt[N]{\frac{x^{2 N}+1}{2}} \leqslant a_{n}(x-1)^{2}+x .
\]
\end{problem}
\begin{problem}
Let $\mathcal{A}$ denote the set of all polynomials in three variables $x, y, z$ with integer coefficients. Let $\mathcal{B}$ denote the subset of $\mathcal{A}$ formed by all polynomials which can be expressed as
\begin{align*}
(x + y + z)P(x, y, z) + (xy + yz + zx)Q(x, y, z) + xyzR(x, y, z)
\end{align*}with $P, Q, R \in \mathcal{A}$. Find the smallest non-negative integer $n$ such that $x^i y^j z^k \in \mathcal{B}$ for all non-negative integers $i, j, k$ satisfying $i + j + k \geq n$.
\end{problem}
\begin{problem}
Let $\mathbb{Q}_{>0}$ denote the set of all positive rational numbers. Determine all functions $f:\mathbb{Q}_{>0}\to \mathbb{Q}_{>0}$ satisfying$$f(x^2f(y)^2)=f(x)^2f(y)$$for all $x,y\in\mathbb{Q}_{>0}$
\end{problem}
\begin{problem}
	Let $ABCD$ be a quadrilateral inscribed in a circle $\Omega.$ Let the tangent to $\Omega$ at $D$ meet rays $BA$ and $BC$ at $E$ and $F,$ respectively. A point $T$ is chosen inside $\triangle ABC$ so that $\overline{TE}\parallel\overline{CD}$ and $\overline{TF}\parallel\overline{AD}.$ Let $K\ne D$ be a point on segment $DF$ satisfying $TD=TK.$ Prove that lines $AC,DT,$ and $BK$ are concurrent.
\end{problem}
\begin{problem}
Find all positive integers $n>2$ such that
$$ n! \mid \prod_{ p<q\le n, p,q \, \text{primes}} (p+q)$$
\end{problem}
\begin{problem}
	The kingdom of Anisotropy consists of $n$ cities. For every two cities there exists exactly one direct one-way road between them. We say that a path from $X$ to $Y$ is a sequence of roads such that one can move from $X$ to $Y$ along this sequence without returning to an already visited city. A collection of paths is called diverse if no road belongs to two or more paths in the collection.

Let $A$ and $B$ be two distinct cities in Anisotropy. Let $N_{AB}$ denote the maximal number of paths in a diverse collection of paths from $A$ to $B$. Similarly, let $N_{BA}$ denote the maximal number of paths in a diverse collection of paths from $B$ to $A$. Prove that the equality $N_{AB} = N_{BA}$ holds if and only if the number of roads going out from $A$ is the same as the number of roads going out from $B$.
\end{problem}
\begin{problem}
Let $n$ be a positive integer. Given is a subset $A$ of $\{0,1,...,5^n\}$ with $4n+2$ elements. Prove that there exist three elements $a<b<c$ from $A$ such that $c+2a>3b$.
\end{problem}
\begin{problem}
	Let $\Gamma$ be the circumcircle of acute triangle $ABC$. Points $D$ and $E$ are on segments $AB$ and $AC$ respectively such that $AD = AE$. The perpendicular bisectors of $BD$ and $CE$ intersect minor arcs $AB$ and $AC$ of $\Gamma$ at points $F$ and $G$ respectively. Prove that lines $DE$ and $FG$ are either parallel or they are the same line.
\end{problem}
\begin{problem}
	Let $n$ be a given positive integer. Sisyphus performs a sequence of turns on a board consisting of $n + 1$ squares in a row, numbered $0$ to $n$ from left to right. Initially, $n$ stones are put into square $0$, and the other squares are empty. At every turn, Sisyphus chooses any nonempty square, say with $k$ stones, takes one of these stones and moves it to the right by at most $k$ squares (the stone should say within the board). Sisyphus' aim is to move all $n$ stones to square $n$.
Prove that Sisyphus cannot reach the aim in less than
\[ \left \lceil \frac{n}{1} \right \rceil + \left \lceil \frac{n}{2} \right \rceil + \left \lceil \frac{n}{3} \right \rceil + \dots + \left \lceil \frac{n}{n} \right \rceil \]turns. (As usual, $\lceil x \rceil$ stands for the least integer not smaller than $x$. )
\end{problem}
\begin{problem}
Find all integers $n \geq 3$ for which there exist real numbers $a_1, a_2, \dots a_{n + 2}$ satisfying $a_{n + 1} = a_1$, $a_{n + 2} = a_2$ and
$$a_ia_{i + 1} + 1 = a_{i + 2},$$for $i = 1, 2, \dots, n$.
\end{problem}
\begin{problem}
Determine all functions $f:(0,\infty)\to\mathbb{R}$ satisfying$$\left(x+\frac{1}{x}\right)f(y)=f(xy)+f\left(\frac{y}{x}\right)$$for all $x,y>0$.
\end{problem}
\begin{problem}
	A circle $\omega$ with radius $1$ is given. A collection $T$ of triangles is called good, if the following conditions hold:
each triangle from $T$ is inscribed in $\omega$;
no two triangles from $T$ have a common interior point.
Determine all positive real numbers $t$ such that, for each positive integer $n$, there exists a good collection of $n$ triangles, each of perimeter greater than $t$.
\end{problem}
\begin{problem}
Let $n>1$ be a positive integer. Each cell of an $n\times n$ table contains an integer. Suppose that the following conditions are satisfied:
Each number in the table is congruent to $1$ modulo $n$.
The sum of numbers in any row, as well as the sum of numbers in any column, is congruent to $n$ modulo $n^2$.
Let $R_i$ be the product of the numbers in the $i^{\text{th}}$ row, and $C_j$ be the product of the number in the $j^{\text{th}}$ column. Prove that the sums $R_1+\hdots R_n$ and $C_1+\hdots C_n$ are congruent modulo $n^4$.
\end{problem}
\begin{problem}
Let $\mathbb{Z}$ be the set of integers. Determine all functions $f: \mathbb{Z} \rightarrow \mathbb{Z}$ such that, for all integers $a$ and $b$,$$f(2a)+2f(b)=f(f(a+b)).$$
\end{problem}
\begin{problem}
	Let $ABCD$ be a cyclic quadrilateral whose sides have pairwise different lengths. Let $O$ be the circumcenter of $ABCD$. The internal angle bisectors of $\angle ABC$ and $\angle ADC$ meet $AC$ at $B_1$ and $D_1$, respectively. Let $O_B$ be the center of the circle which passes through $B$ and is tangent to $\overline{AC}$ at $D_1$. Similarly, let $O_D$ be the center of the circle which passes through $D$ and is tangent to $\overline{AC}$ at $B_1$.

Assume that $\overline{BD_1} \parallel \overline{DB_1}$. Prove that $O$ lies on the line $\overline{O_BO_D}$.
\end{problem}
\begin{problem}
The infinite sequence $a_0,a _1, a_2, \dots$ of (not necessarily distinct) integers has the following properties: $0\le a_i \le i$ for all integers $i\ge 0$, and\[\binom{k}{a_0} + \binom{k}{a_1} + \dots + \binom{k}{a_k} = 2^k\]for all integers $k\ge 0$. Prove that all integers $N\ge 0$ occur in the sequence (that is, for all $N\ge 0$, there exists $i\ge 0$ with $a_i=N$).
\end{problem}
\begin{problem}
Let $n\geq 2$ be an integer and let $a_1, a_2, \ldots, a_n$ be positive real numbers with sum $1$. Prove that$$\sum_{k=1}^n \frac{a_k}{1-a_k}(a_1+a_2+\cdots+a_{k-1})^2 < \frac{1}{3}.$$
\end{problem}
\begin{problem}
	For each $1\leq i\leq 9$ and $T\in\mathbb N$, define $d_i(T)$ to be the total number of times the digit $i$ appears when all the multiples of $1829$ between $1$ and $T$ inclusive are written out in base $10$.

Show that there are infinitely many $T\in\mathbb N$ such that there are precisely two distinct values among $d_1(T)$, $d_2(T)$, $\dots$, $d_9(T)$
\end{problem}
\begin{problem}
Let $ABCDE$ be a convex pentagon such that $BC=DE$. Assume that there is a point $T$ inside $ABCDE$ with $TB=TD,TC=TE$ and $\angle ABT = \angle TEA$. Let line $AB$ intersect lines $CD$ and $CT$ at points $P$ and $Q$, respectively. Assume that the points $P,B,A,Q$ occur on their line in that order. Let line $AE$ intersect $CD$ and $DT$ at points $R$ and $S$, respectively. Assume that the points $R,E,A,S$ occur on their line in that order. Prove that the points $P,S,Q,R$ lie on a circle.
\end{problem}
\begin{problem}
	Let $p$ be an odd prime, and put $N=\frac{1}{4} (p^3 -p) -1.$ The numbers $1,2, \dots, N$ are painted arbitrarily in two colors, red and blue. For any positive integer $n \leqslant N,$ denote $r(n)$ the fraction of integers $\{ 1,2, \dots, n \}$ that are red.
Prove that there exists a positive integer $a \in \{ 1,2, \dots, p-1\}$ such that $r(n) \neq a/p$ for all $n = 1,2, \dots , N.$
\end{problem}
\begin{problem}
	Let $P$ be a point inside triangle $ABC$. Let $AP$ meet $BC$ at $A_1$, let $BP$ meet $CA$ at $B_1$, and let $CP$ meet $AB$ at $C_1$. Let $A_2$ be the point such that $A_1$ is the midpoint of $PA_2$, let $B_2$ be the point such that $B_1$ is the midpoint of $PB_2$, and let $C_2$ be the point such that $C_1$ is the midpoint of $PC_2$. Prove that points $A_2, B_2$, and $C_2$ cannot all lie strictly inside the circumcircle of triangle $ABC$.
\end{problem}
\begin{problem}
Let $a > 1$ be a positive integer and $d > 1$ be a positive integer coprime to $a$. Let $x_1=1$, and for $k\geq 1$, define
$$x_{k+1} = \begin{cases}
x_k + d &\text{if } a \text{ does not divide } x_k \\
x_k/a & \text{if } a \text{ divides } x_k
\end{cases}$$Find, in terms of $a$ and $d$, the greatest positive integer $n$ for which there exists an index $k$ such that $x_k$ is divisible by $a^n$.
\end{problem}
\begin{problem}
Let $n\geqslant 2$ be a positive integer and $a_1,a_2, \ldots ,a_n$ be real numbers such that\[a_1+a_2+\dots+a_n=0.\]Define the set $A$ by
\[A=\left\{(i, j)\,|\,1 \leqslant i<j \leqslant n,\left|a_{i}-a_{j}\right| \geqslant 1\right\}\]Prove that, if $A$ is not empty, then
\[\sum_{(i, j) \in A} a_{i} a_{j}<0.\]
\end{problem}

\end{document}
